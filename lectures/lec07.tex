%FILL IN THE RIGHT INFO.
%\lecture{**LECTURE-NUMBER**}{**DATE**}
\unchapter{Lecture 7}
\lecture{7}{September 24}
\setcounter{section}{0}
\setcounter{theorem}{0}

% **** YOUR NOTES GO HERE:

We start with a method for bounding the derivatives of a holomorphic function.

\section{Cauchy's Estimates}

double reviewed

\isubsection{THM: Cauchy's Estimates}

\begin{theorem}[Cauchy's Estimates]\label{thm:cauchy-estimates}

$\foc$ holomorphic function on $\oic$. Let $z_0 \in \om, r>0 $ s.t. $D=D_r(z_0) \ssubset \om$. Then $\forall n\geq 0$:  
\begin{align*}
    \abs{f^{(n)}}(z_0) \leq \frac{n!}{r^n} \cdot \sup_{\partial D}\abs{f}
\end{align*}

\end{theorem}


\begin{note}
This is linked closely to the coefficients of the power series representation of $f$. If $r$ is small (usually this is the case) then it gives you an upper bound for the growth of the derivatives, and thus for the coefficients. If $r$ is large then then these coefficients will decay.
\end{note}


\begin{proof}
This is a straightforward application of equation (\ref{eq:cauchy-int-fla}). Parameterize $D$ by $z(t) = z_0 +re^{it}$:
\begin{align*}
    \abs{f^{(n)}} &= \abs{\frac{n!}{2\pi i} \int_{\partial D} \frac{f(z)}{(z-z_0)^{n+1}} \dif z }\\
    &\leq \frac{n!}{2\pi} \abs{\int_{0}^{2\pi} \frac{f(z_0 +re^{it})}{(z_0 +re^{it}-z_0)^{n+1}} \cdot i r e^{it} \dif t }\\
    &\leq \frac{n!}{2\pi} \abs{\int_{0}^{2\pi} \frac{f(z_0 +re^{it})}{r^{n+1}e^{i(n+1)t}} \cdot r e^{it} \dif t }\\
    &\leq  \frac{n!}{2\pi} \abs{\int_{0}^{2\pi} \frac{f(z_0 +re^{it})}{r^{n}e^{int}} \dif t }\\
    &\leq \frac{n!}{2\pi} \int_{0}^{2\pi}  \frac{\abs{ f(z_0 +re^{it})  }}{r^{n}} \dif t \\
    &\leq \frac{n!}{2\pi} \sup_{\partial D}\abs{f} \int_{0}^{2\pi}  \frac{1}{r^{n}} \dif t  = \frac{n!}{r^n} \sup_{\partial D}\abs{f}.
\end{align*}
\end{proof}

\begin{note}
This theorem is the same as putting a bound on $a_n$, with $\abs{a_n} \leq \frac{\sup_{\partial D}\abs{f}}{r^n}$.
\end{note}


\begin{definition}[Entire Functions]
A function $f:\C \to \C$, f holomorphic on $\C$, is called an \textbf{entire} holomorphic function.
\end{definition}

\isubsection{COR: Liouville}

\begin{corollary}[Liouville]\label{cor:liouville}

Consider $f:\C \to \C$ entire holomorphic with $\sup_{z\in \C}\abs{f(z)} \leq C$ for some $C$. Then $f$ is constant.

More generally, $f:\C \to \C$ entire holomorphic with $\abs{f(z)} \leq C\abs{z}^k$ for some $C, k$ and for all $z \in \C$. Then $f$ is a polynomial in $z$ with degree $d\leq k$.

\end{corollary}


\begin{note}
This corollary says that if $f$ entire holomorphic grows slower than a polynomial, it is a polynomial. This implies that if $f$ entire holomorphic is not a polynomial, it must grow faster than any polynomial. Some examples are $e^z$, $\cos(z)$, and $\sin(z)$ (unlike the real case where $\cos(z)$ and $\sin(z)$ are bounded).
\end{note}

\begin{proof}
Since $f$ entire holomorphic, we know that it can be written as a power series that converges everywhere:
\begin{align*}
    f(z) = \sum_{n=1}^\infty a_n z^n \text{ (this converges for all $z\in \C$)}.
\end{align*}

We have an estimate for the growth of this $a_n$; applying theorem (\ref{thm:cauchy-estimates}) on $D_r(0)$ for any $r>0$ gives:
\begin{align*}
    \abs{a_n} = \frac{\abs{f^{(n)} (0)  }}{n!} &\leq \frac{\sup_{\partial D_r(0)}\abs{f}}{r^n}\\
    &\leq \frac{Cr^k}{r^n}.
\end{align*}

If $n>k$ this implies that $\abs{a_n} \xrightarrow[]{r\to \infty} 0$. Thus: $f(z) = \sum_{n=1}^{k} a_n z^n$ and $f$ is a polynomial of degree at most $k$.

\end{proof}

\begin{corollary}[Fundamental Theorem of Algebra]
Given any polynomial $P(z) = \sum_{n=1}^{d} a_n z^n$ of degree $d> 0$, $a_n \in \C $, $a_d \neq 0$. Then $\exists z_0 \in \C$ s.t. $P(z_0) = 0$ (it has a root).
\end{corollary}

\begin{note}
This is equivalent to saying that $\C$ is an algebraically closed field. This is false for $\R$.
\end{note}


\begin{proof}
Suppose that $P$ has no roots. Then the reciprocal is defined. $\frac{1}{P(z)}$ is an entire holomorphic function. We want to bound $\abs{\frac{1}{P(z)}} = \frac{1}{\abs{P(z)}}$. Then $P(z) = z^d( a_d + \frac{a_{d-1}}{z} + \cdots + \frac{a_0}{z^d})$. For $\abs{z} \gg 1$, we can examine $\frac{P(z)}{z^d} = ( a_d + \frac{a_{d-1}}{z} + \cdots + \frac{a_0}{z^d}) \xrightarrow[]{\abs{z}\to \infty} a_d$. That is to say that $\exists R > 0 $ s.t. $\forall \abs{z} > R$ we have:
\begin{align*}
    \abs{\frac{P(z)}{z^d}} &= \frac{\abs{P(z)}}{\abs{z^d}} > \frac{\abs{a_d}}{2}>0.\\
    &\Downarrow\\
    \frac{1}{\abs{P(z)}} &\leq \frac{2}{\abs{a_d}} \cdot \frac{1}{\abs{z}^d} \leq C \: \forall \abs{z} > R.
\end{align*}
    
Then outside the disk $D_R(0)$ $f$ is bounded. $f$ is bounded inside the disk since $\frac{1}{P(z)}$ is holomorphic, thus continuous on $D_R(0)$, and thus bounded inside the compact disk. Thus on $\C$:
\begin{align*}
     \abs{\frac{1}{P(z)}} \leq C.
\end{align*}

By corollary (\ref{cor:liouville}), $\frac{1}{P(z)} $ is a constant. Thus $P(z)$ is a constant. This is a contradiction since we assumed that $P(z) $ had positive degree.
\end{proof} 

\begin{note}
By repeating this argument we can show that $P(z)$ of degree d has exactly d roots in $\C$, possibly non-distinct. That is to say that $P(z) = a_d (z-z_1)\cdots (z-z_d)$.
\end{note}


\isubsection{THM: Zeroes are Isolated}
\begin{theorem}[Zeroes of Holomorphic Functions]\label{thm:isolated-zeroes}

$\foc$ holomorphic, $\om$ open and connected, $f \not\equiv 0$ on $\om$. Then the zeroes for $f$ inside $\om$ are isolated. That is to say that if $f(z_0)=0, \, z_0 \in \om$ then $\exists $V $\subset \om$ open such that $f(z) \neq 0 \, \forall z\in V \setminus \{0\}$.

\end{theorem}

\begin{proof}
Suppose $z_0$ is a non-isolated zero of $f$. We claim that $f\equiv 0$ in some neighbourhood of $z_0$. To show the claim, we let $f(z) = \sum_{n=1}^{\infty} a_n (z-z_0)^n$, which is absolutely convergent near $z_0$. Then if the claim is false, $\exists n >0$ s.t. $a_n\neq 0$. Define $g(z) = a_n+a_{n+1}(z-z_0) + \cdots $. Then:
\begin{align*}
    f(z) &= (z-z_0)^n(a_n+a_{n+1}(z-z_0) + \cdots )\\
    &= (z-z_0)^ng(z).
\end{align*}
$g(z)$ is the power series for $f$, but with coefficients shifted by $n$. Thus $g(z)$ is holomorphic on some small disk centered at $z_0$ (an absolutely convergent power series with shifted coefficients is still absolutely convergent on the same disk). $g(z_0) = a_n \neq 0 \xRightarrow[]{cont} g(z) \neq 0$ for $z$ close to $z_0$. Since $f(z) = (z-z_0)^ng(z)$ we get that $z_0$ cannot be a non-isolated zero of $f$.

We return to the theorem. Let $\om ' = \set{ z \in \om | f \equiv 0 \text{ in some neighbourhood of } z }$. The claim shows that $z_0 \in \om '$, and thus $\om ' \neq \varnothing$. $\om '$ is open (since for any point, $f$ is zero around that point). $\om '$ is also closed.

To see this, suppose you have a sequence $\{ z_n \}_{n=0}^\infty \in \om ', z_n \xrightarrow[]{n\to \infty} z_\infty \in \om $. $f(z_n) = 0 \xRightarrow[]{cont} f(z_\infty) = 0$. Then $z_\infty$ is a non-isolated zero of $f$ (because it is a limit of zeroes). By the claim $f \equiv 0 $ around $z_\infty$, thus $z_\infty \in \om '$. Thus $\om '$ is closed. This implies, applying connectedness, that $\om ' = \om$. Thus $f \equiv 0$ on $\om$, contradiction.

\end{proof} 


\begin{definition}[Order of Vanishing]\label{def:order-vanish}
In the proof we showed that if $f$ holomorphic in $\om \ni z_0, \: f \not\equiv 0$, then we can write $f(z) = (z-z_0)^N g(z)$, where $g \neq 0$ near $z_0$. Then N is called the \textbf{order of vanishing} of $f$ at $z_0$.



\end{definition}

\isubsection{COR: Identity Principle}

\begin{corollary}[Identity Principle]\label{cor:ident-princ}
Suppose you have $f,g: \om \to \C$ holomorphic such that $\exists \{ z_j \}_{j=0}^\infty, \, z_j \xrightarrow[]{j \to \infty} z_\infty \in \om$ (with $z_i \neq z_j$ for $i \neq j$) such that $f(z_j) = g(z_j) \, \forall j$. Then $f \equiv g$ on $\om$.
\end{corollary}

\begin{proof}
If $f-g \not\equiv 0$ then the zeroes of $f-g$ must be isolated, but they are not by the assumption (the assumption says to assume that $f-g$ has a non-isolated zero).
\end{proof}

\begin{definition}[Analytic Continuation]
Suppose you have $\om_1 \subset \om_2 $ open connected. Suppose $f: \om_1 \to \C$ holomorphic and $F: \om_2 \to \C$ holomorphic such that $f = F |_{\om_1}$ (ie $f(z) = F(z) \, \forall z \in \om_1$) then $F$ is called an \textbf{analytic continuation} of $f$ to $\om_2$.

\end{definition}


\begin{note}[Uniqueness of Analytic Continuations]

Corollary (\ref{cor:ident-princ}) shows that if $F$ and $F'$ are both analytic continuations of $f$ to $\om_2$, then $F \equiv F'$ (since $F$ and $F'$ will agree on $\om_1$ and thus are equal).

We can thus refer to $F$ as the (unique) analytic continuation of $f$ to $\om_2$.
\end{note}

We now present a bit of review material in preparation for a discussion of normal families.

\section{Ascoli-Arzelà's Theorem}

\begin{proposition}
Suppose that $\exists \foc$ such that $f_n \to f$ uniformly on compact subsets of $\om$. Then $f$ is holomorphic on $\om$.
\end{proposition}


\begin{proof}
Uniform convergence on compact subsets means that for any $K \subset \om$ compact then $\sup_{z\in K} \abs{f_n(z) - f(z)} \xrightarrow[]{n \to \infty} 0$. 

Using theorem (\ref{thm:goursat}) and corollary (\ref{cor:morera}), let $z\in \om$. It suffices to show that $f$ is holomorphic on $D=D_r(z) \ssubset \om $ for some $r>0$. We check that $f$ is holomorphic on $D$. Let $T \subset D$ a triangle. Then by theorem (\ref{thm:goursat}) $\int_{\partial T} f_n(z) \dif z =  0$. On the other hand, for $D \subset K \subset \om$ compact, then if $f_n \tou f$ it is easy to see that $\int_{\partial T} f_n(z) \dif z \tou \int_{\partial T} f(z) \dif z$. Thus $\int_{\partial T} f(z) \dif z = 0$.
\end{proof}

\begin{note}
This proposition says that if $f$ is a uniform limit of holomorphic functions, then $f$ is also holomorphic. This is quite bizarre; a uniform limit of smooth functions on a compact set need not be smooth. Stone-Weierstrass says that every continuous function can be uniformly approximated by polynomials (both smooth and analytic), and thus that every continuous function is a uniform limit. This is quite different compared to the complex case.
\end{note}

\isubsection{THM: Ascoli-Arzelà}

\begin{theorem}[Ascoli-Arzelà]\label{thm:ascoli-arzela}
Let $\om$ open. Let $\{f_n \}_{n=0}^\infty, \, f_n : \om \to \C$ holomorphic on $\om$. Suppose that
\begin{itemize}
    \item $f_n$ is uniformly bounded (ie $\sup_{z\in \om} \abs{f_n(z)} < C$ for some $C$),
    \item $\set{ f_n }$ are uniformly equicontinuous on $\om$ (ie $\forall \epsilon \exists \delta$ s.t. if $\abs{x-y} < \delta, \, x,y \in \om$, then $\abs{f_n(x) - f_n(y)} < \epsilon$ for all n).
\end{itemize}

Then $\exists \foc$ continuous (and holomorphic by the previous proposition) such that $f_{n_j} \tou f$ for some subsequence $n_j$ on compact subsets of $\om$.

\end{theorem}
 
\begin{note}\label{note:after-ascoli-arzela}
Usually Ascoli-Arzelà is assumed on $K$ compact, which results in the uniform convergence on $K$. This is an extension of the usual form. If you have the assumptions on all of $\om$, then in particular you have them for all compact subsets. Thus for every compact set you obtain a uniform limit, and by exhaustion you can see that the limit will always have to be the same function.

Ascoli-Arzelà tells you that on a compact set you have a uniform limit, which is therefore continuous, and the previous proposition tells you that this limit is holomorphic.
\end{note}