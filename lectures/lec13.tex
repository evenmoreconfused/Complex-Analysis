%FILL IN THE RIGHT INFO.
%\lecture{**LECTURE-NUMBER**}{**DATE**}
\unchapter{Lecture 13}
\lecture{13}{October 15}
\setcounter{section}{0}
\setcounter{theorem}{0}

% **** YOUR NOTES GO HERE:

Recall from last lecture corollary (\ref{cor:max-mod-prin2}). In the assumption, we require that $\om$ is bounded. We will provide an example where the Maximum Modulus Principle v2 does not hold when $\om$ is unbounded.

\section{Applications of MMP}

\begin{counterexample}

Take $\om = \text{quarter plane} = \set{ z \in \C \mid \Re (z),\Im(z)  > 0 }$. Then $\partial \om = \set{z \in \C \mid \Re(z) = 0 \lor \Im(z) = 0}$.

\begin{center}
\begin{tikzpicture}[very thick,decoration={
    markings,
    mark=at position 0.5 with {\arrow{>}}}
    ]
    
    \draw [thin] (-1,0)--(3,0);
    \draw [thin] (0,-1)--(0,3);
    
    \fill [pattern=north west lines] (0,0) rectangle (3,3);
    
    \draw [very thick] (3,0) -- (0,0) -- (0,3);
    \draw [right](3,2) node {$\om$};
    \draw [below](1,0) node {$\partial \om$};
    
\end{tikzpicture}
\end{center}

Let $F(z) = e^{-iz^2}$ (holomorphic on $\C$ thus on $\om$ as well). It is easy to check that $z \in \partial \om \implies \abs{F(z)} = 1$. But letting $z = re^{\frac{i \pi}{4}} \in \om \implies \abs{F(z)} = e^{r^2}$ which is unbounded (in particular this is larger than $1$). Thus the Maximum Modulus Principle v2 fails.

\end{counterexample}

\begin{example}\label{ex:automorphism-mmp}
Say $D= D_1(0)$, $f: \overline{D} \to \C$ holomorphic non-constant s.t. $\abs{f(z)} \leq 1 \; \forall z \in \partial D$. We will show that $f(D) \subset D$.

By corollary (\ref{cor:max-mod-prin2}), $\sup_{D} \abs{f} = \sup_{\partial D} \abs{f} \leq 1$. Then $f(D) \subset \overline{D}$. To prove that $f(D) \subset D$, let us use corollary (\ref{cor:max-mod-prin1}): if this fails, then $\exists z_0 \in D$ s.t. $\abs{f(z_0)} = 1$, thus $\abs{f}$ achieves a max at $z_0$ and must be constant, which is a contradiction.
\end{example}

\section{Automorphisms}

We turn to the topic of automorphisms -- holomorphisms that map to a function's own domain.

\begin{theorem}\label{thm:auto-prelim}
Let $\oic$ open, $\foc$ holomorphic and injective. Then:
\begin{enumerate}
    \item $f(\om)$ is open.
    \item $f:\om \to f(\om)$ is bijective.
    \item $f'(z) \neq 0 \; \; \forall z \in \om$.
    \item $f^{-1}: f(\om) \to \om$ is holomorphic.
\end{enumerate}

\end{theorem}

\begin{note}
This does not hold in the real case. Compare point 3 with the function $f(x) = x^3$. This is a smooth injective function from $\R$ to $\R$, but its derivative vanishes at the origin.
\end{note}


\begin{proof}[\ref{thm:auto-prelim}]
\begin{enumerate}
    \item $f$ injective $\implies$ $f$ non-constant $\implies$ $f$ open map. Thus $f(\om)$ is open.
    \item $f$ is injective, so obviously bijective.
    \item $f' \nequiv 0$ on $\om$ (if it was, $f$ would be locally constant and thus not injective). Thus since $f'$ is holomorphic and $f' \nequiv 0$, the zeroes of $f$ are isolated. Let us show that $f'$ is never $0$.
    
    Suppose that $\exists \, z_0$ s.t. $f'(z_0) = 0$. Let $w_0 = f(z_0)$ and Let $F(z) = f(z) - w_0$. Then $F$ has an isolated zero at $z_0$ (if not isolated, $f \equiv w_0 \; \lightning$). We have that $F'(z_0) = f'(z_0) = 0$, so $z_0$ is a zero of $F$ of order $N \geq 2$. For $z$ close to $z_0$, $F'(z) \neq 0$.
    
    We argue exactly the same as in the proof of the theorem (\ref{thm:open-mapping}) and get that (using theorem (\ref{thm:rouche2})) for $w$ close to $w_0$, $f(z) = w$ has exactly $N$ solutions close to $z_0$. But at all such $z$, $f'(z) \neq 0$, so these are all distinct (since if two coincide, $f$ at that point has a zero of order at least two, thus $f'$ at that point is 0). Since $N\geq 2$, we get that $f$ is not injective (since there are at least two distinct zeroes of $f(z) - w$, thus two points that are mapped to $w$).
    
    Hence $f'$ never vanishes on $\om$.
    \item To show that $f^{-1}$ is holomorphic, for any $w_0 \in f(\om)$, write $w_0 = f(z_0)$ (uniquely) and for $w$ near $w_0$, write $w = f(z)$ (uniquely). Then, noting that since $z,z_0$ are unique and since $f^{-1}$ is continuous, then as $w \to w_0$, $z \to z_0$:
    \begin{align*}
        \lim_{w \to w_0}\frac{f^{-1} (w) - f^{-1} (w_0)  }{w-w_0} &= \lim_{z \to z_0} \frac{z-z_0}{f(z) - f(z_0)}\\
        &= \lim_{z \to z_0} \frac{1}{\frac{f(z) - f(z_0)}{z-z_0}}\\
        &= \frac{1}{f'(z_0)}.
    \end{align*}
    Since this limit exists, $f^{-1}$ is holomorphic.
\end{enumerate}
\end{proof}

\begin{definition}[Automorphism]
An \textbf{automorphism of $\om$} is a function $f: \om \to \om \subset \C$ open, $f$ holomorphic bijection. Note that by theorem (\ref{thm:auto-prelim}), $f^{-1}$ is holomorphic and bijective from $\om $ to $\om$.
\end{definition}

\begin{definition}[Automorphism Group]
We define:
\begin{align*}
Aut(\om) \vcentcolon = \set{f:\om \to \om \mid f \text{ automorphism of } \om}.
\end{align*}
This is called the \textbf{automorphism group of $\om$}.
\end{definition}

\begin{note}
$Aut(\om)$ has a natural group structure. Indeed:
\begin{itemize}
    \item Identity is: $Id : \om \to \om$.
    \item Group law is: $f \cdot g = f \circ g$.
    \item Inverse is: $(f)^{-1} = f^{-1}$.
\end{itemize}

\end{note}

\begin{example}[$Aut(\C)$]

We showed in homework 5 (Stein-Shakarchi, Exercise 3.14) that $f: \C \to \C $ holomorphic injective $\implies$ $f = az+b, \, a,b \in \C, \, a \neq 0$. The converse is true trivially. Functions of the form $az+b$ are also obviously bijective. We conclude that:
\begin{align*}
    Aut(\C) = \set{az+b \mid a,b \in \C, \, a \neq 0}.
\end{align*}


\end{example}

\begin{notation}
$\C^* \vcentcolon= \C \setminus \{ 0 \} $.
\end{notation}


\begin{example}[$Aut(\C^*)$]\label{ex:aut-C-star}

$Aut(\C^*) \vcentcolon = \set{f:\C^* \to \C^* \mid f \text{ holomorphic and bijective} }$.

To calculate $Aut(\C^*)$, let us look near $0$. $0$ is an isolated singularity of $f$, so there are three cases:
\begin{enumerate}
    \item $0$ is a removable singularity
    
    Thus $\exists$ $\Tilde{f}: \C \to \C$ holomorphic s.t. $\Tilde{f} \mid_{\C^*} = f$. We shall show that necessarily $\Tilde{f} (0) =0$.
    
    Suppose that $\Tilde{f} (0) = z \neq 0$. Then since $f$ is bijective $\exists ! \, w \neq 0$ s.t. $f(w) = z = f(0)$. Let $\varepsilon >0$ s.t. $D_\varepsilon(0) \cap D_\varepsilon(w) = \emptyset$. By theorem (\ref{thm:open-mapping}) $\Tilde{f} \left( D_\varepsilon(0) \right) $ and $\Tilde{f} \left( D_\varepsilon(w) \right) $ are open, and $z \in \Tilde{f} \left( D_\varepsilon(0) \right) \cap \Tilde{f} \left( D_\varepsilon(w) \right) $. $\Tilde{f} \left( D_\varepsilon(0) \right) \cap \Tilde{f} \left( D_\varepsilon(w) \right) $ is open and non-empty, thus $\exists \Tilde{z} \neq z, \, \Tilde{z} \in \Tilde{f} \left( D_\varepsilon(0) \right) \cap \Tilde{f} \left( D_\varepsilon(w) \right) $. Thus $\exists a \neq 0, \, a \in D_\varepsilon (0)$ and $b \neq 0, \, b \in D_\varepsilon (w)$ with $a \neq b$ s.t. $f(a) = f(b) = \Tilde{z}$. This is in contradiction to the condition that $f$ is injective on $\C ^*$. It follows that $\Tilde{f}(0) = 0$.
    
    Thus $\Tilde{f}: \C \to \C$ is bijective.
    
    Thus $\Tilde{f} \in Aut(\C)$ with $\Tilde{f} (0) = 0$. It follows that $f(z) = az, \, z \neq 0$. Clearly $f(z) = az, \, a \neq 0$ is an automorphism of $\C^*$.
    
    \item $0$ is a pole
    
    Let $g(z) = \frac{1}{f(z)}$. $g(z):\C^* \to \C^* $ is also bijective and holomorphic as a composition of two holomorphic bijections. Thus $g(z) \in Aut(\C^*)$, but since $0$ is a pole of $f$, $0$ is a removable singularity of $g$.
    
    By case 1, $g(z) = az, \, a \neq 0$. It follows that $f(z) = \frac{1}{az}, \, a \neq 0$. Clearly \\ $f(z) = \frac{1}{az}, \, a \neq 0$ is an automorphism of $\C ^*$.
    
    \item $0$ is an essential singularity
    
    This cannot be. This can be seen by applying theorem (\ref{thm:caso-weier}) and theorem (\ref{thm:open-mapping}) to an open neighborhood around $0$.
    
\end{enumerate}

It follows that $Aut(\C^*) = \set{ f(z) = az \text{ or } \frac{1}{az}, \, a \neq 0}$.


\end{example}


Much later we will discuss $Aut(D)$, $Aut(D^*)$, and $Aut(\mathbb{H})$ where $\mathbb{H}$ is the upper half plane.


\section{Riemann Sphere}

We turn our attention now to a topic quite different from what we have seen before. We want to be able to describe the relationship between a sphere and the complex plane. Intuitively, we would like to take the $1$-point compactification $\C \sqcup \{ \infty \} = S^2$ sphere.





\begin{center}
    
\begin{tikzpicture}[scale=0.5] % CENT

\def\R{2.5} % sphere radius
\def\angEl{15} % elevation angle
\def\angAz{-10} % azimuth angle
\def\angPhi{-40} % longitude of point P
\def\angBeta{180-25} % latitude of point P

%% working planes

\pgfmathsetmacro\H{\R*cos(\angEl)} % distance to north pole
\tikzset{xyplane/.style={cm={cos(\angAz),sin(\angAz)*sin(\angEl),-sin(\angAz),
                              cos(\angAz)*sin(\angEl),(0,0)}}}
%\tikzset{xyplane/.style={cm={cos(\angAz),sin(\angAz)*sin(\angEl),-sin(\angAz),
%                              cos(\angAz)*sin(\angEl),(0,-\H)}}}
\LongitudePlane[xzplane]{\angEl}{\angAz}
\LongitudePlane[pzplane]{\angEl}{\angPhi}
\LatitudePlane[equator]{\angEl}{0}

\draw [xyplane] (-2.5*\R,-2*\R) rectangle (2*\R,2*\R);
\coordinate[mark coordinate] (N) at (0,\H);
\coordinate[mark coordinate] (S) at (0,-\H);

\draw (N) +(-0.3ex,0.6ex) node[above right] {$\infty$};
\draw (S) +(-0.4ex,-0.4ex) node[below] {\phantom{}};
\draw [white][fill=white] (S) circle (0.1);

\draw (0,0) node {$\C$};


\end{tikzpicture}
\begin{tikzpicture}
\draw (0,1.5) node {  $\xleftrightarrow[]{\phantom{sentence}}$};
\draw (0,0) node {} ;
\end{tikzpicture}
\begin{tikzpicture}[scale=0.5] % CENT





\def\R{2.5} % sphere radius
\def\angEl{15} % elevation angle
\def\angAz{-10} % azimuth angle
\def\angPhi{-40} % longitude of point P
\def\angBeta{180-25} % latitude of point P

%% working planes

\pgfmathsetmacro\H{\R*cos(\angEl)} % distance to north pole
\tikzset{xyplane/.style={cm={cos(\angAz),sin(\angAz)*sin(\angEl),-sin(\angAz),
                              cos(\angAz)*sin(\angEl),(0,0)}}}
%\tikzset{xyplane/.style={cm={cos(\angAz),sin(\angAz)*sin(\angEl),-sin(\angAz),
%                              cos(\angAz)*sin(\angEl),(0,-\H)}}}
\LongitudePlane[xzplane]{\angEl}{\angAz}
\LongitudePlane[pzplane]{\angEl}{\angPhi}
\LatitudePlane[equator]{\angEl}{0}

%% draw xyplane and sphere

%\draw[xyplane] (-2*\R,-2*\R) rectangle (2.2*\R,2.8*\R);
\draw (0,0) circle (\R);

%% characteristic points

\coordinate (O) at (0,0);

\draw [xyplane] (-2.5*\R,-2*\R) rectangle (2*\R,2*\R);
\coordinate[mark coordinate] (N) at (0,\H);
                                        
%\coordinate[mark coordinate] (Phat2) at (intersection cs: first line={(S)--(P)},
%                                        second line={(0,0)--(Paux)});

%% draw meridians and latitude circles

\DrawLatitudeCircle[\R]{0} % equator





%% draw lines and put labels

\draw (N) +(-0.3ex,0.6ex) node[above right] {$\infty = \text{North Pole}$};

%\draw (Phat2) node[above right] {$\mathbf{R}$};




\end{tikzpicture}
\end{center}









\subsection{Stereographic Projection}


Consider a sphere intersected with a plane. Then given any point $P \in S^2\setminus \{ N \}$ there is a unique line containing $P$ and $N$. This line intersects the $xy$-plane in a point $Q$.













\begin{center}
    
\begin{tikzpicture} % CENT

%% some definitions

\def\R{2.5} % sphere radius
\def\angEl{15} % elevation angle
\def\angAz{-10} % azimuth angle
\def\angPhi{-40} % longitude of point P
\def\angBeta{180-25} % latitude of point P

%% working planes

\pgfmathsetmacro\H{\R*cos(\angEl)} % distance to north pole
\tikzset{xyplane/.style={cm={cos(\angAz),sin(\angAz)*sin(\angEl),-sin(\angAz),
                              cos(\angAz)*sin(\angEl),(0,0)}}}
%\tikzset{xyplane/.style={cm={cos(\angAz),sin(\angAz)*sin(\angEl),-sin(\angAz),
%                              cos(\angAz)*sin(\angEl),(0,-\H)}}}
\LongitudePlane[xzplane]{\angEl}{\angAz}
\LongitudePlane[pzplane]{\angEl}{\angPhi}
\LatitudePlane[equator]{\angEl}{0}

%% draw xyplane and sphere

%\draw[xyplane] (-2*\R,-2*\R) rectangle (2.2*\R,2.8*\R);
\draw (0,0) circle (\R);

%% characteristic points

\coordinate (O) at (0,0);

\draw [xyplane] (-2.5*\R,-2*\R) rectangle (2*\R,2*\R);
\coordinate[mark coordinate] (N) at (0,\H);
\coordinate[mark coordinate] (S) at (0,-\H);
\path[pzplane] (\angBeta:\R) coordinate[mark coordinate] (P);
\path[pzplane] (\R,0) coordinate (PE);
\path[xzplane] (\R,0) coordinate (XE);
\path (PE) ++(0,0) coordinate (Paux); % to aid Phat calculation
%\path (PE) ++(0,-\H) coordinate (Paux); % to aid Phat calculation
\coordinate[mark coordinate] (Phat) at (intersection cs: first line={(N)--(P)},
                                        second line={(0,0)--(Paux)});
                                        
%\coordinate[mark coordinate] (Phat2) at (intersection cs: first line={(S)--(P)},
%                                        second line={(0,0)--(Paux)});

%% draw meridians and latitude circles

\DrawLatitudeCircle[\R]{0} % equator





%% draw lines and put labels

\draw (P) -- (N) +(-0.3ex,0.6ex) node[above right] {$\mathbf{N}$};

\draw (S) +(-0.4ex,-0.4ex) node[below] {$\mathbf{S}$};

\draw (P) -- (Phat) node[above left] {$\mathbf{Q}$};
\draw (P) node[above left] {$\mathbf{P}$};
%\draw (Phat2) node[above right] {$\mathbf{R}$};




\end{tikzpicture}
\end{center}







\begin{definition}[Stereographic Projection from $N$]
The map that takes $P$ to $Q$ is called the \textbf{stereographic projection from $N$}, denoted by $Str_N$.
\end{definition}


\begin{remark}
This is a bijection between $S^2 \setminus \{ N \}$ and $\R^2 = \C$.
\end{remark}

Now taking the stereographic projection from $S$ gives you a different bijection $Str_S$ between $S^2 \setminus \{ S \}$ and $\R^2 = \C$ that takes $P$ to some point $R \in \R^2$.


\begin{center}
    
\begin{tikzpicture} % CENT

%% some definitions

\def\R{2.5} % sphere radius
\def\angEl{15} % elevation angle
\def\angAz{-10} % azimuth angle
\def\angPhi{-40} % longitude of point P
\def\angBeta{180-25} % latitude of point P

%% working planes

\pgfmathsetmacro\H{\R*cos(\angEl)} % distance to north pole
\tikzset{xyplane/.style={cm={cos(\angAz),sin(\angAz)*sin(\angEl),-sin(\angAz),
                              cos(\angAz)*sin(\angEl),(0,0)}}}
%\tikzset{xyplane/.style={cm={cos(\angAz),sin(\angAz)*sin(\angEl),-sin(\angAz),
%                              cos(\angAz)*sin(\angEl),(0,-\H)}}}
\LongitudePlane[xzplane]{\angEl}{\angAz}
\LongitudePlane[pzplane]{\angEl}{\angPhi}
\LatitudePlane[equator]{\angEl}{0}

%% draw xyplane and sphere

%\draw[xyplane] (-2*\R,-2*\R) rectangle (2.2*\R,2.8*\R);
\draw (0,0) circle (\R);

%% characteristic points

\coordinate (O) at (0,0);

\draw [xyplane] (-2.5*\R,-2*\R) rectangle (2*\R,2*\R);
\coordinate[mark coordinate] (N) at (0,\H);
\coordinate[mark coordinate] (S) at (0,-\H);
\path[pzplane] (\angBeta:\R) coordinate[mark coordinate] (P);
\path[pzplane] (\R,0) coordinate (PE);
\path[xzplane] (\R,0) coordinate (XE);
\path (PE) ++(0,0) coordinate (Paux); % to aid Phat calculation
%\path (PE) ++(0,-\H) coordinate (Paux); % to aid Phat calculation
\coordinate[mark coordinate] (Phat) at (intersection cs: first line={(N)--(P)},
                                        second line={(0,0)--(Paux)});
                                        
\coordinate[mark coordinate] (Phat2) at (intersection cs: first line={(S)--(P)},
                                        second line={(0,0)--(Paux)});

%% draw meridians and latitude circles

\DrawLatitudeCircle[\R]{0} % equator





%% draw lines and put labels

\draw (P) -- (N) +(-0.3ex,0.6ex) node[above right] {$\mathbf{N}$};

\draw (P) -- (S) +(-0.4ex,-0.4ex) node[below] {$\mathbf{S}$};

\draw (P) -- (Phat) node[above left] {$\mathbf{Q}$};
\draw (P) node[above left] {$\mathbf{P}$};
\draw (Phat2) node[above right] {$\mathbf{R}$};




\end{tikzpicture}
\end{center}


We want to find a map $\phi : \C \to \C$, $Q \mapsto R$. Consider $Q = (x,y,0)$ and $N= (0,0,1)$. To find $P $ we parameterize $\overline{QN}$ by:
\begin{align*}
    t \cdot (0,0,1) + (1-t) \cdot (x,y,0) = \big( (1-t) \cdot x, \, (1-t) \cdot y, \, t \big) = \gamma(t).
\end{align*}



\begin{center}
    
\begin{tikzpicture} % CENT

%% some definitions

\def\R{2.5} % sphere radius
\def\angEl{15} % elevation angle
\def\angAz{-10} % azimuth angle
\def\angPhi{-40} % longitude of point P
\def\angBeta{180-25} % latitude of point P

%% working planes

\pgfmathsetmacro\H{\R*cos(\angEl)} % distance to north pole
\tikzset{xyplane/.style={cm={cos(\angAz),sin(\angAz)*sin(\angEl),-sin(\angAz),
                              cos(\angAz)*sin(\angEl),(0,0)}}}
%\tikzset{xyplane/.style={cm={cos(\angAz),sin(\angAz)*sin(\angEl),-sin(\angAz),
%                              cos(\angAz)*sin(\angEl),(0,-\H)}}}
\LongitudePlane[xzplane]{\angEl}{\angAz}
\LongitudePlane[pzplane]{\angEl}{\angPhi}
\LatitudePlane[equator]{\angEl}{0}

%% draw xyplane and sphere

%\draw[xyplane] (-2*\R,-2*\R) rectangle (2.2*\R,2.8*\R);
\draw (0,0) circle (\R);

%% characteristic points

\coordinate (O) at (0,0);

\draw [xyplane] (-2.5*\R,-2*\R) rectangle (2*\R,2*\R);
\coordinate[mark coordinate] (N) at (0,\H);
\coordinate[mark coordinate] (S) at (0,-\H);
\path[pzplane] (\angBeta:\R) coordinate[mark coordinate] (P);
\path[pzplane] (\R,0) coordinate (PE);
\path[xzplane] (\R,0) coordinate (XE);
\path (PE) ++(0,0) coordinate (Paux); % to aid Phat calculation
%\path (PE) ++(0,-\H) coordinate (Paux); % to aid Phat calculation
\coordinate[mark coordinate] (Phat) at (intersection cs: first line={(N)--(P)},
                                        second line={(0,0)--(Paux)});
                                        
\coordinate[mark coordinate] (Phat2) at (intersection cs: first line={(S)--(P)},
                                        second line={(0,0)--(Paux)});

%% draw meridians and latitude circles

\DrawLatitudeCircle[\R]{0} % equator





%% draw lines and put labels

\draw (P) -- (N) +(-0.3ex,0.6ex) node[above right] {$\mathbf{N=(0,0,1)}$};

\draw (P) -- (S) +(-0.4ex,-0.4ex) node[below] {$\mathbf{S=(0,0,-1)}$};

\draw (P) -- (Phat) node[above left] {$\mathbf{(x,y,0)=Q}$};
\draw (P) node[above left] {$\mathbf{P}$};
\draw (Phat2) node[above right] {$\mathbf{R}$};




\end{tikzpicture}
\end{center}




Now $P$ is (one) intersection of $\gamma$ and $S^2$ such that:
\begin{align*}
    1 = \abs{\gamma(t) }^2 &= (1-t)^2x^2 + (1-t)^2 y^2 + t^2\\
    \text{(letting $A \vcentcolon = x^2 + y^2$) } &= t^2 (A+1) - 2At + A .
\end{align*}

This is quadratic in $t$. Thus:
\begin{align*}
    t &= \frac{A \pm \sqrt{A^2 - (A^2 - 1)}}{A+1}\\ \text{(ignore $t=1$; this is N) } &= \frac{A-1}{A+1}.
\end{align*}

Thus, with $(1-t) = \frac{2}{A+1}$:
\begin{align*}
    P &= \bigg( \underbrace{\frac{2x}{x^2+y^2+1} }_{\mathfrak{X}}, \,  \underbrace{\frac{2y}{x^2+y^2+1}}_{\mathfrak{Y}} , \, \underbrace{\frac{x^2+y^2-1}{x^2+y^2+1}}_{\mathfrak{Z}}  \bigg)\\
    &= \left( \mathfrak{X}, \, \mathfrak{Y}, \, \mathfrak{Z} \right).
\end{align*}

Now take $\Tilde{\gamma} = \overline{SP} = \big( (1-t) \cdot \mathfrak{X}, \, (1-t) \cdot \mathfrak{Y}, \, (1-t)\cdot \mathfrak{Z} - t \big)$. $R$ is the intersection of $\Tilde{\gamma}$ and the $xy$-plane. Thus $t = \frac{\mathfrak{Z}}{\mathfrak{Z}+1}$ and $1-t = \frac{1}{\mathfrak{Z} + 1}$. It follows (after some arithmetic) that:
\begin{align*}
    R &= \left( \frac{\mathfrak{X}}{\mathfrak{Z} + 1} , \, \frac{\mathfrak{Y}}{\mathfrak{Z}+1} , \, 0 \right) = \left( \frac{x}{x^2+y^2} , \, \frac{y}{x^2+y^2} , \, 0         \right).
\end{align*}

So $\phi = Str_S \circ (Str_N)^{-1}$ maps $Q = (x,y,0)$ to $R = \left( \frac{x}{x^2+y^2} , \, \frac{y}{x^2+y^2} , \, 0         \right)$. Now we can think of $Q,R$ as elements of $\R^2 = \C$:
\begin{align*}
    Q &= z = x+iy\\
    R &= \frac{x}{x^2 + y^2} + i \frac{y}{x^2 + y^2}\\
    &= \frac{z}{x^2 + y^2}\\
    &= \frac{z}{\abs{z}^2}\\ &= \frac{1}{\overline{z}}.
\end{align*}

%Thus our map is $\phi: \C \to \C$, $z \mapsto \frac{1}{\overline{z}}$ .


If we compose one of the two projections with $\overline{z}$, we get that to obtain $S^2$ you can take two copies of $\C$ and glue them on $\C^* \subset \C$ via the map $z \mapsto \frac{1}{z}$.

\begin{center}
    
\begin{tikzpicture}[scale=0.5] % CENT

\def\h{12} % height of new plane
\def\R{2.5} % sphere radius
\def\angEl{15} % elevation angle
\def\angAz{-10} % azimuth angle
\def\angPhi{-40} % longitude of point P
\def\angBeta{180-25} % latitude of point P

%% working planes

\pgfmathsetmacro\H{\R*cos(\angEl)} % distance to north pole
\tikzset{xyplane/.style={cm={cos(\angAz),sin(\angAz)*sin(\angEl),-sin(\angAz),
                              cos(\angAz)*sin(\angEl),(0,0)}}}
%\tikzset{xyplane/.style={cm={cos(\angAz),sin(\angAz)*sin(\angEl),-sin(\angAz),
%                              cos(\angAz)*sin(\angEl),(0,-\H)}}}
\LongitudePlane[xzplane]{\angEl}{\angAz}
\LongitudePlane[pzplane]{\angEl}{\angPhi}
\LatitudePlane[equator]{\angEl}{0}

\draw [xyplane] (-2.5*\R,-2*\R) rectangle (2*\R,2*\R);
\draw [xyplane] ($(-2.5*\R,-2*\R)+(0,\h)$) rectangle ($(2*\R,2*\R)+(0,\h)$);

\draw [xyplane,fill] (0,0) circle (0.1);
\draw [xyplane,fill] (2.5,0) circle (0.1);
\draw [xyplane,fill] (-2.5,0) circle (0.1);

\draw [xyplane,fill] ($(0,0) + (0,\h)$) circle (0.1);
\draw [xyplane,fill] ($(2.5,0) + (0,\h)$) circle (0.1);
\draw [xyplane,fill] ($(-2.5,0) + (0,\h)$) circle (0.1);

\draw [xyplane] [<->] (-2.5,0) -- (-2.5,\h);
\draw [xyplane] [<->] (2.5,0) -- (2.5,\h);

% \coordinate[mark coordinate] (N) at (0,\H);
% \coordinate[mark coordinate] (S) at (0,-\H);

% \draw (N) +(-0.3ex,0.6ex) node[above right] {$\infty$};
% \draw (S) +(-0.4ex,-0.4ex) node[below] {\phantom{}};
% \draw [white][fill=white] (S) circle (0.1);

\draw [xyplane, below left] (-2.5,0) node {$\frac{1}{z}$};
\draw [xyplane, below] (0,0) node {$0$};
\draw [xyplane, below right] (2.5,0) node {$\frac{1}{z}$};

\draw [xyplane, above left] (-2.5,\h) node {$z$};
\draw [xyplane, above] (0,\h) node {$0 = ``\infty"$};
\draw [xyplane, above right] (2.5,\h) node {$z$};

\draw [xyplane] (6,0) node {$\C$};
\draw [xyplane] (6,\h) node {$\C$};


\end{tikzpicture}
\end{center}

\begin{remark}
Those familiar with manifolds will recognize this as one of the more popular choices of atlases with two charts for the $2$-dimensional manifold $S^2$, where $z \xleftrightarrow[]{} \frac{1}{z}$ is the transition between the two charts. Following what one might think in this framework, we use this chart and transition map structure to properly define functions on $S^2$.
\end{remark}


We can use this to define holomorphic functions to and from $S^2$.


\begin{definition}[Holomorphisms from $S^2$ to $\C$]

Consider $f: S^2 \to \C$ a map of sets. $f$ is called \textbf{holomorphic} if on $U_1$, $f \circ Str_N: \C \to \C$ is holomorphic and on $U_2$, $f \circ Str_S: \C \to \C$ is holomorphic.
\end{definition}


























