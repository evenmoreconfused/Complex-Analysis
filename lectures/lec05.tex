%FILL IN THE RIGHT INFO.
%\lecture{**LECTURE-NUMBER**}{**DATE**}
\unchapter{Lecture 5}
\lecture{5}{September 17}
\setcounter{section}{0}
\setcounter{theorem}{0}

% **** YOUR NOTES GO HERE:
\section{Cauchy's Theorem}
We begin with a short discussion on what positive orientation means:
\subsection{Orientation}
Consider a blob $\Omega$ with 2 holes in it:

\begin{center}
\begin{tikzpicture}[
    tangent/.style={
        decoration={
            markings,% switch on markings
            mark=
                at position #1
                with
                {
                    \coordinate (tangent point-\pgfkeysvalueof{/pgf/decoration/mark info/sequence number}) at (0pt,0pt);
                    \coordinate (tangent unit vector-\pgfkeysvalueof{/pgf/decoration/mark info/sequence number}) at (1,0pt);
                    \coordinate (tangent orthogonal unit vector-\pgfkeysvalueof{/pgf/decoration/mark info/sequence number}) at (0pt,1);
                }
        },
        postaction=decorate
    },
    use tangent/.style={
        shift=(tangent point-#1),
        x=(tangent unit vector-#1),
        y=(tangent orthogonal unit vector-#1)
    },
    use tangent/.default=1
]

    \draw[pattern=north west lines][dotted][scale =0.5][tangent=0.25, tangent = 0.55, tangent = 0.625, tangent = 0.95]
        plot [smooth cycle] coordinates {(-10,0) (-7,3.5) (0,5)  (4,2.5) (7,0)  (5,-7) (2,-6)  (-5,-5) } (3,-1.5) circle (2) (-3,0.5) circle (3);
        
\draw [blue, thick, use tangent][->] (0.25,0) -- (-1,0);
\draw [blue, thick, use tangent][->] (0,0) -- (0,-0.5);

\draw [blue, thick, use tangent=2][->] (0.25,0) -- (-1,0);
\draw [blue, thick, use tangent=2][->] (0,0) -- (0,-0.5);

\draw [blue, thick, use tangent=3][->] (0.25,0) -- (-1,0);
\draw [blue, thick, use tangent=3][->] (0,0) -- (0,-0.5);

\draw [blue, thick, use tangent=4][->] (0.25,0) -- (-1,0);
\draw [blue, thick, use tangent=4][->] (0,0) -- (0,-0.5);

\draw node at (-3.75,-2) {$\Omega$};

\end{tikzpicture}

    
\end{center}

% \begin{center}
% \begin{tikzpicture}[
%     tangent/.style={
%         decoration={
%             markings,% switch on markings
%             mark=
%                 at position #1
%                 with
%                 {
%                     \coordinate (tangent point-\pgfkeysvalueof{/pgf/decoration/mark info/sequence number}) at (0pt,0pt);
%                     \coordinate (tangent unit vector-\pgfkeysvalueof{/pgf/decoration/mark info/sequence number}) at (1,0pt);
%                     \coordinate (tangent orthogonal unit vector-\pgfkeysvalueof{/pgf/decoration/mark info/sequence number}) at (0pt,1);
%                 }
%         },
%         postaction=decorate
%     },
%     use tangent/.style={
%         shift=(tangent point-#1),
%         x=(tangent unit vector-#1),
%         y=(tangent orthogonal unit vector-#1)
%     },
%     use tangent/.default=1
% ]

%     \draw[red][][scale =1][tangent=0.1, tangent=0.2, tangent=0.4, tangent=0.55, tangent=0.65, tangent=0.75, tangent=0.9]
%     % \draw[red][][scale =1][tangent=0.9, tangent=0.1, tangent=0.2, tangent=0.4, tangent=0.5, tangent=0.65, tangent=0.8]

%         plot [smooth cycle] coordinates {(0,1) (1,1.15) (2,0)  (1,-1.15) (0,-1) (-1,-1.15) (-2,0) (-1,1.15)};
        
% \draw [red, thick, use tangent=7][-Straight Barb] (0.01,0) -- (-0.01,0);

% \draw [thick, use tangent=1] node[cross=4pt,red] {};
% \draw [thick, use tangent=2] node[cross=4pt,red] {};

% \draw [red, thick, use tangent=3][-Straight Barb] (0.01,0) -- (-0.01,0);

% \draw [thick, use tangent=4] node[cross=4pt,red] {};
% \draw [thick, use tangent=5] node[cross=4pt,red] {};
% \draw [thick, use tangent=6] node[cross=4pt,red] {};

% % \path[use as bounding box] (-2.5,-1.5) rectangle (2.5,1.5);



% % \pattern[pattern=north east lines] (1,-1.5)--(1,1.5)--(3,1.5)--(3,-1.5)--cycle;
% % \pattern[pattern=north east lines] (-1,-1.5)--(-1,1.5)--(-3,1.5)--(-3,-1.5)--cycle;
% \pattern[pattern={mylines[size=10pt,line width=.4pt,angle=35]}] (1,-1.5)--(1,1.5)--(3,1.5)--(3,-1.5)--cycle;
% \pattern[pattern={mylines[size=10pt,line width=.4pt,angle=35]}] (-1,-1.5)--(-1,1.5)--(-3,1.5)--(-3,-1.5)--cycle;
% \draw (1,-1.5)--(1,1.5);
% \draw (-1,-1.5)--(-1,1.5);



% \end{tikzpicture}

    
% \end{center}





















If you are at a point on the curve, the positive direction is the direction such that the domain is on your left.

\subsection{Cauchy's Theorem}

\isubsection{PROP: Complex Green's Theorem}

\begin{proposition}[Complex Green's Theorem]
$\Omega \subset \mathbf{C}$ bounded open, $f:\overline{\Omega} \rightarrow \mathbf{C}, \, C^1 \, f = u+iv, \, u,v:\overline{\Omega} \rightarrow \mathbf{R}$. Assume that $\partial \Omega$ can be parameterized by piecewise smooth curves.

Then (with the parameterization in the positive direction):

\begin{align*}
    \int_{\partial \Omega} f(z)  \dif z = 2i \int_\Omega \frac{\partial f}{ \partial \overline{z}}  \dif A 
\end{align*}

This can be interpreted as a complex version of Green's Theorem.
\end{proposition}

\begin{proof}
$\Omega = \bigcup_{j=1}^n C_j$, each smooth:

\begin{align*}
    \int_{\partial \Omega} f(z)  \dif z = \sum_{j=1}^n \int_{C_j} f(z) \dif z
\end{align*}

Parameterize each $C_j$ separately by $z(t), t\in[a,b]$. Then:

\begin{align*}
\int_{C_j} f(z)  \dif z = \int_{C_j} f(z(t))\cdot z'(t)  \dif z
\end{align*}

Write $z(t) = (x(t),y(t)) = x(t)+iy(t)$, $z'(t) = x'(t)+iy'(t)$.

\begin{align*}
\int_{C_j} f(z)  \dif z &= \int_{C_j} f(z(t))\cdot z'(t)  \dif z\\
&= \int_a^b (u(x(t),y(t)) + iv(x(t),y(t))) \cdot (x'(t)+iy'(t))  \dif t\\
&= \int_a^b u(x(t),y(t))\cdot x'(t) \dif t - \int_a^b v(x(t),y(t)) \cdot y'(t)  \dif t \\ +&i \left[ \int_a^b u(x(t),y(t))\cdot y'(t) \dif t +  \int_a^b v(x(t),y(t))\cdot x'(t) \dif t  \right]\\
&= \int_{C_J} u \dif x - \int_{C_J} v \dif y + i \left[ \int_{C_J} u \dif y + \int_{C_J} v \dif x \right] \\
\implies \int_{\partial \Omega} f(z)  \dif z &= \int_{\partial \Omega} (u \dif x-v \dif y) + i\int_{\partial \Omega} (u \dif y+v \dif x)
\end{align*}
Then both $\int_{\partial \Omega} (u \dif x-v \dif y)$ and $\int_{\partial \Omega} (u \dif y+v \dif x)$ are line integrals of real-valued functions. We can then apply Green's Theorem to them:

\begin{align*}
    \text{(let $\Vec{F} = (u,-v)$) }\int_{\partial \Omega} (u \dif x-v \dif y) &= \int_\Omega \left( -\frac{\partial v}{\partial x} - \frac{\partial u}{\partial y}  \right)  \dif A\\
    \text{(let $\Vec{F} = (v,u)$) }\int_{\partial \Omega} (u \dif y+v \dif x) &= \int_\Omega \left( \frac{\partial u}{\partial x} - \frac{\partial v}{\partial y}  \right)  \dif A\\
    \Downarrow\\
    \int_{\partial \Omega} f(z) \dif z &= i\int_\Omega \bigg(  \underbrace{\frac{\partial u}{\partial x} + i \frac{\partial v}{\partial x}}_{\frac{df}{ \dif z}} + i \bigg( \underbrace{  \frac{\partial u}{\partial y} + i \frac{\partial v}{\partial y}}_{\frac{\partial f}{\partial y}} \bigg)\bigg) \dif A\\
    &= i \int_\Omega \left( \frac{\partial f}{\partial x} + i\frac{\partial f}{\partial y} \right)  \dif A\\
   \text{(apply (1.6))} &= 2i \int_\Omega \frac{\partial f}{\partial \overline{z}}  \dif A\\
\end{align*}

\end{proof}

We now get to the key result of this chapter:

\isubsection{THM: Cauchy's Theorem}
\begin{theorem}[Cauchy's Theorem]
$\Omega \subset \mathbb{C}$ open, connected, $f:\Omega \rightarrow \mathbb{C}$. Then the following are equivalent:

\begin{enumerate}
    \item f is holomorphic (in $\Omega$)
    \item $f\in C^1(\Omega)$ and $\frac{\partial f}{\partial \overline{z}} = 0$ on $\Omega$ (CR equation)
    \item $f\in C^1(\Omega)$ and $\forall \Omega' \ssubset \Omega$ (\, $\overline{\Omega'} \subset \Omega$ and compact) then $\int_{\partial \Omega'} f(z) \dif z = 0$
    \item $f\in C^1(\Omega)$ and $\forall D = D_r(z_0) \ssubset \Omega$ and $\forall z\in D$ we have $f(z) = \frac{1}{2\pi i} \int_{\partial D} \frac{f(w)}{w-z}  \dif w$ (Cauchy's Integral Formula)
    \item $\forall D = D_r(z_0) \ssubset \Omega$ and $\forall z\in D$ we have $f(z) = \sum_{n=0}^\infty a_n (z-z_0)^n$, this power series absolutely convergent on D (ie f is analytic on $\Omega$) and $\forall n\geq 0$, $a_n = \frac{f^{(n)}(z_0) }{n!} = \frac{1}{2\pi i} \int_{\partial D} \frac{f(w)}{(w-z)^{n+1}}  \dif w$.
\end{enumerate}
\end{theorem}





\begin{note}
Note that ($1 \leftrightarrow 2$) gained us some depth compared to before. As discussed f holomorphic implies that f is $C^0$, but now we know it is $C^1$ (in fact (5) implies that f is $C^\infty$).

Note that (3) is a much stronger version of Goursat. Goursat is for triangles only, but this applies to many more cases (Goursat is the same statement with $\overline{\Omega'}=T$ triangle). We also proved this for disks.

Statement (4) is very important; it tells us that if you want to know the value of f(z), we can find the value by computing an integral on a disk containing z.

Statement (5) contains a generalization of Cauchy's Integral Formula in the formula for $a_n$. Statement (4) can be recovered by letting $n=0$.
\end{note}



\begin{proof}
The proof will take a $(1) \Rightarrow (2) \Leftrightarrow (3) \Rightarrow (4) \Rightarrow (5) \Rightarrow (1)$ approach.

\begin{enumerate}
    \item[$(2)\Rightarrow (3):$] This is an immediate result of Complex Green's Theorem (5.1).
    \item[$(3)\Rightarrow (2):$] Suppose not, ie $\exists z$ s.t. $\frac{\partial f}{\partial \overline{z}} \neq 0$. Consider $D= D_r(z_0)\ssubset \Omega$. Apply (3) to give:
    \begin{align*}
        0 = \abs{\int_{\partial D} f(z)  \dif z} \overset{Green}{=}& \abs{2i \int_D \frac{\partial f}{\partial \overline{z}}(z)  \dif A}\\
        =& 2\abs{\int_D \frac{\partial f}{\partial \overline{z}}(z) - \frac{\partial f}{\partial \overline{z}}(z_0)  \dif A + \frac{\partial f}{\partial \overline{z}}(z_0) \int_D \dif A }\\
        \geq&  2\abs{\frac{\partial f}{\partial \overline{z}}(z_0) \underbrace{\int_D \dif A}_{\pi r^2}}  -  2 \abs{\int_D \frac{\partial f}{\partial \overline{z}}(z) - \frac{\partial f}{\partial \overline{z}}(z_0)  \dif A}\\
        =& 2\pi r^2 \underbrace{\abs{\frac{\partial f}{\partial \overline{z}}(z_0)}}_{>0 \text{ assumed}} - 2 \abs{\int_D \frac{\partial f}{\partial \overline{z}}(z) - \frac{\partial f}{\partial \overline{z}}(z_0)  \dif A}\\
    \end{align*}
    $f\in C^1 \implies \exists f' \in C^0 \implies \exists \delta>0$ s.t. if $\abs{z-z_0}< \delta$ then $\abs{ \frac{\partial f}{\partial \overline{z}}(z) - \frac{\partial f}{\partial \overline{z}}(z_0) }< \frac{1}{2}\abs{\frac{\partial f}{\partial \overline{z}}(z_0)}$. Thus:
    
    \begin{align*}
        2 \abs{\int_D \frac{\partial f}{\partial \overline{z}}(z) - \frac{\partial f}{\partial \overline{z}}(z_0)  \dif A} &\leq 2 \int_D \abs{ \frac{\partial f}{\partial \overline{z}}(z) - \frac{\partial f}{\partial \overline{z}}(z_0)}  \dif A\\
        &< \pi r^2 \abs{\frac{\partial f}{\partial \overline{z}}(z_0)}\\
        &\Downarrow\\
        0 &\geq \left( 2\pi r^2 - \pi r^2\right) \abs{\frac{\partial f}{\partial \overline{z}}(z_0)} > 0 \, \, \, \bot
    \end{align*}
    
    \item[$(3) \Rightarrow (4):$] Assume (3). Then (2) holds. Hence $\frac{df}{d\overline{z}} = 0$ in $\Omega$. Fix $z\in\Omega$. Let: $g(w) = \frac{f(w)-f(z)}{w-z}$ for $w\in\Omega \setminus \{z\}$. Then $g\in C^1(\Omega \setminus \{z\})$ and $\frac{dg}{d\overline{w}}(w) = 0$. Thus $g$ satisfies (2) and thus (3) on $\Omega \setminus \{z\}$. Apply (3) on $\Omega' = \{ \epsilon < \abs{w-z} < \delta \, | \, 0<\epsilon<\delta\}$, with $\delta$ small enough that $\Omega' \ssubset \Omega$. Then $\Omega'$ is an annulus centered at $z$:
    
    \begin{center}
    \begin{tikzpicture}[scale = 2]
    
        \draw[dotted][scale = 0.5] plot [smooth cycle] coordinates {(-3,3) (0,4) (4,3) (3,-3) (-4,-3.25) (-6,0)};
    
    
    
        \draw [color=red][pattern=north west lines][dotted][thick]  (0,0) circle (1.4);
        
        \draw [color=red][fill=white][dotted][thick] (0,0) circle (1);
        \draw[fill] (0,0) circle [radius=0.025];
        \node [below] at (0,0) {$z$};
        \draw [thick] (0,0) -- (75-90:1.4);
        \draw [thick] (0,0) -- (75-90+120:1);
        \node [below] at (75-90:0.7) {$\delta$};
        \node [left] at (75-90+120:0.5) {$\epsilon$};
        
        \node [below] at (-2,-1.35) {$\Omega$};
        \node [below] at (270+30:1.4) {$\Omega'$};
        
        
        
    \end{tikzpicture}
    \end{center}
    
    
    Applying (3) we get by (and taking the integral going clockwise on both curves):
    
    \begin{align*}
        0 = \int_{\partial \Omega'} g(w) \dif w &= \int_{\partial C_\delta} g(w)  \dif w - \int_{\partial C_\epsilon} g(w)  \dif w\\
        &\Downarrow\\
        \int_{\partial C_\epsilon} g(w)  \dif w &= \int_{\partial C_\delta} g(w)  \dif w
    \end{align*}
    
    Then apply the Taylor expansion for $f\in C^1$.
    
    \begin{lemma}
    For $f \in C^1$, w sufficiently close to z, we have that:
    
    \begin{align*}
    f(w) = f(z) + \frac{\partial f}{ \partial z} (z) (w-z) + \frac{ \partial f}{ \partial \overline{z} } (z) (\overline{w}-\overline{z})+ o((z-w))
    \end{align*}
    
    This is the Taylor expansion of a $C^1$ function of a complex variable. Note that $o(f)$ is a function such that $\abs{\frac{o(f)}{\abs{f}}} \rightarrow 0$.
    \end{lemma}
    \begin{proof}
    Use the real Taylor Expansion as follows:
    
    
    
    \begin{align*}
        z&=x+iy\\
        w&=a+ib\\
        \text{(Taylor) } f(w) &= f(z) + \frac{\partial f }{\partial x}(z)(a-x)+ \frac{\partial f }{\partial y}(z)(b-y) +o((z-w))
    \end{align*}
    
    Then:
    
    \begin{align*}
        \frac{\partial f}{ \partial z} (z) (w-z) + \frac{ \partial f}{ \partial \overline{z} } (z) (\overline{w}-\overline{z}) &=  \frac{1}{2}\left( \frac{\partial f}{\partial x} - i\frac{\partial f}{\partial y}\right)(a+ib-x-iy)\\
        &+ \frac{1}{2}\left( \frac{\partial f}{\partial x} + i\frac{\partial f}{\partial y}\right)(a-ib-x+iy)\\
        &= \frac{\partial f }{\partial x}(z)(a-x)+ \frac{\partial f }{\partial y}(z)(b-y)
    \end{align*}
    \end{proof}
    
    Then we apply this lemma to f. f is holomorphic so we get (after some rearranging):
    
    \begin{align*}
        \frac{f(w)-f(z)}{w-z} = g(w) &= \frac{\partial f}{\partial z}(z) + \frac{o((z-w))}{w-z}\\
        \text{(let w be close to z) }&=\frac{\partial f}{\partial z}(z)\\
    \end{align*}
    Now g can be extended to a continuous function on all $\Omega$ (before it was defined for $w \neq z$) by letting $g(z) = \frac{\partial f}{\partial z}(z)$. Thus:
    
    \begin{align*}
        \abs{\int_{\partial C_\epsilon} g(w)  \dif w} &\leq L(C_\epsilon) \cdot \sup_{w\in C_\epsilon} \abs{g(w)}\\
        &= 2\pi \epsilon \underbrace{\sup_{w\in C_\epsilon} \abs{g(w)}}_{\text{bd. by (5.4)}}\\
        \text{(letting $\epsilon \rightarrow 0$) } & \Downarrow\\
        0 = \int_{\partial C_\epsilon} g(w)  \dif w &= \int_{\partial C_\delta} g(w)  \dif w
    \end{align*}
    
    Then:
    
    \begin{align*}
        \int_{\partial C_\delta} g(w)  \dif w &= \int_{\partial C_\delta} \frac{f(w)}{w-z}  \dif w - f(z) \underbrace{\int_{\partial C_\delta} \frac{ \dif w}{w-z}}_{compute}
    \end{align*}
    Let $w=z+\delta e^{it}, t\in [0,2\pi]$:
    
    \begin{align*}
        \int_{\partial C_\delta} \frac{ \dif w}{w-z} &= \int_{0}^{2\pi} \frac{i\delta e^{it}}{\delta e^{it}} \dif t\\
        &= i\int_{0}^{2\pi} \dif t =2 \pi i
    \end{align*}
    
    Thus:
    
    \begin{align*}
         f(z) = \frac{1}{2\pi i}\int_{\partial C_\delta} \frac{f(w)}{w-z}  \dif w 
    \end{align*}
    
    This proves (4) when $z=z_0$. To finish the proof we assume $z_0 \neq z$.
    
    Let $D=D_\delta(z_0)$ and $D_\epsilon (z)$ with $\epsilon$ small such that $D_\epsilon (z) \subset D_\delta(z_0)$. Let $\Omega' = D_\delta(z_0) \setminus D_\epsilon (z) \ssubset \Omega$.
    
        
    \begin{center}
    \begin{tikzpicture}[scale = 2]
    
        \draw[dotted][scale = 0.5] plot [smooth cycle] coordinates {(-3,3) (0,4) (4,3) (3,-3) (-4,-3.25) (-6,0)};
    
    
    
        \draw [color=red][pattern=north west lines][dotted][thick]  (0,0) circle (1.4);
        
        \draw [color=red][fill=white][dotted][thick] (50:0.7) circle (0.4);
        \draw[fill] (50:0.7) circle [radius=0.025];
        \node [below] at (50:0.7) {$z$};
        \draw [color=white][fill=white] (270:0.125)+(75-90:0.7) circle (0.085);
        
        
        \draw [color=white][fill=white] (270:0.125) circle (0.085);
        
        \draw[fill] (0,0) circle [radius=0.025];
        
        \node at (0,-0.125) {$z_0$};
        
        \draw [thick] (0,0) -- (75-90:1.4);
        \draw [shift=(50:0.7)][thick] (0,0) -- (75-90+120:0.4);
        \node at ($ (0,-0.125)+(75-90:0.7) $) {$\delta$};
        \node [above left] at (55:0.7) {$\epsilon$};
        
        \node [below] at (-2,-1.35) {$\Omega$};
        \node [below] at (270+30:1.4) {$\Omega'$};
        
        
        
        
    \end{tikzpicture}
    \end{center}
    
    Consider the function in $w\in\Omega'$, given by$\frac{f(w)}{w-z}$. This is holomorphic and $C^1$ in $\Omega'$ (since $z \notin \Omega'$). Applying (3) gives us:
    
    \begin{align*}
        0 = \int_{\partial \Omega'} \frac{f(w)}{w-z}  \dif w &= \int_{\partial D_\delta (z_0)} \frac{f(w)}{w-z}  \dif w - \underbrace{\int_{\partial D_\epsilon (z)} \frac{f(w)}{w-z}  \dif w}_{=2\pi i  \text{ by above}}\\
        &= \int_{\partial D_\delta (z_0)} \frac{f(w)}{w-z}  \dif w - 2\pi i f(w)\\
        &\Downarrow\\
        f(z) &= \frac{1}{2\pi i}\int_{\partial C_\delta} \frac{f(w)}{w-z} \dif w
    \end{align*}
    Which proves it in all cases.
    
    
\item[$(4) \Rightarrow (5):$] Fix a disc $D= D_r(z_0) \ssubset \Omega$. Let $z\in D$. By (4) then:

\begin{align*}
    f(z) &= \frac{1}{2\pi i} \int_{\partial D} \frac{f(w)}{w-z}  \dif w
\end{align*}
    
We expand $\frac{1}{w-z}$ in a power series centered at $z_0$ (so we look for a power series in ($z-z_0$)). First note that:

\begin{align*}
    \text{(since w $\in \partial D$)  } \abs{w-z_0} &= r\\
    \text{(since z $\in D \setminus \partial D$)  } \abs{z-z_0} &< r\\
    \implies \abs{\frac{z-z_0}{w-z_0}} &< 1
\end{align*}
Thus:
\begin{align*} 
    \frac{1}{w-z} = \frac{1}{w-z_0-(z-z_0)} &= \frac{1}{w-z_0} \cdot \frac{1}{1-   \underbrace{ \left( \frac{z-z_0}{w-z_0} \right)   }_{<1}}\\
    &= \frac{1}{w-z_0} \underbrace{ \sum_{n=0}^\infty \left( \frac{z-z_0}{w-z_0} \right)^n }_{\text{absolutely convergent}}
\end{align*}
    
Applying this to the Cauchy Integral Formula gives us:

\begin{align*}
     f(z) &= \frac{1}{2\pi i} \int_{\partial D} \left( \sum_{n=0}^\infty \left( \frac{z-z_0}{w-z_0} \right)^n \frac{f(w)}{w-z_0} \right) \dif w\\
     \text{(sum ab. conv.) } &= \sum_{n=0}^\infty \underbrace{ \left[  \frac{1}{2\pi i} \int_{\partial D}  \frac{f(w)}{(w-z_0)^{n+1}} \dif w  \right] }_{a_n} (z-z_0)^n \\
\end{align*}
    
Thus $f$ is analytic and is given by the formula claimed in (5). Finally $a_n = \frac{f^{(n)} (z_0)}{n!}$. Indeed if $f(z) = \sum_{n=0}^\infty a_n (z-z_0)^n$ absolutely convergent on some disc $D$, then we proved that $f(z)$ is holomorphic and you can differentiate this power series term by term. This gives:

\begin{align*}
    &f(z) = \sum_{n=0}^\infty a_n (z-z_0)^n\\
    \implies &f(z_0) = a_0\\
    &f'(z) = \sum_{n=1}^\infty n \cdot a_n (z-z_0)^{n-1}\\
    \implies &f'(z_0) = a_1\\
    &f''(z) = \sum_{n=2}^\infty n(n-1) \cdot a_n (z-z_0)^{n-2}\\
    \implies &f''(z_0) = 2\cdot a_2\\
    &\vdots \\
    &f^{(k)}(z) = \sum_{n=k}^\infty n(n-1)\hdots(n-k+1) \cdot a_n (z-z_0)^{n-k}\\
    \implies &f^{(k)}(z_0) = k! \cdot a_k\\
\end{align*}

This result holds for all convergent power series, thus it holds for this one in particular.
    
    
    
    
\item[$(1) \Rightarrow (2):$] Assume $f$ holomorphic on $\Omega$. $f$ satisfies the CR equations by (1.5). Use the local existence of antiderivatives. 

$\forall z \in \Omega \, \exists D = D_r(z) \ssubset \Omega$, by the local existence of antiderivatives, $\exists F: \Omega \rightarrow \mathbb{C}$ holomorphic and $F' = f$. $F$ is holomorphic and $C^1$ (both of its derivatives are continuous) since $\frac{\partial F}{ \partial \overline{z}} =0$ and $\frac{\partial F}{ \partial z} =f(z)$ which is $C^0$ (since holomorphic functions are continuous). Thus $F$ satisfies (2) on $D$, and thus $F$ satisfies (5) and is analytic on $D$. Thus so is $f= F'$ (since the derivative of a convergent power series is also a convergent power series).
   
   
\item[$(5) \Rightarrow (1):$] This was discussed when complex power series were discussed, and is a direct application of theorem (2.3).
    
    
\end{enumerate}

\end{proof} 


\begin{note}
In statement (2), (3), and (4), $f\in C^1(\Omega)$ means that $f\in C^1$ in the real sense. That is to say that for $u,v$ such that $f=u+iv$ then $u,v \in C^1$.
\end{note}