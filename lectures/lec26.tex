%FILL IN THE RIGHT INFO.
%\lecture{**LECTURE-NUMBER**}{**DATE**}
\unchapter{Lecture 26}
\lecture{26}{December 1}
\setcounter{section}{0}
\setcounter{theorem}{0}

% **** YOUR NOTES GO HERE:


We now finally prove the functional equation for $\xi$. First we start with several preliminary results.

\section{Functional Equation 3}

Let $f: \R \to \C$ a smooth function that decays at least quadratically at $\infty$, so that $\sum_{n=-\infty}^{\infty} f(\theta+n)$ converges $\forall \theta \in \R$. Then recall that the Fourier transform of this function, for all $\xi \in \R$, is:
\begin{align*}
    \hat{f}(\xi) \defas \int_{-\infty}^\infty e^{-2 \pi i x \xi} f(x) \dif x.
\end{align*}
This integral converges since $e^{-2 \pi i x \xi}$ has modulus $1$, and since $f(x)$ decays quadratically.

\begin{proposition}[Poisson Summation Formula]\label{prop:poisson-sum}
Let $f$ be a function that has a defined Fourier transform. Then $\forall \theta \in \R$:
\begin{align*}
    \sum_{n=-\infty}^\infty f(\theta + n) = \sum_{n=-\infty}^\infty \hat{f}(n) \cdot e^{2 \pi i n \theta}.
\end{align*}
\end{proposition}
\begin{proof}
Let $\theta \in \R$. Let $\psi(\theta) \defas \sum_{n=-\infty}^{\infty} f(n+\theta)$. $\psi : \R \to \C$ is smooth, and has the property that $\psi(\theta+1) = \psi(\theta) $. We use without proof that we can write $\psi$ as a Fourier Series:
\begin{align*}
    \psi(\theta) = \sum_{n=-\infty}^{\infty} c_n \cdot e^{2 \pi i n \theta},
\end{align*}
where
\begin{align*}
    c_n &= \int_0^1 \psi(\theta) \cdot e^{-2 \pi i n \theta} \dif \theta \\ &= \int_0^1 \sum_{k=-\infty}^{\infty} f(k+\theta) \cdot e^{-2 \pi i n \theta} \dif \theta \\&=  \sum_{k=-\infty}^{\infty} \int_0^1 f(k+\theta) \cdot e^{-2 \pi i n \theta} \dif \theta \\ \text{(change var $\theta \to \theta +k$) } &= \sum_{k=-\infty}^{\infty} \int_k^{k+1} f(\theta) \cdot e^{-2 \pi i n \theta} \dif \theta \\
    &= \int_{-\infty}^{\infty} f(\theta) \cdot e^{-2 \pi i n \theta} \dif \theta \\&= \hat{f}(n).
\end{align*}
Thus we have that:
\begin{align*}
    \sum_{n=-\infty}^{\infty} f(n+\theta) = \psi(n) = \sum_{n=-\infty}^{\infty} \hat{f}(n) \cdot e^{2 \pi i n \theta}.
\end{align*}
And we are done.
\end{proof}

\subsection{The Theta Function}

\begin{definition}[Theta Function]
We define, for $u \in (0,\infty) \subset \R$, the \textbf{theta function} as:
\begin{align*}
    \theta(u) = \sum_{n=-\infty}^{+\infty} e^{- \pi n^2 u}.
\end{align*}
\end{definition}
The summand decays to $0$ super-exponentially fast, and thus this infinite sum converges.

Recall the definition of $\xi$:
\begin{align*}
    \xi(s) \defas \pi^{-\frac{s}{2}} \cdot \Gamma\br{\frac{s}{2}} \cdot \zeta(s).
\end{align*}

\begin{lemma}\label{lem:zeta-theta-integral}
For $\Re(s) > 1$ we can write:
\begin{align*}
    \xi(s) = \frac{1}{2} \int_0^\infty u^{\frac{s}{2} - 1} \br{\theta(u) - 1} \dif u.
\end{align*}
\end{lemma}

\begin{note}
Notice that this definition is similar to the Gamma Function. The $\theta(u) - 1$ takes the role of the $e^{-t}$ in the definition of $\Gamma$, while the $u^{\frac{s}{2} - 1} $ is some slightly different power of the $u$ that we find in the definition of $\Gamma$.
\end{note}

\begin{proof}[\ref{lem:zeta-theta-integral}]
We first note that
\begin{align*}
    \theta(u) &= \sum_{n=-\infty}^{+\infty} e^{- \pi n^2 u} = 2 \cdot \sum_{n=1 }^{\infty} e^{- \pi n^2 u} + 1.\\
    \implies  \sum_{n=1 }^{\infty}& e^{- \pi n^2 u} = \frac{\theta(u) - 1}{2}.
\end{align*}

Notice that for $n  \neq 0$ (otherwise the change of variable is not well-defined), using the change of variable $t = \pi n^2 u$, $\dif u = \frac{\dif t}{ \pi n^2}$ we have:
\begin{align*}
    \int_0^\infty e^{- \pi n^2 u} \cdot  u^{\frac{s}{2} -1} \dif u &= \bigg( \underbrace{ \int_0^\infty e^{-t} t^{\frac{s}{2} - 1} \dif t}_{\Gamma\br{\frac{s}{2}}} \bigg) \cdot \br{\pi n^2}^{- \frac{s}{2}} \\
    &= \Gamma\br{\frac{s}{2}} \cdot \pi^{- \frac{s}{2}} \cdot n^{-s}.
\end{align*}

Then summing $\sum_{n=1}^\infty$ yields:
\begin{align*}
    \xi(s) = \Gamma\br{\frac{s}{2}} \cdot \pi^{- \frac{s}{2}} \cdot \sum_{n=1}^\infty   n^{-s} &= \int_0^\infty \bigg( \underbrace{ \sum_{n=1}^\infty  e^{- \pi n^2 u} }_{\frac{\theta(u) - 1}{2}}\bigg) \cdot  u^{\frac{s}{2} -1} \dif u\\
    &=\frac{1}{2} \int_0^\infty u^{\frac{s}{2} - 1} \br{\theta(u) - 1} \dif u.
\end{align*}
And we are done.
\end{proof}

\isubsection{PROP: Theta Functional Equation}

\begin{proposition}\label{prop:theta-func-eq}
For all $u > 0$ we have
\begin{align*}
    \theta\big(\tfrac{1}{u}\big) = u^{\frac{1}{2}} \cdot \theta(u).
\end{align*}
\end{proposition}

\begin{proof}
We wish to show, for all $u>0$, that:
\begin{align*}
    \theta(u) = \sum_{n=-\infty}^{+\infty} e^{- \pi n^2 u} = u^{- \frac{1}{2}} \sum_{n=-\infty}^{+\infty} e^{- \pi \frac{n^2}{u}} = u^{- \frac{1}{2}} \cdot  \theta\big(\tfrac{1}{u}\big).
\end{align*}
We apply proposition (\ref{prop:poisson-sum}). Consider the special case $\theta = 0$ in the formula, which yields:
\begin{align*}
    \sum_{n=-\infty}^\infty f( n) = \sum_{n=-\infty}^\infty \hat{f}(n).
\end{align*}

We let $f(x) = e^{-\pi u x^2}$, with $u>0$. Then recall that we found in example (\ref{ex:fourier-trans-gaussian}) that:
\begin{align*}
    \int_{-\infty}^\infty e^{-\pi x^2} e^{-2\pi i x \xi}  \dif x = e^{-\pi \xi ^2 }.
\end{align*}
Then change variable, letting $y = \frac{x}{\sqrt{u}}$. Then:
\begin{align*}
    \int_{-\infty}^\infty e^{-\pi x^2} e^{-2\pi i x \xi}  \dif x &= \sqrt{u} \int_{-\infty}^\infty e^{-\pi u y^2} e^{-2\pi i y \xi \sqrt{u}}  \dif y.
\end{align*}
Renaming $\xi \sqrt{u} \xrightarrow{} \xi$ and $y \xrightarrow{} x$ yields:
\begin{align*}
    \hat{f}(\xi)=\int_{-\infty}^\infty e^{-\pi u x^2} e^{-2\pi i x \xi}  \dif x = \frac{1}{\sqrt{u}} e^{- \pi \frac{\xi^2}{u}}.
\end{align*}
Then applying proposition (\ref{prop:poisson-sum}) for $f(x) = e^{- \pi u x^2}$ gives that:
\begin{align*}
    \theta(u) = \sum_{n=-\infty}^\infty e^{- \pi u n^2} \overset{PSF}{=} \frac{1}{\sqrt{u} } \sum_{n=-\infty}^\infty e^{- \pi \frac{n^2}{u}} = u^{- \frac{1}{2}} \cdot  \theta\big(\tfrac{1}{u}\big).
\end{align*}
And we are done.

\end{proof}







\begin{remark}\label{rem:theta-func-eq-different-form}
We can rewrite proposition (\ref{prop:theta-func-eq}) as (for $u>0$):
\begin{align*}
    \frac{\theta(u) - 1}{2} = u^{- \frac{1}{2}} \cdot \frac{\theta\big(\tfrac{1}{u}\big) - 1}{2} + \frac{1}{2 u^{\frac{1}{2}}} - \frac{1}{2}.
\end{align*}
\end{remark}

We now finally prove the functional equation for $\xi$.\\

\begin{proof}[\ref{thm:r-func-eq}]
By lemma (\ref{lem:zeta-theta-integral}) we write:
\begin{align*}
    \xi(s) &= \int_0^\infty u^{\frac{s}{2} - 1} \br{\frac{\theta(u) - 1}{2}} \dif u\\
    &= \int_0^1 u^{\frac{s}{2} - 1} \br{\frac{\theta(u) - 1}{2}} \dif u + \int_1^\infty u^{\frac{s}{2} - 1} \br{\frac{\theta(u) - 1}{2}} \dif u\\
    \text{(rmk (\ref{rem:theta-func-eq-different-form})) } &= \int_0^1 u^{\frac{s}{2} - 1} \br{u^{- \frac{1}{2}} \cdot \frac{\theta\big(\tfrac{1}{u}\big) - 1}{2} + \frac{1}{2 u^{\frac{1}{2}}} - \frac{1}{2}} \dif u + \int_1^\infty u^{\frac{s}{2} - 1} \br{\frac{\theta(u) - 1}{2}} \dif u.
\end{align*}
We consider each part of the left integral:
\begin{align*}
    \int_0^1 u^{\frac{s}{2} - 1} \br{u^{- \frac{1}{2}} \cdot \frac{\theta\big(\tfrac{1}{u}\big) - 1}{2} + \frac{1}{2 u^{\frac{1}{2}}} - \frac{1}{2}} \dif u &= \int_0^1 u^{\frac{s}{2} - 1} \br{u^{- \frac{1}{2}} \cdot \frac{\theta\big(\tfrac{1}{u}\big) - 1}{2}} \dif u && \circled{1} \\& + \int_0^1 u^{\frac{s}{2} - 1} \br{\frac{1}{2 u^{\frac{1}{2}}}} \dif u && \circled{2} \\& - \int_0^1 u^{\frac{s}{2} - 1} \br{\frac{1}{2}} \dif u && \circled{3}
\end{align*}
then we solve each individually.
\begin{enumerate}
    \item[\circled{1} :] We use the change of variable $v = \frac{1}{u}, \, \dif v = - \frac{1}{u^2} \dif u$.
    \begin{align*}
        \int_0^1 u^{\frac{s}{2} - 1} \br{u^{- \frac{1}{2}} \cdot \frac{\theta\big(\tfrac{1}{u}\big) - 1}{2}} \dif u &= \int_1^\infty v^{-\frac{s}{2} + 1} \cdot v^{ \frac{1}{2}} \cdot \frac{\theta (v ) - 1}{2} \cdot v^{-2} \, \dif v\\
        &= \int_1^\infty v^{-\frac{s}{2} - \frac{1}{2}} \cdot \frac{\theta (v ) - 1}{2} \dif v.
    \end{align*}
    Note that this is almost exactly the same as the second part of the integral above.
    \item[\circled{2} :] This is straightforward to compute.
    \begin{align*}
        \int_0^1 u^{\frac{s}{2} - 1} \br{\frac{1}{2 u^{\frac{1}{2}}}} \dif u &= \frac{1}{2} \int_0^1 u^{\frac{s}{2} - \frac{3}{2}} \dif u\\
        &= \frac{1}{s-1}.
    \end{align*}
    \item[\circled{3} :] This is also straightforward.
    \begin{align*}
        \int_0^1 u^{\frac{s}{2} - 1} \br{\frac{1}{2}} \dif u = \frac{1}{s}.
    \end{align*}
\end{enumerate}
Thus we have, for $\Re(s) > 1$, that:
\begin{align*}
    \xi(s) &= \int_1^\infty u^{-\frac{s}{2} - \frac{1}{2}} \cdot \frac{\theta (u ) - 1}{2} \dif u + \frac{1}{s-1} - \frac{1}{s} + \int_1^\infty u^{\frac{s}{2} - 1} \br{\frac{\theta(u) - 1}{2}} \dif u\\
    &= \int_1^\infty \br{\frac{\theta(u) - 1}{2}} \br{u^{-\frac{s}{2} - \frac{1}{2}} + u^{\frac{s}{2}-1}  } \dif u +\frac{1}{s-1} - \frac{1}{s}.
\end{align*}
Clearly $\frac{1}{s-1} - \frac{1}{s}$ is meromorphic with simple poles at $s=0$ and $s=1$. Then note that $\int_1^\infty \br{\frac{\theta(u) - 1}{2}} \br{u^{-\frac{s}{2} - \frac{1}{2}} + u^{\frac{s}{2} -1  }} \dif u$ is an entire function of $s$. The logic here is the same as the logic used in the proof of theorem (\ref{thm:r-zeta-extension-thm}), where it was shown that $\int_1^\infty \frac{t^{s-1}}{e^t - 1} \dif t$ is an entire holomorphic function in $s$.

Hence we have found the meromorphic extension of $\xi$ on $\C$ (with simple poles at $0,1$). Furthermore, the RHS of this formula is symmetric under $(s \leftrightarrow 1-s)$. Thus $\xi(s) = \xi(1-s)$, and we have proven that the functional equation for $\xi$ holds.

\end{proof}

\subsection{Locations of zeroes of Riemann Zeta}

We finish by proving the non-vanishing of $\zeta$ on the boundary of the critical strip $\set{z \in \C \mid 0 \leq \Re(z) \leq 1}$.

\begin{theorem}\label{thm:r-no-zeroes-on-bdy}
If $\Re(s) = 0$ or $\Re(s) = 1$, then $\zeta(s) \neq 0$.
\end{theorem}
\begin{note}
$s=1$ is a pole of $\zeta$, so it is not defined at it, but in any case it is certainly not a zero.
\end{note}
\begin{proof}[\ref{thm:r-no-zeroes-on-bdy}]
First assume that we have proved the theorem for $\Re(s) = 1$. Let $\Re(s) = 0$. Then $\Re(1-s) = 1$. Then we apply the functional equation $\zeta(s) = 2^s \cdot \pi^{s-1} \cdot \sin\br{\frac{\pi s}{2}} \cdot \Gamma(1-s) \cdot \zeta(1-s)$. $2^s \neq 0$, $\pi^{s-1} \neq 0$, and $\Gamma(1-s) \neq 0$. $\zeta(1-s) \neq 0$ by the assumption that we already proved the statement for $\Re(s)=1$. $\sin \big( \frac{\pi s}{2} \big)$ only vanishes (for $\Re(s) = 0$) at $s=0$, but $\zeta(1)$ is a pole, and cancels it out (that is to say that $\sin \big( \frac{\pi s}{2} \big) \cdot \zeta(1) \neq 0$). It follows that $\zeta(s) \neq 0$ on the line $\Re(s) = 0$.

Now let $\Re(s) = 1$. We introduce two lemmas.
\begin{lemma}\label{lem:r-no-zeroes-on-bdy-1}
Fix the principal branch of $\log(z)$ on $\Re(z) > 0$. Then for $\Re(s) > 1$, we have:
\begin{align*}
    \log\br{\zeta(s)} = \sum_{n=0}^\infty c_n \cdot n^{-s} \;\;\;\;\; \text{where $c_n \geq 0$}.
\end{align*}
Taking the $\log$ of $\zeta(s)$ is okay, since $\zeta(s) \neq 0$ for $\Re(s) > 1$.
\end{lemma}
\begin{proof}
This follows from Euler's Product Formula.
Noting that for $\abs{z} < 1$:
\begin{align*}
    \log \br{ \frac{1}{1-z}} = \sum_{m=1}^\infty \frac{z^m}{m}
\end{align*}
we have that:
\begin{align*}
    \log\br{\zeta(s) } &= \log  \Bigg( \prod_{p \text{ prime} } \frac{1}{1-p^{-s}} \Bigg) \\&= \sum_{p } \log \Bigg( \frac{1}{1-p^{-s}}\Bigg) \\&= \sum_{p } \sum_{m=1}^\infty \frac{p^{-ms}}{m} \\&= \sum_{n=1}^\infty c_n \cdot n^{-s},\\
    \text{with }c_n = &\begin{cases} \frac{1}{m} &\text{ if $n=p^m$, $p$ prime,} \\ 0 &\text{ otherwise.}\end{cases}
\end{align*}
Notably, $c_n \geq 0$, and thus we are done.
\end{proof}

\begin{lemma}\label{lem:r-no-zeroes-on-bdy-2}
Let $s = \sigma + i t$, with $\sigma = \Re(s) > 1$. Then:
\begin{align*}
    \log \abs{ \zeta^3 (\sigma) \cdot \zeta^4 (\sigma+it) \cdot \zeta(\sigma + 2 it) } \geq 0.
\end{align*}
\end{lemma}

\begin{proof}
Note that:
\begin{align*}
    \log \abs{ \zeta^3 (\sigma) \cdot \zeta^4 (\sigma+it) \cdot \zeta(\sigma + 2 it) } &= 3 \log \abs{ \zeta (\sigma) } \\&\;\; + 4 \log \abs{ \zeta (\sigma+it) } \\&\;\; + \log \abs { \zeta(\sigma + 2 it) } \\&= 3 \Re \br{ \log(\zeta(\sigma))} \\&\;\;+ 4 \Re \br{ \log(\zeta(\sigma+i t))} \\&\;\;+ \Re \br{ \log(\zeta(\sigma+ it))}.
\end{align*}
But then note that:
\begin{align*}
    \Re (\zeta(s)) &= \sum_{n=1}^\infty \Re \big(n^{-s}\big) \\&= \sum_{n=1}^\infty \Re \br{ e^{-(\sigma + it) \log(n)}} \\&= \sum_{n=1}^\infty \Re \br{ e^{- \sigma \log(n)} \cdot e^{-i t \log(n)}  } \\&= \sum_{n=1}^\infty n^{- \sigma} \cos (t \log(n)).
\end{align*}


Now if we let $\theta_n = t \log(n)$, we have that, by lemma (\ref{lem:r-no-zeroes-on-bdy-1}), that:
\begin{align*}
    \log \abs{ \zeta^3 (\sigma) \cdot \zeta^4 (\sigma+it) \cdot \zeta(\sigma + 2 it) } &= \sum_{n=1}^\infty c_n \cdot n^{-\sigma} \br{ 3 + 4  \cos\br{\theta_n} + \cos\br{ 2 \theta_n}}\\
     &= \sum_{n=1}^\infty c_n \cdot n^{-\sigma} 2 \cdot \br{ 1 + \cos(\theta_n) }^2 \\ &\geq 0.
\end{align*}
Where the last equality comes from the fact that $\cos\br{ 2 \theta_n} = 2 \cos^2 ( \theta_n) - 1$.
\end{proof}

Now suppose for a contradiction that $\zeta$ has a zero on $\set{\Re(s) = 1}$. That is to say that $\zeta ( 1 + i t_0 ) = 0$ for some $t_0 \in \R$. Certainly then $t_0 \neq 0$, since $\zeta$ has a pole at $s=1$.

Consider $\abs{\zeta(\sigma+i t_0) }^4$. Then by assumption, $\zeta(\sigma+i t_0)$ vanishes as $\sigma \to 1^+$. Thus as $\sigma \to 1^+$ we have that:
\begin{align*}
    \abs{\zeta(\sigma+i t_0) }^4 \leq C (\sigma - 1)^4.
\end{align*}
Now consider $\abs{\zeta(\sigma)}^3$. Similarly, since $\zeta(1)$ is a simple pole, we have that as $\sigma \to 1^+$:
\begin{align*}
    \abs{\zeta(\sigma)}^3 \leq \frac{C}{(\sigma - 1)^3}.
\end{align*}
Finally, consider $\abs{\zeta(\sigma + 2 it) }$. We know that this point is not a pole since $t_0 \neq 0$. Thus as $\sigma \to 1^+$ we have that:
\begin{align*}
    \abs{\zeta(\sigma + 2 it) } \leq C.
\end{align*}

All together, we have that:
\begin{align*}
    \abs{ \zeta^3 (\sigma) \cdot \zeta^4 (\sigma+it) \cdot \zeta(\sigma + 2 it) } \leq C (\sigma - 1) \xrightarrow[]{ \sigma \to 1^+} 0.
\end{align*}
Thus:
\begin{align*}
    \log \abs{ \zeta^3 (\sigma) \cdot \zeta^4 (\sigma+it) \cdot \zeta(\sigma + 2 it) }  \xrightarrow[]{ \sigma \to 1^+} - \infty.
\end{align*}
Which is a clear contradiction to lemma (\ref{lem:r-no-zeroes-on-bdy-2}). Thus we are done.
\end{proof}




























% We now finally prove the functional equation for $\xi$.

% \begin{proof}
% Recall the definition of $\xi$:
% \begin{align*}
%     \xi(s) \defas \pi^{-\frac{s}{2}} \cdot \Gamma\br{\frac{s}{2}} \cdot \zeta(s)
% \end{align*}

% We define, for $u \in (0,\infty) \subset \R$, the theta function:
% \begin{align*}
%     \theta(u) \defas \sum_{n=-\infty}^{+\infty} e^{- \pi n^2 u}
% \end{align*}
% The summand decays to $0$ super-exponentially fast, and thus this infinite sum converges.\\

% We claim that for $\Re(s) > 1$ we can write:
% \begin{align*}
%     \xi(s) = \frac{1}{2} \int_0^\infty u^{\frac{s}{2} - 1} \br{\theta(u) - 1} \dif u
% \end{align*}
% Notice that this definition is similar to the Gamma Function. The $\theta(u) - 1$ takes the role of the $e^{-t}$ in the definition of $\Gamma$, while the $u^{\frac{s}{2} - 1} $ is some slightly different power of the $u$ that we find in the definition of $\Gamma$.\\

% To prove this, notice that for $n  \neq 0$ (otherwise the change of variable is not well-defined), and using the change of variable $t = \pi n^2 u$, $\dif u = \frac{\dif t}{ \pi n^2}$ we have:
% \begin{align*}
%     \int_0^\infty e^{- \pi n^2 u} \cdot  u^{\frac{s}{2} -1} \dif u &= \bigg( \underbrace{ \int_0^\infty e^{-t} t^{\frac{s}{2} - 1} \dif t}_{\Gamma\br{\frac{s}{2}}} \bigg) \cdot \br{\pi n^2}^{- \frac{s}{2}} \\
%     &= \Gamma\br{\frac{s}{2}} \cdot \pi^{- \frac{s}{2}} \cdot n^{-s}
% \end{align*}
% Noting that
% \begin{align*}
%     \theta(u) &= \sum_{n=-\infty}^{+\infty} e^{- \pi n^2 u} = 2 \cdot \sum_{n=1 }^{\infty} e^{- \pi n^2 u} + 1
% \end{align*}
% Then summing $\sum_{n=1}^\infty$ yields:
% \begin{align*}
%     \xi(s) = \Gamma\br{\frac{s}{2}} \cdot \pi^{- \frac{s}{2}} \cdot \sum_{n=1}^\infty   n^{-s} &= \int_0^\infty \bigg( \underbrace{ \sum_{n=1}^\infty  e^{- \pi n^2 u} }_{\frac{\theta(u) - 1}{2}}\bigg) \cdot  u^{\frac{s}{2} -1} \dif u\\
%     &=\frac{1}{2} \int_0^\infty u^{\frac{s}{2} - 1} \br{\theta(u) - 1} \dif u
% \end{align*}
% And our claim has been proven.\\

% We make a second key
% \end{proof}