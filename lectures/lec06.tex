%FILL IN THE RIGHT INFO.
%\lecture{**LECTURE-NUMBER**}{**DATE**}
\unchapter{Lecture 6}
\lecture{6}{September 22}
\setcounter{section}{0}
\setcounter{theorem}{0}

% **** YOUR NOTES GO HERE:

Double reviewed

\section{Cauchy's Theorem Continued}

This lecture the proof of Cauchy's Theorem was continued. The rest of the proof was put with the first part of the proof in lecture 5.

We start this lecture with a corollary of Cauchy's Theorem, preceded by a rewording of a previous theorem.

\begin{corollary}[Local Existence of Antiderivatives version 2]\label{cor:local-prim-2}
$ \oic $ open, $\foc$ continuous such that $\int_{\partial T} f(z) \dif z = 0$, $ \forall$ $T \subset \om$ triangle. Then $\forall$ disks $D\subset \Omega$ $\exists F:D\rightarrow \C$ holomorphic $C^1$ s.t. $F'(z)=f(z) \, \forall z\in D$. Thus an antiderivative of $f$ exists.
\end{corollary}

\begin{proof}
The proof for corollary (\ref{cor:local-prim}) works just as well using these assumptions.
\end{proof}

\isubsection{COR: Morera's Theorem}

\begin{corollary}[Morera's Theorem]\label{cor:morera}
$\Omega \subset \mathbb{C}$ open connected, $f:\Omega \rightarrow \mathbb{C}$ continuous such that $\int_{\partial T} f(z) \dif z = 0$, $ \forall$ T triangle. Then $f$ is holomorphic.
\end{corollary}

\begin{note}
This is the converse of theorem (\ref{thm:goursat}) (Goursat's Theorem). The proof is similar to to the proof of corollary (\ref{cor:local-prim-2}) with some slight modifications.
\end{note}






\begin{proof} Now Morera follows. Since corollary (\ref{cor:local-prim-2}) applies to $f$, we get that it has antiderivatives, so $\exists F$ s.t. $F'(z)=f(z)$. We know that $F$ is analytic, thus so is $f$. It follows that $f$ is holomorphic.

\end{proof}





We now use Cauchy's Theorem to compute some real variable integrals. Computing real integrals is one of the original motivations for exploring complex analysis and complex integrals.

\isubsection{EX: Fourier Transform}
\begin{example}[Fourier Transform of Gaussian]\label{ex:fourier-trans-gaussian}

The goal here is to compute the Fourier transform in $\R$ of the Gaussian $e^{- \pi x^2}$.  That is to say given $\xi \in \R$ we want to compute:
\begin{align*}
    \int_{-\infty}^{\infty} e^{- \pi x^2} e^{- 2 \pi i x \xi} \dif x.
\end{align*}

First let us dispense with the case where $\xi = 0$, that is to say compute:
\begin{align*}
    \int_{-\infty}^{\infty} e^{- \pi x^2} \dif x.
\end{align*}

Observe that if you square it, it becomes a double integral:
\begin{align*}
    \left(  \int_{-\infty}^{\infty} e^{- \pi x^2}  \dif x \right)^2 = \int_{\R ^2}  e^{- \pi( x^2+y^2)}  \dif A.
\end{align*}

Changing it to polar coordinates, with $\dif A = r \dif \theta \dif r $ and $x^2+y^2 = r^2$:
\begin{align*}
    \int_{\R ^2}  e^{- \pi( x^2+y^2)}  \dif A &= \int_{0}^{\infty} \int_{0}^{2 \pi} e^{- \pi r^2}  r \dif \theta \dif r \\
    &= 2 \pi \int_{0}^{\infty}  e^{- \pi r^2}  r \dif r \\
    &= -e^{-\pi r^2} \Bigr|_0^\infty = 1.
\end{align*}

Now let $\xi > 0$. Consider $f(z) =  e^{- \pi z^2}$, holomorphic on $\C$. We use Cauchy's Theorem for $f$ on a well chosen contour:

\begin{center}
\begin{tikzpicture}[very thick,decoration={
    markings,
    mark=at position 0.6 with {\arrow{>}}}
    ]
    
    \draw [thin] [->] (-4,0)--(4,0);
    \draw [thin] [->] (0,-1)--(0,3);
    
    \draw[postaction={decorate}] [blue,line width=1.5pt] (-3,0) -- (3,0);
    \draw[postaction={decorate}] [blue,line width=1.5pt] (3,2) -- (-3,2);
    \draw[postaction={decorate}] [blue,line width=1.5pt] (3,0) -- (3,2);
    \draw[postaction={decorate}] [blue,line width=1.5pt] (-3,2) -- (-3,0);
    
    
    \draw (-3,0)[below] node {$-R$};
    \draw (3,0)[below] node {$R$};
    \draw (3,2)[above] node {$R+i\xi$};
    \draw (-3,2)[above] node {$-R+i\xi$};
    
    \draw (1,0)[below] node {$\gamma_1$};
    \draw (3,1)[right] node {$\gamma_2$};
    \draw (-1,2)[above] node {$\gamma_3$};
    \draw (-3,1)[left] node {$\gamma_4$};
    
    \draw (0,1)[right] node {$\Omega_R$};
    
\end{tikzpicture}
\end{center}

$\gamma_R$ is a piecewise smooth closed curve, and $\om_R$ is the rectangle inside $\gamma_R$. $f(z)$ is holomorphic on $\C$, so in particular $f(z)$ is holomorphic on $\overline{\om_R}$. By Cauchy's Theorem then:
\begin{align*}
      0  &= \int_{\gamma_R} f(z) \dif z.\\
    \text{Thus: } 0 &= \int_{\gamma_1} e^{-\pi z^2} \dif z + \int_{\gamma_2} e^{-\pi z^2} \dif z + \int_{\gamma_3} e^{-\pi z^2} \dif z + \int_{\gamma_4} e^{-\pi z^2} \dif z.
\end{align*}

Using $z(t) = t, \, z'(t) = 1,  \, t \in [-R,R]$:
\begin{align*}
    \int_{\gamma_1} e^{-\pi z^2} \dif z = \int_{-R}^R e^{-\pi t^2}  \dif t & \xrightarrow{R\to \infty} 1.
\end{align*}

Using $z(t) = R + it, \, z'(t) = i,  \, t \in [0,\xi]$:
\begin{align*}
    \int_{\gamma_2} e^{-\pi z^2} \dif z &= \int_{0}^\xi e^{-\pi (R+it)^2} i \dif t \\ 
    &= \int_{0}^\xi e^{-\pi (R^2-t^2+2itR)} i \dif t\\
    &= i \int_{0}^\xi e^{-\pi (R^2-t^2)} e^{-2 \pi i t R}  \dif t.
\end{align*}

Then:
\begin{align*}
    \abs{\int_{\gamma_2} e^{-\pi z^2} \dif z} &= \abs{\int_{0}^\xi e^{-\pi (R^2-t^2)} e^{-2 \pi i t R}  \dif t}\\
    & \leq \int_{0}^\xi \underbrace{\abs{ e^{-\pi (R^2-t^2)} }}_{e^x > 0 \forall x \in \R} \cdot \underbrace{ \abs{e^{-2 \pi i t R} }}_{=1} \dif t\\
    &= \int_{0}^\xi e^{-\pi (R^2-t^2)}  \dif t \\ &= \int_{0}^\xi e^{-\pi R^2} e^{\pi t^2}  \dif t \\ 
    & \leq e^{-\pi R^2}  \int_{0}^\xi e^{\pi \xi^2} \dif t \\ &= e^{-\pi R^2} e^{\pi \xi^2} \xi \xrightarrow{R\to \infty} .
\end{align*}

Similarly:
\begin{align*}
    &\int_{\gamma_4} e^{-\pi z^2} \dif z \xrightarrow{R\to \infty} 0.
\end{align*}

Finally using $z(t) = i\xi +t, \, z'(t) = 1,  \, t \in [-R,R]$ (this is a reverse parameterization):
\begin{align*}
    \int_{\gamma_3} e^{-\pi z^2} \dif z &= - \int_{-R}^R e^{-\pi (t+i\xi)^2}  \dif t\\
    &= - \int_{-R}^R e^{-\pi (t+i\xi)^2}  \dif t\\
    & = - \int_{-R}^R e^{-\pi (t^2-\xi^2)} e^{-2\pi i t \xi}  \dif t\\
    &= - e^{\pi \xi ^2 } \int_{-R}^R e^{-\pi t^2} e^{-2\pi i t \xi}  \dif t.
\end{align*}

Letting $R \to \infty$, we get exactly what we wanted to evaluate in the beginning of the example. By the fact that the four integrals summed give zero we get that for $ \xi > 0$:
\begin{align*}
    - e^{\pi \xi ^2 } \int_{-\infty}^\infty &e^{-\pi t^2} e^{-2\pi i t \xi}  \dif t = -1.\\
    &\Downarrow\\
    \int_{-\infty}^\infty &e^{-\pi t^2} e^{-2\pi i t \xi}  \dif t = e^{-\pi \xi ^2 }.
\end{align*}

Similarly for $\xi < 0$ (apply a similar rectangle below the axis instead of above it):
\begin{align*}
    \int_{-\infty}^\infty e^{-\pi t^2} e^{-2\pi i t \xi}  \dif t = e^{-\pi \xi ^2 }.
\end{align*}

This implies that the Fourier transform of the Gaussian is itself.

\end{example}
\isubsection{EX: An Integral}
\begin{example}

In this example we want to compute:
\begin{align*}
    \int_0^\infty \frac{1-cos(x)}{x^2} \dif x.
\end{align*}
    
We use $f(z) = \frac{1-e^{iz}}{z^2}$, which is holomorphic on $\C \setminus \{0\}$. Then with $x\in \R$:
\begin{align*}
    \Re(f(x)) = \Re \left( \frac{1-e^{ix}}{x^2} \right) = \frac{1-cos(x)}{x^2}.
\end{align*}

Now apply Cauchy to the following contour $\gamma_{R,\epsilon}$:

\begin{center}
\begin{tikzpicture}[very thick,decoration={
    markings,
    mark=at position 0.5 with {\arrow{>}}}
    ]
    \clip (-4,-1.1) rectangle (4,4);
    %\draw (-4,-1) rectangle (4,4);
    \draw[postaction={decorate}][rotate = -120] [blue,line width=1.5pt] (0,0) circle [radius=3];
    \draw[postaction={decorate}][xscale=-1][rotate = -60] [blue,line width=1.5pt] (0,0) circle [radius=1];
    
    \draw[color=white][fill=white] (-3.5,0) rectangle (3.5,-2);
    
    
    \draw [thin] [->] (-4,0)--(4,0);
    \draw [thin] [->] (0,-1)--(0,4);
    
    \draw[postaction={decorate}] [blue,line width=1.5pt] (1-0.026,0) -- (3+0.026,0);
    \draw[postaction={decorate}] [blue,line width=1.5pt] (-3-0.026,0) -- (-1+0.026,0);
    
    %\draw[postaction={decorate}] [blue,line width=1.5pt] (-3,0) -- (3,0);
    %\draw[postaction={decorate}] [blue,line width=1.5pt] (3,2) -- (-3,2);
   % \draw[postaction={decorate}] [blue,line width=1.5pt] (3,0) -- (3,2);
    %\draw[postaction={decorate}] [blue,line width=1.5pt] (-3,2) -- (-3,0);
    
    
    \draw (-3,0)[below] node {$-R$};
    \draw (3,0)[below] node {$R$};
    \draw (-1,0)[below] node {$-\epsilon$};
    \draw (1,0)[below] node {$+\epsilon$};
    
    \draw[thick] (-0.1,0.1) -- (0.1,-0.1);
    \draw[thick] (0.1,0.1) -- (-0.1,-0.1);
    
    \draw (-2,0)[below] node {$\gamma_1$};
    \draw (2,0)[below] node {$\gamma_3$};
    \draw (120:1)[above] node {$\gamma_2$};
    \draw (120:3)[above] node {$\gamma_4$};
    
    \draw (0,1.75) [right] node {$\Omega_{R,\epsilon}$};
    
    
    
\end{tikzpicture}
\end{center}


$f$ is holomorphic on $\Omega_{R,\epsilon} \, \forall R > \epsilon >0 $. Thus Cauchy's Theorem applies and gives:
\begin{align*}
    0 = \int_{\gamma_1} f(z) \dif z +\int_{\gamma_2} f(z) \dif z +\int_{\gamma_3} f(z) \dif z +\int_{\gamma_4} f(z) \dif z .
\end{align*}

The $\gamma_1$ and $\gamma_3$ terms are useful to us since taking the real part and $\epsilon \to 0, R \to \infty$ gives us the integral we're looking for over the real line (useful to us since this function is symmetric).

Now consider the $\gamma_4$ term, $\int_{\gamma_4} f(z) \dif z $.

Consider $z=x+iy, \, y>0 $. Then $\abs{e^{iz}} = \abs{e^{ix}} \abs{ e^{-y}} = \abs{ e^{-y}} \leq 1 $ (does not apply if $z$ is in the lower half-plane). Then:
\begin{align*}
    \abs{f(z)} &= \frac{\abs{1-e^{iz}}}{\abs{z}^2}\\
    & \leq \frac{1+\abs{e^{iz}}}{\abs{z}^2} \leq \frac{2}{\abs{z}^2}.
\end{align*}

Now since $\gamma_4$ is contained in the upper half-plane:
\begin{align*}
   \abs{ \int_{\gamma_4} f(z) \dif z } &\leq L(\gamma_4) \sup_{z\in \gamma_4} \abs{f(z)}\\
   &\leq \pi R \cdot \frac{2}{R^2} \xrightarrow[]{R\to \infty} 0.
\end{align*}


Now consider the $\gamma_2$ term, $\int_{\gamma_2} f(z) \dif z $. Paramaterize $\gamma_2$ (backwards) by $z(t) = \epsilon e^{it}, t\in [0,\pi]$. Now:
\begin{align*}
    e^{iz} &= 1 + iz + \frac{(iz)^2}{2} + \frac{(iz)^3}{3!} + \hdots\\
   1- e^{iz}  &=  -iz - \frac{(iz)^2}{2} - \frac{(iz)^3}{3!} - \hdots\\
   &\Downarrow\\
   \frac{ 1- e^{iz}}{ z^2} &= - \frac{i}{z} - \frac{(iz)^2}{2z^2} - \frac{(iz)^3}{3!z^2} - \hdots\\
   &= - \frac{i}{z} + g(z),\\
   \text{with } \abs{g(z)} &\leq C \text{ for some C as } z \to 0.
\end{align*}

Note that, since $\sup_{z\in \gamma_2} (g(z))$ is bounded:
\begin{align*}
    \abs{\int_0^\pi g(z) \cdot i \epsilon e^{it} \dif t} \leq \sup_{z\in \gamma_2} (g(z)) \cdot \epsilon \pi \xrightarrow[]{\epsilon \to 0} 0.
\end{align*}


And thus:
\begin{align*}
    \int_{\gamma_2} \underbrace{\frac{ 1- e^{iz}}{ z^2}}_{f(z)} \dif z &= - \int_0^{\pi} f(\epsilon e^{it} ) \cdot i \epsilon e^{it} \dif t\\
    &= + \int_0^{\pi} \frac{i}{ \epsilon e^{it}} i \epsilon e^{it} \dif t - \int_0^\pi g(z) \cdot i \epsilon e^{it} \dif t\\
   \text{(letting $\epsilon \to 0$) } &= -\pi.
\end{align*}

Thus finally, noting that $\int_{\gamma_1} f(z) \dif z + \int_{\gamma_3} f(z) \dif z = \int_\R \frac{ 1- e^{ix}}{ x^2} \dif x$:
\begin{align*}
    0 &= \sum_{i=1}^4 \int_{\gamma_i} f(z) \dif z \\
    \text{($\epsilon \to 0$, $R \to \infty$) }&= \int_\R \frac{ 1- e^{ix}}{ x^2} \dif x - \pi. \\
   &\Downarrow\\
    \Im \left( \int_\R \frac{ 1- e^{ix}}{ x^2} \dif x \right) &=  0.\\
    \Re \left( \int_\R \frac{ 1- e^{ix}}{ x^2} \dif x \right) &=  \pi = \int_{-\infty}^{\infty} \frac{ 1- cos(x)}{ x^2} \dif x.\\
    &\Downarrow \text{ even function}\\
    \int_0^{\infty} \frac{ 1- cos(x)}{ x^2} \dif x &= \frac{\pi}{2}.\\
\end{align*}

And we are done.

\end{example}




