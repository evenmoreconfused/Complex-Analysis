%FILL IN THE RIGHT INFO.
%\lecture{**LECTURE-NUMBER**}{**DATE**}
\unchapter{Lecture 14}
\lecture{14}{October 20}
\setcounter{section}{0}
\setcounter{theorem}{0}

% **** YOUR NOTES GO HERE:

\section{Riemann Sphere Continued}

Let $S^2 = \set{ (\mathfrak{X}, \mathfrak{Y}, \mathfrak{Z}) \mid \mathfrak{X}^2 + \mathfrak{Y}^2 +\mathfrak{Z}^2  =1 } \subset \R^3$ be the unit sphere centered at $0$. Last lecture we talked about the stereographic projection from $N = (0,0,1)$ and $S = (0,0,-1)$. Let $U_1 = S^2 \setminus \{ (0,0,1) \}$ and $U_2 = S^2 \setminus \{ (0,0,-1) \}$. These are both open subsets of $S^2$.



\begin{center}
    
\begin{tikzpicture}[scale=0.5] % CENT

%% some definitions

\def\R{2.5} % sphere radius
\def\angEl{15} % elevation angle
\def\angAz{-10} % azimuth angle
\def\angPhi{-40} % longitude of point P
\def\angBeta{180-25} % latitude of point P

%% working planes

\pgfmathsetmacro\H{\R*cos(\angEl)} % distance to north pole
\tikzset{xyplane/.style={cm={cos(\angAz),sin(\angAz)*sin(\angEl),-sin(\angAz),
                              cos(\angAz)*sin(\angEl),(0,0)}}}
%\tikzset{xyplane/.style={cm={cos(\angAz),sin(\angAz)*sin(\angEl),-sin(\angAz),
%                              cos(\angAz)*sin(\angEl),(0,-\H)}}}
\LongitudePlane[xzplane]{\angEl}{\angAz}
\LongitudePlane[pzplane]{\angEl}{\angPhi}
\LatitudePlane[equator]{\angEl}{0}

%% draw xyplane and sphere

%\draw[xyplane] (-2*\R,-2*\R) rectangle (2.2*\R,2.8*\R);
\draw (0,0) circle (\R);

%% characteristic points

\coordinate (O) at (0,0);

\draw [xyplane] (-2.5*\R,-2*\R) rectangle (2*\R,2*\R);
\coordinate[mark coordinate] (N) at (0,\H);
\coordinate[mark coordinate] (S) at (0,-\H);
\path[pzplane] (\angBeta:\R) coordinate[mark coordinate] (P);
\path[pzplane] (\R,0) coordinate (PE);
\path[xzplane] (\R,0) coordinate (XE);
\path (PE) ++(0,0) coordinate (Paux); % to aid Phat calculation
%\path (PE) ++(0,-\H) coordinate (Paux); % to aid Phat calculation
\coordinate[mark coordinate] (Phat) at (intersection cs: first line={(N)--(P)},
                                        second line={(0,0)--(Paux)});
                                        
\coordinate[mark coordinate] (Phat2) at (intersection cs: first line={(S)--(P)},
                                        second line={(0,0)--(Paux)});

%% draw meridians and latitude circles

\DrawLatitudeCircle[\R]{0} % equator





%% draw lines and put labels

\draw (P) -- (N) +(-0.3ex,0.6ex) node[above right] {$\mathbf{N=(0,0,1)}$};

\draw (P) -- (S) +(-0.4ex,-0.4ex) node[below] {$\mathbf{S=(0,0,-1)}$};

\draw (P) -- (Phat) node[above left] {$\mathbf{(x,y,0)=Q}$};
\draw (P) node[above left] {$\mathbf{P}$};
\draw (Phat2) node[above right] {$\mathbf{R}$};




\end{tikzpicture}
\end{center}





Define $\phi_1:U_1 \to \C, \, P \mapsto Q$ to be the stereographic projection from $N$. We showed last time that:
\begin{align*}
\phi_1 (\mathfrak{X}, \mathfrak{Y}, \mathfrak{Z}) &= \frac{\mathfrak{X}}{1 - \mathfrak{Z}} + i \frac{ \mathfrak{Y}}{1 - \mathfrak{Z}}.\\
\phi_1^{-1} (x+iy) &= \bigg( \frac{2x}{x^2+y^2+1} , \,  \frac{2y}{x^2+y^2+1} , \, \frac{x^2+y^2-1}{x^2+y^2+1}  \bigg).
\end{align*}


Define $\phi_2:U_2 \to \C, \, P \mapsto \overline{R}$ to be the stereographic projection from $S$ composed with complex conjugation. Then:
\begin{align*}
\phi_2 (\mathfrak{X}, \mathfrak{Y}, \mathfrak{Z}) &= \frac{\mathfrak{X}}{1 + \mathfrak{Z}} - i \frac{ \mathfrak{Y}}{1 + \mathfrak{Z}}.
\end{align*}

$U_1 \cap U_2 = S^2 \setminus \set{N,S}$. It is clear that $\phi_1 (U_1 \cap U_2) = \phi_2(U_1 \cap U_2) = \C^*$. We showed in the last lecture that:
\begin{align*}
    \phi_2 \circ \phi_1^{-1} : \, \, \C^* &\to \C^*,\\
    z &\mapsto \frac{1}{z}.
\end{align*}

\begin{note}[If you know about manifolds already]
$\set{ (U_1, \phi_1) , (U_2, \phi_2) }$ is an atlas for $S^2$ as a Riemann surface, since $ \phi_2 \circ \phi_1^{-1} (z) = \frac{1}{z}$ is holomorphic.
\end{note}


\subsection{Holomorphic Maps on $S^2$}
We now want to define the concept of holomorphic maps from and to $S^2$.

\begin{definition}[Holomorphic Functions from $S^2$ to $\C$]

$\om \in S^2$ open (can be $S^2$ itself). Let $\foc$ a map of sets. We say that \textbf{$f$ is holomorphic} if on $\phi_1 ( U_1 \cap \om) \subset \C$ open, $F_1 \vcentcolon= f \circ \phi_1^{-1} : \phi_1(U_1 \cap \om) \to \C$ is holomorphic and if on $\phi_2 ( U_2 \cap \om) \subset \C$ open, $F_2 \vcentcolon= f \circ \phi_2^{-1} : \phi_2(U_2 \cap \om) \to \C$ is holomorphic.\\

Equivalently, $f$ is holomorphic if I have $F_1$ holomorphic on $\Tilde{\om} \subset \C$ holomorphic and $F_2 = f \circ \phi_2^{-1} = F \circ \phi_1 \circ \phi_2^{-1} = F \left( \frac{1}{z} \right)$ is holomorphic near $0$ (if $\Tilde{\om}$ contains a neighborhood of $\infty$).
\end{definition}


\begin{center}
    
\begin{tikzpicture}[scale=1] % CENT

\def\R{1} % sphere radius
\def\angEl{25} % elevation angle
%\def\angEl{15} % elevation angle
\def\angAz{-10} % azimuth angle
%\def\angAz{-10} % azimuth angle

\tikzset{xyplane/.style={cm={cos(\angAz),sin(\angAz)*sin(\angEl),-sin(\angAz), cos(\angAz) *sin(\angEl),(0,0)}}}

\LongitudePlane[xzplane]{\angEl}{\angAz}



% \draw[xyplane] (-2,-2) rectangle (2,2);
% \draw[xyplane] node at (0,0) {$\times$};



\pgfmathsetmacro\H{1.5*cos(\angEl)}


\draw (0,0) circle (1.5);
\DrawLatitudeCircle[1.5]{0} % equator

\draw (0,\H) node {$\times$};
\draw (0,-\H) node {$\times$};


%top right
\draw[xyplane] (-2+8,-2) rectangle (2+8,2);
%\draw[xyplane] node at (8,0) {$\times$};
%bottom left
\draw[xyplane] (-2,-2 -8) rectangle (2,2-8);
\draw[xyplane] node at (0,-8) {$\times$};
%bottom right
\draw[xyplane] (-2+8,-2 -8) rectangle (2+8,2-8);
\draw[xyplane] node at (0+8,-8) {$\times$};



%from sphere
\draw[xyplane][->] (0+2,0) -- (8-2-0.5,0);
\draw[xyplane][->] (0,0-2) -- (0,-8+2+0.5);

%from bottom left
\draw[xyplane][<->] (0+2+0.5,0-8) -- (8-2-0.5,0-8);


%from bottom right
\draw[xyplane][->] (8,-8+2+0.5) -- (8,0-2-0.5);



%across middle
% \draw[xyplane][->] (0+2,0-2) -- (8-2,-8+2);
% \draw[xyplane][->] (0+2,0-8+2) -- (8-2,0-2);

\draw[xyplane][->] ($ (8,0)+(225:8) $) -- ($ (0,-8)+(45:8) $);
\draw[xyplane][color = white][fill = white] (4,-4) circle (0.25);

\draw[xyplane][->] (-45:2.8) -- (-45:8);



\draw[xyplane][above] node at (4,0) {$f$};
\draw[xyplane][above] node at (4,-8) {$z \mapsto \frac{1}{z}$};

\draw[xyplane][right] node at (0,-4.75) {$\phi_1$};
\draw[xyplane][above right] node at (-45:7) {$\phi_2$};
\draw[xyplane][right] node at (8,-3.75) {$F_2$};
\draw[xyplane][above left] node at ($ (0,-8)+(45:7) $) {$F_1$};

\draw[xyplane][below right] node at (8+2,-1.5) {$\mathbb{C}$};
\end{tikzpicture}

\end{center}

% \begin{tikzpicture}
% for i in range [0,30, 60 , ... , 6000]
% \draw circle($1/i$) at 0;
% \end{tikzpicture}


\begin{definition}[Holomorphic Functions from $\C$ to $S^2$]

If $\oic$ open, $f: \om \to S^2$ a map of sets, we say that \textbf{$f$ is holomorphic} if both $\phi_1 \circ f: \om \to \C$ and $\phi_2 \circ f: \om \to \C$ are holomorphic.\\

Equivalently, $F = \phi_1 \circ f$ holomorphic and $\phi_2 \circ f = \phi_2 \circ \phi_1^{-1} \circ F = \frac{1}{F}$ holomorphic on $\om \cap \set{F \neq 0}$ ie $F$ has only poles, where $poles = f^{-1} (\text{north pole})  =f^{-1} ( `` \infty" )$.






















\begin{center}
    
\begin{tikzpicture}[scale=1] % CENT

\def\R{1} % sphere radius
\def\angEl{25} % elevation angle
%\def\angEl{15} % elevation angle
\def\angAz{-10} % azimuth angle
%\def\angAz{-10} % azimuth angle

\tikzset{xyplane/.style={cm={cos(\angAz),sin(\angAz)*sin(\angEl),-sin(\angAz), cos(\angAz) *sin(\angEl),(0,0)}}}

\LongitudePlane[xzplane]{\angEl}{\angAz}



% \draw[xyplane] (-2,-2) rectangle (2,2);
% \draw[xyplane] node at (0,0) {$\times$};



\pgfmathsetmacro\H{1.5*cos(\angEl)}


\draw (0,0) circle (1.5);
\DrawLatitudeCircle[1.5]{0} % equator

\draw (0,\H) node {$\times$};
\draw (0,-\H) node {$\times$};


%left
\draw[xyplane] (-2-8,-2) rectangle (2-8,2);
%\draw[xyplane] node at (-8,0) {$\times$};
%bottom
\draw[xyplane] (-2-4,-2 -8) rectangle (2-4,2-8);
%\draw[xyplane] node at (0,-8) {$\times$};
%top
\draw[xyplane] (-2-4,-2 +8) rectangle (2-4,2+8);
%\draw[xyplane] node at (0-4,8) {$\times$};



%from left
\draw[xyplane][->]  (-8+2+0.5,0) -- (0-2,0);
\draw[xyplane][->] (-8+1.5,-2-0.5) -- (-5.5,-8+2+0.5);
\draw[xyplane][->] (-8+1.5,2+0.5) -- (-5.5,+8-2-0.5);

%from sphere
\draw[xyplane][->] (-110:2) -- (-2.5,-8+2+0.5);
\draw[xyplane][->] (110:2) -- (-2.5,+8-2-0.5);





\draw[xyplane][above right] node at ($ (110:1) + (-2.5/2,5.5/2) $) {$\phi_1$};
\draw[xyplane][below right] node at ($ (-110:1) + (-2.5/2,-5.5/2) $) {$\phi_2$};

\draw[xyplane][above left] node at ($ (-6.5/2,+2.5/2) + (-5.5/2,+5.5/2) $) {$\phi_1 \circ f$};
\draw[xyplane][below left] node at ($ (-6.5/2,-2.5/2) + (-5.5/2,-5.5/2) $) {$\phi_2 \circ f$};

\draw[xyplane][above] node at (-4,0) {$f$};
\draw[above right] node at (50:1.5) {$S^2$};


\draw[xyplane][below right] node at (-8+2,-1.5) {$\mathbb{C}$};
\end{tikzpicture}

\end{center}




























\end{definition}

\begin{definition}[Holomorphic Functions from $S^2$ to $S^2$]

$\om \subset S^2$ open, $f:\om \to S^2$ a map of sets. \textbf{$f$ is holomorphic} if:
\begin{align*}
    \phi_1 \circ f \circ \phi_1^{-1} : \phi_1 \left( U_1 \cap U_1 \right) \to \phi_1 \left( U_1 \cap U_1\right),\\
    \phi_1 \circ f \circ \phi_2^{-1} : \phi_2 \left( U_1 \cap U_2 \right) \to \phi_1 \left( U_1 \cap U_2\right),\\
    \phi_2 \circ f \circ \phi_1^{-1} : \phi_1 \left( U_2 \cap U_1 \right) \to \phi_2 \left( U_2 \cap U_1\right),\\
    \phi_2 \circ f \circ \phi_2^{-1} : \phi_2 \left( U_2 \cap U_2 \right) \to \phi_2 \left( U_2 \cap U_2\right),\\
\end{align*}
are all holomorphic.

Equivalently, letting $F= \phi_1 \circ f \circ \phi_1^{-1}$ then $f$ holomorphic as a map $S^2 \to S^2$ if:
\begin{itemize}
    \item F holomorphic,
    \item $f:z \mapsto \infty$ then $F$ has a pole at $z$,
    \item $f:\infty \mapsto z$ then $F$ holomorphic at $\infty$,
    \item $f: \infty \mapsto \infty$ then $F$ has a pole at $\infty$.
\end{itemize}
\end{definition}










\begin{center}
    
\begin{tikzpicture}[scale=0.5] % CENT

\def\R{1} % sphere radius
\def\angEl{25} % elevation angle
%\def\angEl{25} % elevation angle
\def\angAz{-10} % azimuth angle
%\def\angAz{-10} % azimuth angle

\tikzset{xyplane/.style={cm={cos(\angAz),sin(\angAz)*sin(\angEl),-sin(\angAz), cos(\angAz) *sin(\angEl),(0,0)}}}

\LongitudePlane[xzplane]{\angEl}{\angAz}



\pgfmathsetmacro\H{1.5*cos(\angEl)}




\draw[xyplane] (-2,-2) rectangle (2,2);

\draw[xyplane] ($ (-2+6,-2) + (0,{-6*sin(\angAz)})$) rectangle ($ (2+6,2) + (0,{-6*sin(\angAz)})$);

\draw[xyplane] ($ (-2+6+8,-2) + (0,{-14*sin(\angAz)})$) rectangle ($ (2+6+8,2) + (0,{-14*sin(\angAz)})$);

\draw[xyplane] ($ (-2+12+8,-2) + (0,{-20*sin(\angAz)})$) rectangle ($ (2+12+8,2) + (0,{-20*sin(\angAz)})$);




\draw[xyplane] node (A) at ($(3,0) + (0,{-3*sin(\angAz)})$) {};
\draw[xyplane] node (B) at ($ (17,0) + (0,{-17*sin(\angAz)})$) {};
\draw node (Ap) at ($ (A) + (0,6) $) {};
\draw node (Bp) at ($ (B) + (0,6) $) {};

\draw[xyplane] node (C) at ($ (0,0) + (0,{-0*sin(\angAz)})$) {};
\draw[xyplane] node (D) at ($ (6,0) + (0,{-6*sin(\angAz)})$) {};
\draw[xyplane] node (E) at ($ (14,0) + (0,{-14*sin(\angAz)})$) {};
\draw[xyplane] node (F) at ($ (20,0) + (0,{-20*sin(\angAz)})$) {};

\draw node (Cp) at ($ (C) + (0,1) $) {};
\draw node (Dp) at ($ (D) + (0,1) $) {};
\draw node (Ep) at ($ (E) + (0,1) $) {};
\draw node (Fp) at ($ (F) + (0,1) $) {};



\DrawLatitudeCircleShiftOne[1.5]{0} % equator
\DrawLatitudeCircleShiftTwo[1.5]{0} % equator

\draw (Ap) circle (1.5);
\draw (Bp) circle (1.5);


\draw [->] ($ (Ap) + (0:1.75) $) -- ($ (Bp) + (180:1.75) $) node [midway, above] {$f$};
\draw [->] ($ (Ap) + (-120:1.75)$) -- (Cp);
\draw [->] ($ (Ap) + (-60:1.75)$) -- (Dp);
\draw [->] ($ (Bp) + (-120:1.75)$) -- (Ep);
\draw [->] ($ (Bp) + (-60:1.75)$) -- (Fp);

\draw [left] node at ($ (Ap) + (-120:3) + (-0.2,0) $) {$\phi_1$};
\draw [right] node at ($ (Ap) + (-60:3) + (0.2,0) $) {$\phi_2$};
\draw [left] node at ($ (Bp) + (-120:3) + (-0.2,0) $) {$\phi_1$};
\draw [right] node at ($ (Bp) + (-60:3) + (0.2,0) $) {$\phi_2$};


\end{tikzpicture}

\end{center}




































\begin{remark}
It follows that $f: S^2 \to S^2$ holomorphic $\iff$ $F=\phi_1 \circ f \circ \phi_1^{-1}$ meromorphic in $\C$ and meromorphic at $\infty$. By theorem (\ref{thm:char-rat}) this is equivalent to $F$ being a rational function ie $F$ is the ratio of two complex polynomials.
\end{remark}

\begin{exercise}
Compute $Aut(S^2)$. That is to say, characterize the bijective holomorphic maps from $S^2 $ to $S^2$.
\end{exercise}



\section{Homotopy and Contour Integrals}

Homotopies are often discussed in an introduction to algebraic topology. We will give a simple-minded yet perfectly rigorous introduction to homotopies.\\

A homotopy is in a sense a ``continuous family of continuous paths." This must be defined, but in our treatment we will insist on having piecewise smooth paths. Say $\gamma_0, \gamma_1$ are two piecewise smooth curves in $
C$. Up to reparameterization we may assume that both are parameterized by the same interval $t \in [a,b] \subset \R$. Assume that $\gamma_0(a) = \gamma_1(a) = z_0$ and $\gamma_0(b) = \gamma_1(b) = z_1$.

\begin{center}
\begin{tikzpicture}[decoration={
    markings,
    mark=at position 0.6 with {\arrow{>}}}
    ]
    \node[label={[above left] $z_0$}] (a3) at (0,0) {};
    \node[label={[above right]$z_1$}] (b3) at (10,0) {};
    \draw[fill] (a3) circle (0.08);
    \draw[fill] (b3) circle (0.08);
    \draw[postaction={decorate}][thick] (a3) to[out=50,in=150]node[above]{$\gamma_1$} (b3);
    % \foreach \o/\i in {40/160,30/170,20/180,10/190,-10/200}
       % \draw (a3) to[out=\o,in=\i]  (b3);
    \draw[postaction={decorate}][thick] (a3) to[out=-20,in=-130]node[below]{$\gamma_0$} (b3);
\end{tikzpicture}
\end{center}

\begin{definition}[Homotopic Curves]
Suppose that $\gamma_0, \gamma_1 \subset \om$ are two such curves. We say that \textbf{$\gamma_0$ and $\gamma_1$ are homotopic in $\om$} if $\exists F : [0,1] \times [a,b] \to \om$ continuous such that:
\begin{enumerate}
    \item $\forall s \in [0,1] , \, \Tilde{\gamma}_s(t) \defas F(s,t)$ is a piecewise smooth curve $\Tilde{\gamma}_s : [a,b] \to \om$,
    \item $\Tilde{\gamma}_0 = F(0,t) \equiv \gamma_0$ and $\Tilde{\gamma}_1 = F(1,t) \equiv \gamma_1$,
    \item $\Tilde{\gamma}_s(a) = z_0$ and $\Tilde{\gamma}_s(b) = z_1$ $\forall s \in [0,1]$.
\end{enumerate}

\begin{center}
\begin{tikzpicture}[decoration={
    markings,
    mark=at position 0.6 with {\arrow{>}}}
    ]
    \node[label={[above left] $z_0$}] (a3) at (0,0) {};
    \node[label={[above right]$z_1$}] (b3) at (10,0) {};
    \draw[fill] (a3) circle (0.08);
    \draw[fill] (b3) circle (0.08);
    \draw[postaction={decorate}][thick] (a3) to[out=50,in=150]node[above]{$\gamma_1$} (b3);
    \foreach \o/\i in {40/160,30/170,10/190,-10/200,-15/210}
       \draw (a3) to[out=\o,in=\i]  (b3);
    \draw[postaction={decorate}] (a3) to[out=20,in=-180]node[above]{$\gamma_s$} (b3);
    \draw[postaction={decorate}][thick] (a3) to[out=-20,in=-130]node[below]{$\gamma_0$} (b3);
\end{tikzpicture}
\end{center}

\end{definition}

We can think of $\Tilde{\gamma}_s$ as an interpolating family of piecewise smooth curves with similar beginning and ending points which ``vary continuously in S", where the ``continuous variation" is encoded in the condition that $F$ is continuous.


\begin{counterexample}[Moral Counterexample]
Let $\om = \C^*$.

\begin{center}
\begin{tikzpicture}[decoration={
    markings,
    mark=at position 0.6 with {\arrow{>}}}
    ]
    \node[label={[above left] $-1$}] (a3) at (0,0) {};
    \node[label={[above right]$1$}] (b3) at (10,0) {};
    \node[label={[above right]$0$}] (c3) at (5,0) {};
    \draw[shift=(0:5)][thick] (-0.1,0.1) -- (0.1,-0.1) (0.1,0.1) -- (-0.1,-0.1);
    \draw[fill] (a3) circle (0.08);
    \draw[fill] (b3) circle (0.08);
    \draw[postaction={decorate}][thick] (a3) to[out=50,in=150]node[above]{$\gamma_1$} (b3);
    % \foreach \o/\i in {40/160,30/170,20/180,10/190,-10/200}
       % \draw (a3) to[out=\o,in=\i]  (b3);
    \draw[postaction={decorate}][thick] (a3) to[out=-20,in=-130]node[below]{$\gamma_0$} (b3);
\end{tikzpicture}
\end{center}

These two curves should not be homotopic in $\om$, since $F$ must map into $\om$.
\end{counterexample}

\isubsection{PROP: Homotopy Invariance of Contour Integrals}
\begin{proposition}[Homotopy Invariance of Contour Integrals]\label{prop:hom-inv-int}

$\oic$ open, $\foc$ holomorphic, $\gamma_0,\gamma_1:[a,b] \to \om$ piecewise smooth curves that are homotopic in $\om$. Then:
\begin{align*}
    \int_{\gamma_0} f(z) \dif z = \int_{\gamma_1} f(z) \dif z. 
\end{align*}

\end{proposition}

\begin{example}
Consider the unit circle in $\om = \C^*$.

\begin{center}
\begin{tikzpicture}[thick,decoration={
    markings,
    mark=at position 0.5 with {\arrow{>}}}
    ]
    \draw[postaction={decorate}][xscale=1][rotate = 90] (0,0) circle [radius=1];
    \draw[postaction={decorate}][xscale=-1][rotate = -90] (0,0) circle [radius=1];
    
    \draw[fill] (-1,0) circle (0.08);
    \draw[fill] (1,0) circle (0.08);
    
    \draw (0,1)[below] node {$\gamma_1$};
    \draw (0,-1)[above] node {$\gamma_0$};
    \draw (-0.1,-0.02)[below left] node {$0$};
    \draw[thick] (-0.1,0.1) -- (0.1,-0.1);
    \draw[thick] (0.1,0.1) -- (-0.1,-0.1);
    
\end{tikzpicture}
\end{center}

Then $\gamma_0$ and $\gamma_1$ are not homotopic by the above proposition as if they were, letting $f(z) = \frac{1}{z}$:
\begin{align*}
    \int_{\gamma_0} \frac{1}{z} \dif z - \int_{\gamma_1} \frac{1}{z} \dif z = 0,
\end{align*}
but we already know that:
\begin{align*}
    \int_{\gamma_0} \frac{1}{z} \dif z - \int_{\gamma_1} \frac{1}{z} \dif z = \int_{\partial D_1(0)} \frac{1}{z} \dif z = 2 \pi i.
\end{align*}

It follows that $\gamma_0$ and $\gamma_1$ are not homotopic.
\end{example}

\begin{proof}[\ref{prop:hom-inv-int}]
Let $F$ be a homotopy between $\gamma_0$ and $\gamma_1$ in $\om$. let $K = F([0,1] \times [a,b]) \subset \om$. $[0,1] \times [a,b]$ is compact and $F$ is continuous, so $K$ is compact.

Since $K$ is compact and $\om$ is open, $\exists \varepsilon > 0$ such that $\forall z \in K, \, D_{3\varepsilon}(z) \subset \om$ (ie thickening $K$ by $3 \varepsilon$ still lies in $\om$). Also, since $F$ is continuous and $[0,1] \times [a,b]$ is compact, then $F$ is uniformly continuous on $[0,1] \times [a,b]$. Thus $\exists \delta >0$ s.t. for any $s_1, s_2 \in [0,1]$ with $\abs{s_1-s_2} < \delta$ we have $\sup_{t \in [a,b]} \abs{\gamma_{s_1}(t) - \gamma_{s_2}(t)} < \varepsilon $.



This means that $\gamma_{s_1}$ and $\gamma_{s_2}$ lie within distance epsilon of each other (for $s_1,s_2$ sufficiently small):


\begin{center}
\begin{tikzpicture}[decoration={
    markings,
    mark = between positions 0 and 1 step 0.11 with {\draw[fill] circle (0.02); \draw circle (0.7);},
    mark=at position 0.330 with {\draw (0,0) -- (110:0.35) node[right] {$\varepsilon$} -- (110:0.7);},
    }
    ]
    \node (a3) at (0,0) {};
    \node (b3) at (10,0) {};
    \draw[fill] (a3) circle (0.08);
    \draw[fill] (b3) circle (0.08);
    \draw[postaction={decorate}][thick] (a3) to[out=20,in=180]node[]{} (b3);
    % \foreach \o/\i in {40/160,30/170,20/180,10/190,-10/200}
       % \draw (a3) to[out=\o,in=\i]  (b3);
    \draw[thick] (a3) to[out=10,in=190]node[]{} (b3);
    \draw (45:1.8) node {$\gamma_{s_1}$};
    \draw (-15:1.8) node {$\gamma_{s_2}$};
\end{tikzpicture}
\end{center}


Fix any such $s_1, s_2$. We choose disks of radius $2 \varepsilon$: $D_0, \cdots ,D_n$, $n$ large, which cover $\gamma_{s_1} \cup \gamma_{s_2}$:

\begin{center}
\begin{tikzpicture}
    
    \node (z0) at (0,0) {};
    \node (z1) at (10,0) {};
    \draw[fill] (z0) circle (0.08);
    \draw[fill] (z1) circle (0.08);
    
    \draw[postaction=
        {decoration={markings,
        mark=at position 0.1 with {\draw node (a1) {};},
        mark=at position 0.2 with {\draw node (a2) {};},
        mark=at position 0.3 with {\draw node (a3) {};},
        mark=at position 0.4 with {\draw node (a4) {};},
        mark=at position 0.5 with {\draw node (a5) {};},
        mark=at position 0.6 with {\draw node (a6) {};},
        mark=at position 0.7 with {\draw node (a7) {};},
        mark=at position 0.8 with {\draw node (a8) {};},
        mark=at position 0.9 with {\draw node (a9) {};},
        },
        decorate}][thick] 
        (z0) to[out=20,in=180]node[]{} (z1);
        
    \draw[postaction=
        {decoration={markings,
        mark=at position 0.1 with {\draw node (b1) {};},
        mark=at position 0.2 with {\draw node (b2) {};},
        mark=at position 0.3 with {\draw node (b3) {};},
        mark=at position 0.4 with {\draw node (b4) {};},
        mark=at position 0.5 with {\draw node (b5) {};},
        mark=at position 0.6 with {\draw node (b6) {};},
        mark=at position 0.7 with {\draw node (b7) {};},
        mark=at position 0.8 with {\draw node (b8) {};},
        mark=at position 0.9 with {\draw node (b9) {};},
        },
        decorate}][thick] 
        (z0) to[out=10,in=190]node[]{} (z1);

    \node (c1) at ($(a1)!0.5!(b1)$) {};
    \node (c2) at ($(a2)!0.5!(b2)$) {};
    \node (c3) at ($(a3)!0.5!(b3)$) {};
    \node (c4) at ($(a4)!0.5!(b4)$) {};
    \node (c5) at ($(a5)!0.5!(b5)$) {};
    \node (c6) at ($(a6)!0.5!(b6)$) {};
    \node (c7) at ($(a7)!0.5!(b7)$) {};
    \node (c8) at ($(a8)!0.5!(b8)$) {};
    \node (c9) at ($(a9)!0.5!(b9)$) {};

    \def\r{1.4};
    \def\rs{0.08};
    \def\d{0.25};
    \draw[thick] (z0) circle (\r);
        \draw (z0)+(100:\r+0.25) node {$D_0$};
        % \draw (z0)+(0,\d) node {$z_0$};
        % \draw (z0)+(0,-\d) node {$w_0$};
        % \draw[fill] (a1) circle (\rs);
        % \draw (a1)+(0,\d) node {$z_1$};
        % \draw[fill] (b1) circle (\rs);
        % \draw (b1)+(0,-\d) node {$w_1$};
    \draw[thick] (c2) circle (\r);
        \draw (c2)+(100:\r+0.25) node {$D_1$};
        % \draw[fill] (a3) circle (\rs);
        % \draw (a3)+(0,\d) node {$z_2$};
        % \draw[fill] (b3) circle (\rs);
        % \draw (b3)+(0,-\d) node {$w_2$};
    \draw[thick] (c4) circle (\r);
        % \draw[fill] (a5) circle (\rs);
        % \draw (a5)+(0,\d) node {$a$};
        % \draw[fill] (b5) circle (\rs);
        % \draw (b5)+(0,-\d) node {$a$};
    \draw[thick] (c6) circle (\r);
        % \draw[fill] (a7) circle (\rs);
        % \draw (a7)+(0,\d) node {$a$};
        % \draw[fill] (b7) circle (\rs);
        % \draw (b7)+(0,-\d) node {$a$};
    \draw[thick] (c8) circle (\r);
        % \draw[fill] (a9) circle (\rs);
        % \draw (a9)+(0,\d) node {$z_n$};
        % \draw[fill] (b9) circle (\rs);  
        % \draw (b9)+(0,-\d) node {$w_n$}; 
    \draw[thick] (z1) circle (\r);
        \draw (z1)+(100:\r+0.25) node {$D_n$};
    % \draw (z1)+(0,\d) node {$z_{n+1}$};
    % \draw (z1)+(0,-\d) node {$w_{n+1}$};
    
    \draw (22:2.0) node {$\gamma_{s_1}$};
    \draw (-2:2.0) node {$\gamma_{s_2}$};
    \draw (0,0) -- (25:-0.7) node[above left] {$2\varepsilon$} -- (25:-1.4);
\end{tikzpicture}
\end{center}




We choose consecutive points on each curve:
\begin{align*}
    \set{z_0 = \gamma_{s_1}(a), z_1, \cdots, z_{n+1} = \gamma_{s_1}(b)  } &\text{ on } \gamma_{s_1},\\
    \set{w_0 = \gamma_{s_2}(a), w_1, \cdots, w_{n+1} = \gamma_{s_2}(b)  } &\text{ on } \gamma_{s_2},\\
\end{align*}
such that $z_i, \, z_{i+1}, \, w_i, \, w_{i+1} \in D_i, \, i \in \set{0, \cdots , n}$.

\begin{center}
\begin{tikzpicture}
    
    \node (z0) at (0,0) {};
    \node (z1) at (10,0) {};
    \draw[fill] (z0) circle (0.08);
    \draw[fill] (z1) circle (0.08);
    
    \draw[postaction=
        {decoration={markings,
        mark=at position 0.1 with {\draw node (a1) {};},
        mark=at position 0.2 with {\draw node (a2) {};},
        mark=at position 0.3 with {\draw node (a3) {};},
        mark=at position 0.4 with {\draw node (a4) {};},
        mark=at position 0.5 with {\draw node (a5) {};},
        mark=at position 0.6 with {\draw node (a6) {};},
        mark=at position 0.7 with {\draw node (a7) {};},
        mark=at position 0.8 with {\draw node (a8) {};},
        mark=at position 0.9 with {\draw node (a9) {};},
        },
        decorate}][thick] 
        (z0) to[out=20,in=180]node[]{} (z1);
        
    \draw[postaction=
        {decoration={markings,
        mark=at position 0.1 with {\draw node (b1) {};},
        mark=at position 0.2 with {\draw node (b2) {};},
        mark=at position 0.3 with {\draw node (b3) {};},
        mark=at position 0.4 with {\draw node (b4) {};},
        mark=at position 0.5 with {\draw node (b5) {};},
        mark=at position 0.6 with {\draw node (b6) {};},
        mark=at position 0.7 with {\draw node (b7) {};},
        mark=at position 0.8 with {\draw node (b8) {};},
        mark=at position 0.9 with {\draw node (b9) {};},
        },
        decorate}][thick] 
        (z0) to[out=10,in=190]node[]{} (z1);

    \node (c1) at ($(a1)!0.5!(b1)$) {};
    \node (c2) at ($(a2)!0.5!(b2)$) {};
    \node (c3) at ($(a3)!0.5!(b3)$) {};
    \node (c4) at ($(a4)!0.5!(b4)$) {};
    \node (c5) at ($(a5)!0.5!(b5)$) {};
    \node (c6) at ($(a6)!0.5!(b6)$) {};
    \node (c7) at ($(a7)!0.5!(b7)$) {};
    \node (c8) at ($(a8)!0.5!(b8)$) {};
    \node (c9) at ($(a9)!0.5!(b9)$) {};

    \def\r{1.4};
    \def\rs{0.08};
    \def\d{0.25};
    \draw[thick] (z0) circle (\r);
        \draw (z0)+(100:\r+0.25) node {$D_0$};
        \draw (z0)+(0,\d) node {$z_0$};
        \draw (z0)+(0,-\d) node {$w_0$};
        \draw[fill] (a1) circle (\rs);
        \draw (a1)+(0,\d) node {$z_1$};
        \draw[fill] (b1) circle (\rs);
        \draw (b1)+(0,-\d) node {$w_1$};
    \draw[thick] (c2) circle (\r);
        \draw (c2)+(100:\r+0.25) node {$D_1$};
        \draw[fill] (a3) circle (\rs);
        \draw (a3)+(0,\d) node {$z_2$};
        \draw[fill] (b3) circle (\rs);
        \draw (b3)+(0,-\d) node {$w_2$};
    \draw[thick] (c4) circle (\r);
        \draw[fill] (a5) circle (\rs);
        % \draw (a5)+(0,\d) node {$a$};
        \draw[fill] (b5) circle (\rs);
        % \draw (b5)+(0,-\d) node {$a$};
    \draw[thick] (c6) circle (\r);
        \draw[fill] (a7) circle (\rs);
        % \draw (a7)+(0,\d) node {$a$};
        \draw[fill] (b7) circle (\rs);
        % \draw (b7)+(0,-\d) node {$a$};
    \draw[thick] (c8) circle (\r);
        \draw[fill] (a9) circle (\rs);
        \draw (a9)+(0,\d) node {$z_n$};
        \draw[fill] (b9) circle (\rs);  
        \draw (b9)+(0,-\d) node {$w_n$}; 
    \draw[thick] (z1) circle (\r);
        \draw (z1)+(100:\r+0.25) node {$D_n$};
    \draw (z1)+(0,\d) node {$z_{n+1}$};
    \draw (z1)+(0,-\d) node {$w_{n+1}$};
    
    \draw (22:2.0) node {$\gamma_{s_1}$};
    \draw (-2:2.0) node {$\gamma_{s_2}$};
    \draw (0,0) -- (25:-0.7) node[above left] {$2\varepsilon$} -- (25:-1.4);
\end{tikzpicture}
\end{center}

On each disk $D_i$ let $F_i$ be an antiderivative of $f$. Then on $D_i \cap D_{i+1}$ we have $F_i$ and $F_{i+1}$ as antiderivatives of $f$. These must differ from each other by a constant ie. $F_{i+1} = F_i + c_i$ on $D_i \cap D_{i+1}$ for some $c_i \in \C$. Thus:
\begin{align*}
    F_{i+1} (z_{i+1} ) - F_i ( z_{i+1}) &= F_{i+1} (w_{i+1}) - F_i ( w_{i+1}) = 0.\\
    &\Downarrow\\
     F_{i+1} (z_{i+1} ) -  F_{i+1} (w_{i+1}) &= F_i ( z_{i+1}) - F_i ( w_{i+1}).
\end{align*}
 
 
Thus:
\begin{align*}
    \int_{\gamma_{s_1}} f(z) \dif z - \int_{\gamma_{s_2}} f(z) \dif z & \overset{    \scalebox{0.35}{FTC}   }{=} \sum_{i=0}^n \left[ F_i (z_{i+1} ) - F_i (z_{i} )\right] - \sum_{i=0}^n \left[ F_i (w_{i+1} ) - F_i (w_{i} )\right]\\
    &= \sum_{i=0}^n \left[ \big( F_i (z_{i+1}) - F_i (w_{i+1} ) \big) - \big( F_i (z_{i}) - F_i (w_{i} ) \big) \right]\\
    &= \sum_{i=0}^n \left[ \big( F_{i+1} (z_{i+1}) - F_{i+1} (w_{i+1} ) \big) - \big( F_i (z_{i}) - F_i (w_{i} ) \big) \right]\\
    \text{(telescopic sum) } &= \big( F_n (z_{n+1}) - F_n ( w_{n+1} ) \big) - \big( F_0 (z_{0}) - F_0 ( w_{0} ) \big)\\
    &= 0 - 0\\
    &= 0.
\end{align*}
 
Thus we have proved that with $s_1, s_2 \in [0,1]$ and $\abs{s_1 - s_2} < \delta$:
\begin{align*}
\int_{\gamma_{s_1}} f(z) \dif z = \int_{\gamma_{s_2}} f(z) \dif z.
\end{align*}

\begin{center}
\begin{tikzpicture}[decoration={
    markings,
    mark=at position 0.6 with {\arrow{>}}}
    ]
    \node[label={[above left] $z_0$}] (a3) at (0,0) {};
    \node[label={[above right]$z_1$}] (b3) at (10,0) {};
    \draw[fill] (a3) circle (0.08);
    \draw[fill] (b3) circle (0.08);
    \draw[] (a3) to[out=50,in=150]node[above]{$\gamma_1$} (b3);
    \foreach \o/\i in {40/160,30/170,20/180,10/190,-10/200}
       \draw (a3) to[out=\o,in=\i]  (b3);
    \draw[] (a3) to[out=-20,in=-130]node[below]{$\gamma_0$} (b3);
\end{tikzpicture}
\end{center}

Let then $s_1 =0$, $s_2$ small, then repeat finitely many times (around $\frac{1}{\delta}$ times) until you get to $s_2 = 1$.

\end{proof}


\begin{definition}[Simply Connected]
$\oic$ open subset shall be called \textbf{simply connected} if $\om$ is connected and every two $\gamma_1, \, \gamma_2$ piecewise smooth curves in $\om$ with same initial and end points are homotopic in $\om$.
\end{definition}


\begin{corollary}\label{cor:simp-conn-int-to-zero}
$\foc$ holomorphic and $\om$ simply connected. Then $\forall ,\gamma$ piecewise smooth closed curve in $\om$:
\begin{align*}
    \int_{\gamma} f(z) \dif z = 0.
\end{align*}
\end{corollary}

\begin{proof}

Pick any point on the curve $\gamma(c)$, with $c \in (a,b)$. Let:
\begin{align*}
    \gamma_1 &= \gamma\mid_{[0,c]}\\
    -\gamma_2 &= \gamma\mid_{[c,b]}\\
    \Rightarrow \;\; \gamma &= \gamma_1 \cup (-\gamma_2).
\end{align*}

Where $- \gamma_2 $ is just a reverse orientation.

\begin{center}
\begin{tikzpicture}[decoration={
    markings,
    mark=at position 0.6 with {\arrow{>}}}
    ]
    \node[] (a3) at (0,0) {};
    \node[] (b3) at (10,0) {};
    \draw[fill] (a3) circle (0.08);
    \draw[fill] (b3) circle (0.08);
    \draw[postaction={decorate}][thick] (a3) to[out=50,in=150]node[above]{$\gamma_2$} (b3);
]
    \draw[postaction={decorate}][thick] (a3) to[out=-20,in=-130]node[below]{$\gamma_1$} (b3);
\end{tikzpicture}
\end{center}

Then:
\begin{align*}
    \int_\gamma f(z) \dif z = \int_{\gamma_1} f(z) \dif z - \int_{\gamma_2} f(z) \dif z = 0.
\end{align*}
\end{proof}

\begin{remark}
This is not the same as Cauchy, though similar. This is different because in this corollary, $\om$ is assumed to be simply connected, but $\gamma$ is not assumed to be $\gamma = \partial \om ', \, \om ' \ssubset \om$.
\end{remark}

