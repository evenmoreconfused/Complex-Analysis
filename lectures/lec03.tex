
%FILL IN THE RIGHT INFO.
%\lecture{**LECTURE-NUMBER**}{**DATE**}
\unchapter{Lecture 3}
\lecture{3}{September 10}
\setcounter{section}{0}
\setcounter{theorem}{0}

% **** YOUR NOTES GO HERE:

REVIEWED
\section{Complex Power Series}

Recall the concept of complex power series discussed in the previous lecture: a function $f(z) = \sum_{n=0}^\infty a_nz^n$ with $a_n \in \mathbb{C}$, $z\in \mathbb{C}$. Recall that $R_f=( \limsup\limits_{n\rightarrow \infty} \, \abs{a_n}^\frac{1}{n} )^{-1}$, and that if $z<R_f, \, f(z)$ converges absolutely, and if $z>R_f$, $f(z)$ diverges.

\begin{example}[Logarithm]
\begin{align*}
    log(1+z) & \defas z - \frac{z^2}{2} +\frac{z^3}{3} - \frac{z^4}{4} + \cdots\\
    &= \sum_{n=1}^\infty \frac{(-1)^{n+1}z^n}{n}\\
    R&=\frac{1}{\limsup\limits_{n\rightarrow \infty} \, \abs{a_n}^\frac{1}{n}} = \frac{1}{\limsup\limits_{n\rightarrow \infty} \, n^\frac{1}{n}} = 1
\end{align*}
Note that the complex logarithm is more complicated than the real logarithm; it won't be a single function, but a family of functions. This power series is only one of the family. An example of the non-simplicity of the logarithm is as follows: given $0 \neq z \in \mathbb{C}$, we want $w=log(z)$ s.t. $e^w=z$. Given w we can easily find z. Given z, can we find w?
\begin{align*}
    z&=re^{i\theta}, \, r=\abs{z}>0, \, 0 \leq \theta \leq 2\pi, \, \theta = Arg(z)\\
    w&=x+iy, \, x,y\in \mathbb{R}\\\\
    e^{x+iy} &= 
    \underbrace{e^x}_{\in \mathbb{R}} \underbrace{e^{iy}}_{\in \mathbb{C}} = re^{i\theta}\\ \implies e^x &=r \implies x=log(r)\\
   \text{and } e^{iy} &= e^{i\theta} \implies y = \theta +2\pi k, \, k\in \mathbb{Z}
\end{align*}
That is to say that given $z\neq 0$, there are infinitely many w that solve $e^w=z$:
$$w = log(z) = i\cdot (Arg(z) +2\pi k), \, k\in \mathbb{Z}$$

which differ by adding multiples of $2 \pi i$.
\end{example}

\section{Contour Integral Properties}

We now continue some of the properties of contour integrals:

\begin{enumerate}
    \item The value of the integral is not dependant on the parameterization (as long as orientation is preserved).
    \item If $\Tilde{\gamma}$ is $\gamma$ with reverse orientation then:
    
    Let $z(t), \, a\leq t \leq b$ be a parameterization of $\gamma$. Let $\Tilde{z}(t) = z(b+a-t), \, a\leq t \leq b$ be a parameterization of $\Tilde{\gamma}$. Assume $\gamma$ is smooth (piece-wise smooth yields a similar argument on the different smooth intervals).
    \begin{align*}
         \int_{\Tilde{\gamma}} f(z)  \dif z &= \int_{a}^b f( \underbrace{z(b+a-t)}_{\Tilde{z}(t)}) \cdot (\underbrace{-z'(b+a-t)}_{\Tilde{z}'(t)})  \dif t\\
        \textit{(change variable $t \to b+a -t$)}  &= - \int_a^b f(z(t))\cdot z'(t) \dif t = - \int_{\gamma} f(z)  \dif z
    \end{align*}
    
    \item $\forall \lambda, \mu \in \mathbb{C}, \, f,g$ continuous on $\mathbb{C}$:
    \begin{align*}
        \int_{\gamma} (\lambda f(z) + \mu g(z))  \dif z = \lambda \int_{\gamma} f(z)  \dif z + \mu \int_{\gamma} g(z)  \dif z
    \end{align*}
    
    \item For any $\gamma$ and $f(z)$:
    \begin{align*}
    \text{with } L(\gamma) &= arclength(\gamma) = \int_a^b \abs{z'(t)}  \dif t\\
    \abs{\int_{\gamma} f(z)  \dif z} &= \abs{\int_a^b f(z(t)) z'(t)  \dif t}\\
    &\leq \int_a^b \abs{f(z(t))} \cdot \abs{z'(t)}  \dif t\\
    &\leq \sup_{z \in \gamma[a,b]} \abs{f(z)} \cdot \underbrace{\int_a^b \abs{z'(t)}}_{L(\gamma)}  \dif t\\
    &= \sup\limits_{z \in \gamma[a,b]} \abs{f(z)} \cdot L(\gamma)
    \end{align*}
   
   
\end{enumerate}


\section{Complex Antiderivatives}

We now consider the complex equivalent of antiderivatives.

\begin{definition}[Antiderivatives]
Let $\Omega \subset \mathbb{C}$ open $f:\Omega \rightarrow \mathbb{C}$ holomorphic. $F:\Omega \rightarrow \mathbb{C}$ is an \textbf{antiderivative} of $f(z)$ on $\Omega$ if $F(z)$ is holomorphic and $F'(z) = f(z) \, \forall z\in\Omega $.
\end{definition}

\begin{example}
$F(z) = \frac{z^{n+1}}{n+1}$ is an antiderivative of $f(z) = z^n$ on any $\Omega \subset \mathbb{C}$.
\end{example}
\begin{example}
$f(z) = e^z$ is an antiderivative of itself.
\end{example}

\isubsection{THM: Fundamental Theorem of Calculus}
\begin{theorem}[Fundamental Theorem of Calculus]
Suppose that $F(z)$ is an antiderivative of $f(z)$ on some $\Omega \subset \mathbb{C}$. Let $\gamma$ be a piece-wise smooth path in $\Omega$. Then:
\begin{align*}
    \int_\gamma f(z)  \dif z &= F(\gamma(b)) - F(\gamma(a))
\end{align*}
\end{theorem}


\begin{corollary}
If $\gamma$ is a closed curve (a curve $\gamma$ with $\gamma(a)=\gamma(b)$) then:
\begin{align*}
    \int_\gamma f(z)  \dif z &= 0
\end{align*}

\end{corollary}

\begin{proof}[FTC]
Assume that $\gamma$ is smooth.
\begin{align*}
    \int_\gamma f(z)  \dif z &= \int_a^b f(z(t)) z'(t)  \dif t\\
    &= \int_a^b \underbrace{F'(z(t)) z'(t)}_{\frac{d}{  dt} F(z(t))}  \dif t\\
    &= \int_a^b \left( \frac{d}{ dt} F(z(t)) \right)  \dif t\\
    \text{(apply real FTC) }&= F(z(b))-F(z(a))\\
    &= F(\gamma(b)) - F(\gamma(a))
\end{align*}

If $\gamma$ is only piece-wise smooth:
\begin{align*}
    \int_\gamma f(z) \dif z &= \sum_{j=0}^{N-1} \int_ {a_j}^{a_{j+1}} f(z(t))z'(t) \dif t\\
   \text{(apply above) } &= \sum_{j=0}^{N-1} (F(z(a_{j+1}))-F(z(a_j)))\\
   &= F(z(b)) - F(z(a))
\end{align*}
\end{proof}

\begin{corollary}
If $f:\Omega \rightarrow \mathbb{C}$ is a holomorphic function on $\Omega$ open and $\underline{\textbf{connected}}$, then $f'(z)=0 \, \forall z \in \Omega \, \implies \, f(z)$ is constant.
\end{corollary}

\begin{proof}
Since $\Omega$ is open and connected it follows (by elementary topology) that $\Omega$ is path-connected (ie given $z_0,z_1 \in \Omega \, \exists \gamma:[a,b] \rightarrow \Omega$ st $ \gamma(a) = z_0$ and $\gamma(b) = z_1$).

Apply FTC to f'(z):
\begin{align*}
    0 = \int_\gamma \underbrace{f'(z)}_{=0}  \dif z &= f(b) - f(a)\\
    \implies f(a) &= f(b)
\end{align*}
\end{proof}
\begin{remark}
Connectedness is important here, as if we don't have it we could have a function $f(z)$ defined on two sets, and constant locally on both, but not globally connected.
\end{remark}

Next week we'll continue with the proof for $f(z)$ holomorphic $\implies$ $f(z)$ analytic
