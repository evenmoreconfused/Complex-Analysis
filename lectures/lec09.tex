%FILL IN THE RIGHT INFO.
%\lecture{**LECTURE-NUMBER**}{**DATE**}
\unchapter{Lecture 9}
\lecture{9}{October 1}
\setcounter{section}{0}
\setcounter{theorem}{0}

% **** YOUR NOTES GO HERE:


We spend more time on poles this lecture.

\section{Poles and the Residue Formula}

\begin{note}
Recall that a holomorphic function $f$ has a pole at $z_0$ if:

\begin{align*}
    \Tilde{f}(z) \vcentcolon = \begin{cases} \frac{1}{f(z)} & z \in D \setminus \{ z_0 \} \\ 0 & z = z_0  \end{cases}
\end{align*}

is holomorphic on $D$ (a disk around $z_0$). Naturally this implies that $lim_{z \to z_0} \frac{1}{f(z)} = 0$. This means that $lim_{z \to z_0} f(z) = \pm \infty$. This is to say that a pole is somehow a point where $f$ goes to $\pm \infty$.

\end{note}



We revisit an example from last lecture:





\begin{example}\label{ex:res-example}
The function
\begin{align*}
f(z) = \frac{1}{z^2+1} = \frac{1}{(z-i)(z+i)}
\end{align*}
has two poles: $z = \pm i$. Consider the pole $i$. Then:
\begin{align*}
    \frac{1}{z^2+1} = \frac{1}{(z-i)(z+i)} = \frac{\frac{1}{z+i}}{z-i} = \frac{g(z)}{z-z_0}
\end{align*}
for $z_0 = i$. $\frac{1}{z+i}$ is indeed a non-vanishing and holomorphic function around $i$. We then expand $g(z)$ in a power series at $z_0 = i$:
\begin{align*}
    g(z) = \sum _{n=0}^\infty a_n (z-i)^n &= a_0 + \cdots\\ &= \frac{1}{2i} + \cdots.
\end{align*}
Thus $a_{-1} = \frac{1}{2i}$ in the Laurent series expansion, so $Res_i(f) = \frac{1}{2i}$. Similarly, $Res_{-i}(f) = \frac{-1}{2i}$.
\end{example}

\begin{lemma}\label{lem:pole-res}
Say $f$ has a pole at $z_0 \in \om$ of order $N \geq 1$. Then:
\begin{align*}
    Res_{z_0}(f) = \lim_{z \to z_0} \left[ \frac{1}{(N-1)!} \left( \frac{\partial}{\partial z} \right)^{N-1} \left[ (z-z_0)^N \cdot f(z) \right] \right].
\end{align*}
\end{lemma}

\begin{note}
Multiplying by $(z-z_0)^N$ makes your function holomorphic. We then take the derivative and divide by $(N-1)!$ to readjust it after taking the derivative.
\end{note}

\begin{proof}[\ref{lem:pole-res}]
Expand $f$ by lemma (\ref{lem:laurent-series}). Then:
\begin{align*}
    f(z) &= \frac{a_{-N}}{(z-z_0)^N} + \cdots + \frac{a_{-1}}{z-z_0} + G(z)\\
    (z-z_0)^N f(z) &= a_{-N} + \cdots + (z-z_0)^{N-1} a_{-1} + (z-z_0)^N G(z).
\end{align*}

Then note that $(z-z_0)^N G(z)$ vanishes to order at least $N$ at the point $z_0$. Since taking the derivative reduces the order of vanishing by $1$, the $N-1$'th derivative of $(z-z_0)^N G(z)$ vanishes to order at least $1$ at the point $z_0$. Thus:
\begin{align*}
    \lim_{z \to z_0} \left[ \left( \frac{\partial}{\partial z} \right)^{N-1} \left[ (z-z_0)^N f(z)  \right] \right] &=  a_{-1} (N-1)! + \lim_{z \to z_0} \left[ \left( \frac{\partial}{\partial z} \right)^{N-1} \left[ (z-z_0)^N G(z) \right] \right]\\
    &= a_{-1}  (N-1)! + 0\\
    &= a_{-1}  (N-1)!.
\end{align*}
\end{proof}

\begin{example}
Consider:
\begin{align*}
    f(z) = \frac{e^z}{z^5} = \frac{g(z)}{z^5}.
\end{align*}

Then the denominator vanishes at $0$, and the numerator never vanishes. Thus $f$ has a pole of order $5$ at $z_0 = 0$. We approach this in two ways:
\begin{enumerate}
    \item We directly compute the residue. Expand $e^z = \sum \frac{z^n}{n!}$ as a power series around $z_0 = 0$. Then:
    \begin{align*}
        f(z) = \frac{1}{z^5} + \frac{1}{z^4} +\frac{1}{2z^3} + \frac{1}{6z^2} + \frac{1}{24z} + \cdots.
    \end{align*}
    We can directly observe that the residue is $\frac{1}{24}$.
    
    \item Use lemma (\ref{lem:pole-res}). Then $N=5$:
    
    \begin{align*}
        Res_{0}(f) &= \lim_{z \to 0} \frac{1}{4!} \left( \frac{\partial }{\partial z} \right)^4 e^z \\
        &= \lim_{z \to 0} \frac{1}{4!} e^z\\
        &= \frac{1}{4!}.
    \end{align*}
\end{enumerate}
\end{example}

\begin{lemma}\label{lem:res-lemma}
Suppose $f$ has a pole at $z_0 \in \om$ and expand $f(z) = \sum_{n=-N}^\infty a_n (z-z_0)^n$ in Laurent series around $z_0$ for some $D = D_r(z_0) \ssubset \om$. Then $\forall n$:
\begin{align*}
    a_n = \frac{1}{2 \pi i} \int_D \frac{f(z)}{(z-z_0)^{n+1}} \dif z.
\end{align*}
\end{lemma}

\begin{proof} On $\DP$:
\begin{align*}
    \frac{f(z)}{(z-z_0)^{n+1}} &= \sum_{j=-N}^\infty a_j (z-z_0)^{j-n-1}\\
    \int_D \frac{f(z)}{(z-z_0)^{n+1}} \dif z &= \sum_{j=-N}^\infty a_j \underbrace{\int_D (z-z_0)^{j-n-1} \dif z}_{2\pi i \text{ if } j-n-1=-1;\text{ 0 else}}\\
    &= a_n \cdot 2 \pi i.
\end{align*}

\end{proof}

We now present a theorem that is theoretically not very powerful, but is very useful for calculations:

\isubsection{THM: Residue Formula}

\begin{theorem}[Residue Formula]\label{thm:residue-formula}

Suppose $f$ holomorphic on $\om \setminus \set{ z_1,\cdots, z_N}, \, z_j \in \om $. Suppose that $f$ has poles at $z_1,\cdots, z_N$. Let $\om ' \ssubset \om$ open subset with piecewise smooth boundary with $z_1,\cdots, z_N \in \om '$. Then:
\begin{align*}
    \frac{1}{2\pi i} \int_{\partial \om '} f(z) \dif z = \sum_{i=1}^N Res_{z_i} (f).
\end{align*}

\end{theorem}

\begin{note}
Note that you don't have to have $\om ' \ssubset \om$, just that it's holomorphic on the closure of $\om$. Note that this also contains theorem (\ref{thm:cauchy-theorem}) as a trivial case (when $N$ = 0). In general there is no reason for the sum of the residues to be zero.
\end{note}





\begin{center}
\begin{tikzpicture}[
    tangent/.style={
        decoration={
            markings,% switch on markings
            mark=
                at position #1
                with
                {
                    \coordinate (tangent point-\pgfkeysvalueof{/pgf/decoration/mark info/sequence number}) at (0pt,0pt);
                    \coordinate (tangent unit vector-\pgfkeysvalueof{/pgf/decoration/mark info/sequence number}) at (1,0pt);
                    \coordinate (tangent orthogonal unit vector-\pgfkeysvalueof{/pgf/decoration/mark info/sequence number}) at (0pt,1);
                }
        },
        postaction=decorate
    },
    use tangent/.style={
        shift=(tangent point-#1),
        x=(tangent unit vector-#1),
        y=(tangent orthogonal unit vector-#1)
    },
    use tangent/.default=1
]
    \draw[dotted]
        plot [smooth cycle] coordinates {(-0.6*10,0) (-0.6*7,0.6*3.5) (0,0.6*5)  (0.6*4,0.6*2.5) (0.6*7,0)  (0.6*5,0.6*-7) (0.6*2,-0.6*6)  (-0.6*5,-0.6*5) };
    \draw[scale =0.5][tangent=0][tangent=0.1][tangent=0.15][tangent=0.3][tangent=0.4][tangent=0.5][tangent=0.6][tangent=0.7][tangent=0.8][tangent=0.9][tangent=1][tangent=1.1]
        plot [smooth cycle] coordinates {(-10,0) (-7,3.5) (0,5)  (4,2.5) (7,0)  (5,-7) (2,-6)  (-5,-5) } (0,0) circle (1) (170:5) circle (1) (-80:4) circle (1);
        
\draw [thick, use tangent=5][->] (0,0) -- (-0.01,0);
\draw [thick, use tangent=3][->] (0,0) -- (-0.01,0);
\draw [thick, use tangent=7][->] (0,0) -- (-0.01,0);

\draw [thick, use tangent=9][->] (0,0) -- (-0.01,0);
\draw [thick, use tangent=10][->] (0,0) -- (-0.01,0);
\draw [thick, use tangent=11][->] (0,0) -- (-0.01,0);


\draw[thick] (-0.1,0.1) -- (0.1,-0.1) (0.1,0.1) -- (-0.1,-0.1);
\draw (0,0) circle [radius=0.5];
\draw (0,0)[below] node {$z_1$};

\draw[shift=(170:2.5)][thick] (-0.1,0.1) -- (0.1,-0.1) (0.1,0.1) -- (-0.1,-0.1);
\draw (170:2.5) circle [radius=0.5];
\draw (170:2.5)[below] node {$z_2$};

\draw[shift=(-80:2)][thick] (-0.1,0.1) -- (0.1,-0.1) (0.1,0.1) -- (-0.1,-0.1);
\draw (-80:2) circle [radius=0.5];
\draw (-80:2)[below] node {$z_3$};

\draw node at (-3.75,-2) {$\Omega'$};
\draw node at (-4.3,-2.55) {$\Omega$};


\end{tikzpicture}

    
\end{center}





\begin{proof}
Apply theorem (\ref{thm:cauchy-theorem}) to some small disks around each pole. By assumption, $f$ is holomorphic on $\om ' \setminus \bigcup_{i=1}^N D_r(z_i)$ for some small $r>0$. Theorem (\ref{thm:cauchy-theorem}) applies on this new domain (parameterizing $\partial D_r(z_i)$ counterclockwise to achieve a negative):

\begin{align*}
    0 &= \int_{\partial \om '} f(z) \dif z - \sum_{j=1}^N \int_{\partial D_r(z_j)} f(z) \dif z\\
    \text{(apply lemma (\ref{lem:res-lemma})) } &= \int_{\partial \om '} f(z) \dif z - \sum_{j=1}^N 2\pi i \cdot Res_{z_j} (f).
\end{align*}

\end{proof}


\section{Applications of Residue Formula}

\begin{example}
$D = D_2(z)$. Compute $\int_{\partial D} \frac{z+1}{z^2 + 1 } \dif z$. This is hard to compute directly, but easy to do with residues. $f$ has two simple poles at $z_0 = \pm i$:
\begin{align*}
    \int_{\partial D} \frac{z+1}{z^2 + 1 } \dif z &= 2\pi i \left( Res_{i} \left( \frac{z+1}{z^2 + 1 } \right) + Res_{-i} \left( \frac{z+1}{z^2 + 1 } \right) \right)\\ \text{(from example (\ref{ex:res-example})) } &= 2\pi i \left( \frac{1+i}{2i} + \frac{-i+1}{-2i} \right) = 2 \pi i.
\end{align*}

\end{example}

\begin{example}
Evaluate $\int_{-\infty}^\infty \frac{\dif x }{ 1+ x^2}$. We use the contour of the  semicircle on the upper half plane of radius $R$.

\begin{center}
\begin{tikzpicture}[very thick,decoration={
    markings,
    mark=at position 0.6 with {\arrow{>}}}
    ]
    \clip (-4,-1.1) rectangle (4,4);
    %\draw (-4,-1) rectangle (4,4);
    \draw[postaction={decorate}][rotate = 200] [blue,line width=1.5pt] (0,0) circle [radius=3];
    
    \draw[color=white][fill=white] (-3.5,0) rectangle (3.5,-2);
    
    
    \draw [thin] [->] (-4,0)--(4,0);
    \draw [thin] [->] (0,-1)--(0,4);
    
    \draw[postaction={decorate}] [blue,line width=1.5pt] (-3-0.026,0) -- (3+0.026,0);
    
    %\draw[postaction={decorate}] [blue,line width=1.5pt] (-3,0) -- (3,0);
    %\draw[postaction={decorate}] [blue,line width=1.5pt] (3,2) -- (-3,2);
   % \draw[postaction={decorate}] [blue,line width=1.5pt] (3,0) -- (3,2);
    %\draw[postaction={decorate}] [blue,line width=1.5pt] (-3,2) -- (-3,0);
    
    
    \draw (-3,0)[below] node {$-R$};
    \draw (3,0)[below] node {$R$};
    
    \draw[thick] (-0.1,0.1+0.7) -- (0.1,-0.1+0.7);
    \draw[thick] (0.1,0.1+0.7) -- (-0.1,-0.1+0.7);
    
    \draw[thick] (-0.1,0.1-0.7) -- (0.1,-0.1-0.7);
    \draw[thick] (0.1,0.1-0.7) -- (-0.1,-0.1-0.7);
    
    
    \draw (-2,0)[below] node {$\gamma_1$};
    \draw (120:3)[above] node {$\gamma_2$};
    \draw (180-120:3)[above] node {$\gamma_R$};
    \draw (0,0.7) [above left] node {$i$};
    \draw (0,-0.7) [above left] node {$-i$};
    
    
    
    
\end{tikzpicture}
\end{center}

Now note that:
\begin{align*}
    \abs{f(z)} &= \frac{1}{\abs{a+z^2}} \leq \frac{2}{\abs{z}^2} = \frac{2}{R^2},\\
   \abs{ \int_{\gamma_2} \frac{\dif z }{ 1+ z^2} } &\leq L(\gamma_2) \cdot \frac{2}{R^2} = \pi R \frac{2}{R^2} \xrightarrow[]{R\to \infty} 0.
\end{align*}

Thus:
\begin{align*}
    \int_{\partial R} \frac{\dif z}{1+z^2} = \int_{\gamma_1} \frac{\dif z}{1+z^2} &+ \cancelto{0}{\int_{\gamma_2} \frac{\dif z}{1+z^2}} = 2\pi i \cdot \res{i}{\frac{1}{1+z^2}} = \pi.\\
    \implies \int_{\gamma_1} \frac{\dif z}{1+z^2} &= \int_{\R} \frac{\dif z}{1+z^2} = \pi.
\end{align*}

\end{example}



\begin{example}\label{ex:box-example}
Consider $\int_{-\infty}^\infty \frac{e^{ax}}{1+e^x} \dif x$. Note that $\left(\frac{e^{ax}}{1+e^x} \right) \xrightarrow[]{x \to \pm \infty} 0$. Thus $0 < a<1$ (otherwise the integral will diverge at $\pm \infty$).

We propose to use $f(z) = \frac{e^{ax}}{1+e^x}$. This has poles when $e^z = -1$, thus when $z = \pi i + 2 \pi i k, \, k \in \mathbb{Z}$.

We propose this contour:


\begin{center}
\begin{tikzpicture}[very thick,decoration={
    markings,
    mark=at position 0.6 with {\arrow{>}}}
    ]
    
    \draw [thin] [->] (-4,0)--(4,0);
    \draw [thin] [->] (0,-1)--(0,3);
    
    \draw[postaction={decorate}] [blue,line width=1.5pt] (-3,0) -- (3,0);
    \draw[postaction={decorate}] [blue,line width=1.5pt] (3,2) -- (-3,2);
    \draw[postaction={decorate}] [blue,line width=1.5pt] (3,0) -- (3,2);
    \draw[postaction={decorate}] [blue,line width=1.5pt] (-3,2) -- (-3,0);
    
    
    \draw (-3,0)[below] node {$-R$};
    \draw (3,0)[below] node {$R$};
    \draw (3,2)[above] node {$R+2 \pi i$};
    \draw (-3,2)[above] node {$-R+2 \pi i$};
    
    \draw (1,0)[below] node {$\gamma_1$};
    \draw (3,1)[right] node {$\gamma_2$};
    \draw (-1,2)[above] node {$\gamma_3$};
    \draw (-3,1)[left] node {$\gamma_4$};
    
    
    \draw[thick] (-0.1,0.1+1) -- (0.1,-0.1+1);
    \draw[thick] (0.1,0.1+1) -- (-0.1,-0.1+1);
    \draw (0,1) [above right] node {$\pi i$};
    
\end{tikzpicture}
\end{center}

By the residue formula:
\begin{align*}
    \int_{\gamma_1} + \int_{\gamma_2} + \int_{\gamma_3} + \int_{\gamma_4} = 2 \pi i \cdot \res{\pi i}{f}.
\end{align*}

Since these are simple poles:
\begin{align*}
    \res{\pi i}{f} &= \lim_{z \to \pi i} (z - \pi i) f(z)\\
    &= \lim_{z \to \pi i} \frac{e^{az} (z-\pi i)}{1+e^z}.
\end{align*}

We expand the denominator using $h(z) = e^z$ at $z_0 = \pi i$ :
\begin{align*}
    h(z) &= h(z_0) + h'(z_0)(z-z_0) + o( \, \abs{z-z_0})\\
    &= e^{\pi i} + e^{\pi i} (z- \pi i) +  o( \, \abs{z-\pi i})\\
    &= -1 -(z-\pi i) + o( \, \abs{z-\pi i}).\\
    e^z + 1 &= -(z-\pi i) + o( \, \abs{z-\pi i}).
\end{align*}

Thus:
\begin{align*}
    \res{\pi i}{f} &= \lim_{z \to \pi i} \frac{e^{az} (z-\pi i)}{1+e^z}\\
    &= \lim_{z \to \pi i} \left( -e^{az} + o(1) \right) = - e^{a\pi i}.
\end{align*}

Evaluating the $\gamma_1$ term:
\begin{align*}
    \int_{\gamma_1} f(z) \dif z &= \int_{-R}^R \frac{e^{ax}}{1+e^x} \dif x \xrightarrow[]{R \to \infty } \int_{-\infty}^\infty \frac{e^{ax}}{1+e^x} \dif x.
\end{align*}

Evaluating the $\gamma_3$ term, remaining conscious of the fact that $\gamma_3$ is $\gamma_1$ evaluated backwards but shifted up $2 \pi i$:
\begin{align*}
    \int_{\gamma_3} f(z) \dif z &= \int_{\gamma_3} \frac{e^{az}}{1+e^z} \dif z\\
    &= \int_{\gamma_1} \frac{e^{a(z+2\pi i)}}{1+e^{(z+2\pi i)}} \dif z\\
    &= - e^{2\pi i a} \int_{\gamma_1}  f(z) \dif z.
\end{align*}
 Where the minus sign is because we parameterize $\gamma_3$ in the opposite direction compared to $\gamma_1$, and the factor of $e^{2\pi i a}$ comes from the fact that the parameterization of $\gamma_3$ involves adding $2\pi i$ to the parameterization of $\gamma_1$.

Evaluating the $\gamma_2$ term with parameterization $z(t) = R + it, \, t \in [0,2\pi ]$:
\begin{align*}
    \abs{\int_{\gamma_2} f(z) \dif z }&=\abs{ \int_0^{2\pi} \frac{e^{a(R+it)}}{1+e^{R+it}} i \dif t }\\
    &\leq  \int_0^{2\pi} \abs{ \frac{e^{a(R+it)}}{1+e^{R+it}} i} \dif t \\
    &= \int_0^{2\pi} \frac{e^{aR}}{\abs{ 1+e^{R+it} }} \dif t\\
    &\leq \int_0^{2\pi}  \frac{e^{aR}}{e^R -1 }       \dif t\\
    \text{(for some constant C) }&\leq \int_0^{2\pi}  C e^{(a-1)R}       \dif t \xrightarrow[]{R \to \infty} 0.
\end{align*}

Similarly for $\gamma_4$ with parameterization $z(t) = -R - it, \, t \in [0,2\pi ]$:
\begin{align*}
    \abs{\int_{\gamma_4} f(z) \dif z } &=\abs{ \int_0^{2\pi} \frac{e^{-a(R+it)}}{1+e^{-(R+it)}} i \dif t }\\
    &\leq \int_0^{2\pi} \abs{\frac{e^{-a(R+it)}}{1+e^{-(R+it)}}}  \dif t\\
    \text{(for some constant C) } &\leq \int_0^{2\pi} Ce^{-aR}  \dif t \xrightarrow[]{R \to \infty} 0.
\end{align*}

Finally this implies that, since $\sin(z)  = \frac{e^{iz} - e^{-iz}}{2 i}$:
\begin{align*}
    - 2 \pi i e^{a\pi i} &= (1-e^{2\pi i a}) \int_\R \frac{e^{ax}}{1+e^x} \dif x.\\
    \implies \int_\R \frac{e^{ax}}{1+e^x} \dif x &= \frac{2\pi i}{e^{\pi ia} - e^{-\pi i a}} = \frac{\pi}{\sin(\pi a)}.
\end{align*}

And thus we are done.

\end{example}