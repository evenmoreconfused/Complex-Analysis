%FILL IN THE RIGHT INFO.
%\lecture{**LECTURE-NUMBER**}{**DATE**}
\unchapter{Lecture 24}
\lecture{24}{November 24}
\setcounter{section}{0}
\setcounter{theorem}{0}

% **** YOUR NOTES GO HERE:

\section{Functional Equation for Riemann Zeta Function}


\begin{theorem}[Functional Equation]
$\forall s \in \C$, we have that:
\begin{align*}
    \zeta(s) = 2^s \cdot \pi^{s-1} \cdot \sin\br{\frac{\pi s}{2}} \cdot \Gamma(1-s) \cdot \zeta(1-s)
\end{align*}

\end{theorem}

\begin{note}
This relation compares the value of $\zeta(s) $ with $\zeta (1-s)$. These are not equal, but they are instead related through a "nice" multiplication factor.
\end{note}

\begin{proof}
To be discussed later.
\end{proof}

Before the proof, we derive several consequences of this theorem. We begin with a proposition.

% \begin{proposition}[Duplication Formula]
% For all $s \in \C$, we have that:
% \begin{align*}
%     \Gamma\br{\frac{s}{2}} \cdot \Gamma\br{\frac{s+1}{2}} = \pi ^{\frac{1}{2}} \cdot 2^{1-s} \cdot \Gamma(s)
% \end{align*}
% \end{proposition}

% \begin{proof}
% asdf
% \end{proof}

\begin{proposition}[Lindel{\"o}f]
For all $z \not\in \Z$, we have that:
\begin{align*}
    \pi \cot(\pi z) = \frac{1}{z} + \sum_{n=1}^\infty \frac{2 z }{z^2-n^2}
\end{align*}
\end{proposition}
\begin{proof}
Define $f(z) \defas \pi \cot(\pi z)$. First we notice that $\frac{1}{z+n} + \frac{1}{z-n} = \frac{2z}{z^2 - n^2}$. Thus:
\begin{align*}
    \frac{1}{z} + \sum_{n=1}^\infty \frac{2 z }{z^2-n^2} &= \frac{1}{z} + \sum_{n=1}^\infty \br{\frac{1}{z+n} + \frac{1}{z-n}}
\end{align*}
While it is tempting to write $\sum_{n=-\infty}^\infty \frac{1}{z+n}$, we cannot manipulate the sum casually since it is very nearly divergent (since $\sum_{n=1}^\infty \frac{1}{z+n}$ is divergent). We can however apply a limit. Then we have that:
\begin{align*}
    \frac{1}{z} + \sum_{n=1}^\infty \br{\frac{1}{z+n} + \frac{1}{z-n}} = \lim_{N \to \infty}\sum_{\abs{n} \leq N} \frac{1}{z+n} = \vcentcolon g(z)
\end{align*}
The we want to compare $f(z)$ and $g(z)$. Both functions are meromorphic in $\C$. Both have simple poles precisely at $z \in \Z$. Note also that $\res{0}{f} = \res{0}{g} = 1$. Furthermore they are "periodic" with period $1$, namely that $\forall z \not\in \Z$, $f(z+1) = f(z)$ and $g(z+1) = g(z)$. This is clear for $f$ by the definition of $\cos$ and $\sin$. This follows for $g$ as:
\begin{align*}
    g(z+1) = \lim_{N \to \infty}\sum_{\abs{n} \leq N} \frac{1}{z+1+n} &= \lim_{N \to \infty} \br{ \frac{1}{z+1+N} - \frac{1}{z-N} + \sum_{\abs{n} \leq N} \frac{1}{z+n}}\\
    &= 0 - 0 + \lim_{N \to \infty}\sum_{\abs{n} \leq N} \frac{1}{z+n}\\
    &= g(z)
\end{align*}

It follows that $f(z) - g(z)$ is also periodic with period $1$. Since $f$ and $g$ both have a simple pole at $z=0$ with residue $1$, then $f-g$ has a removable singularity at $z=0$. By periodicity it follows that all singularities of $f-g$ are removable. Thus $f-g$ is entire holomorphic and $1$-periodic (periodic with period $1$).\\



We now show that $f-g$ is bounded in $\C$. Since $f-g$ is $1$-periodic, $f-g$ is uniquely determined by its restriction to the strip $S \defas \set{z \in \C \mid \, \abs{\Re(z)} \leq \frac{1}{2}}$ ie $\forall z \in \C \, \, \exists z_0 \in S$ s.t. $(f-g)(z) = (f-g)(z_0)$ with $z = z_0 + k$ for some $k \in \Z$. Thus it suffices to show that $\sup_{S} \abs{f-g} \leq C $ for some $C$. It further suffices to show that $\sup_{\Tilde{S}} \abs{f-g} \leq C $ for some $C$ where $\Tilde{S} \defas \set{z \in \C \mid \, \abs{\Re(z)} \leq \frac{1}{2}, \, \abs{\Im (z)} >1 }$ (since $\Tilde{S} \setminus S$ is relatively compact, and thus $f-g$ is bounded on it). We will show that $\sup_{\Tilde{S}} \abs{f} \leq C $ and $\sup_{\Tilde{S}} \abs{g} \leq C $ for some $C$.\\

\begin{enumerate}
    \item[$f$:] It suffices to show that $\sup_{\Tilde{S}} \abs{\cot(\pi z)} \leq C $. Let $z = x+iy \in \Tilde{S}$. Then $\abs{x} \leq \frac{1}{2}$ and $\abs{y} > 1$. Note that:
    \begin{align*}
        \abs{e^{-2 \pi y} + e^{-2 \pi i x}} &\leq \abs{e^{-2 \pi y} } + \abs{ e^{-2 \pi i x}} = 1+e^{-2 \pi y} \\
        \text{and that } \;\; \abs{e^{-2 \pi y} - e^{-2 \pi i x}} &\geq \abs{e^{-2\pi y} -1}
    \end{align*}
    We have that:
    \begin{align*}
        \abs{\cot(\pi z)} &= \abs{i \cdot  \frac{e^{i \pi z} + e^{-i \pi z}}{e^{i \pi z} - e^{-i \pi z}}}\\
        &= \abs{\frac{e^{i \pi x} e^{- \pi y} + e^{-i \pi x} e^{\pi y}}{e^{i \pi x}e^{-\pi y} - e^{-i \pi x} e^{\pi y}}}\\
        &= \abs{ \frac{ e^{-2 \pi y} + e^{-2 \pi i x} }{ e^{-2 \pi y} - e^{ - 2 \pi i x}}}\\
        &\leq \frac{1+e^{-2 \pi y}}{\abs{e^{-2\pi y} -1}}
    \end{align*}
    If $y>1$ we have that $e^{-2 \pi y} < \frac{1}{2}$. Thus $e^{-2\pi y} -1 < 0$ and thus:
    \begin{align*}
        \abs{\cot(\pi z)} \leq \frac{1+e^{-2 \pi y}}{1-e^{-2\pi y}} \leq \frac{\br{1+\frac{1}{2} }}{\br{\frac{1}{2}}} = 3
    \end{align*}
    If $y<-1$ we have that $e^{-2 \pi y} > 1$. Thus $e^{-2\pi y} -1 > 0$ and thus:
    \begin{align*}
        \abs{\cot(\pi z)} \leq \frac{1+e^{-2 \pi y}}{e^{-2\pi y} - 1} 
    \end{align*}
    These two exponential terms may blow up if $y$ is very small. Thus we need to fiddle around a little more. Note that in fact if $y<-1$ then $e^{-2 \pi y} > 2$. Then $\frac{1}{2} e^{-2 \pi y} > 1$. Then:
    \begin{align*}
        e^{-2 \pi y} -1 &> e^{-2 \pi y} - \frac{1}{2} e^{-2 \pi y}\\
        &= \frac{1}{2} e^{-2 \pi y}
    \end{align*}
    Thus we have, for $y<-1$, that:
    \begin{align*}
        \abs{\cot(\pi z)} \leq \frac{1+e^{-2 \pi y}}{e^{-2\pi y} - 1} \leq \frac{\frac{1}{2} e^{-2\pi y} +e^{-2 \pi y}}{e^{-2 \pi y} - \frac{1}{2}e^{-2 \pi y}} = 3
    \end{align*}
    Thus $f$ is always bounded.
    \item[$g$:] We must show that $\sup_{\Tilde{S}} \abs{g} \leq C $ for some $C$. Note that $z \in \Tilde{S}$ means that $z$ is not an integer. Let $z = x+iy \in \Tilde{S}$. Then $\abs{x} \leq \frac{1}{2}$ and $\abs{y} > 1$. Note that:
    \begin{align*}
        \abs{z} = \sqrt{x^2 + y^2} \leq \abs{x} + \abs{y} \leq \frac{1}{2}\abs{y} + \abs{y} = \frac{3}{2} \abs{y}
    \end{align*}
    and
    \begin{align*}
        \abs{z} = \abs{x+iy} \geq \abs{y} - \abs{x} \geq \frac{1}{2}
    \end{align*}
    and, noting that $z^2-n^2 = (x+iy)^2 -n^2 = (x^2 - y^2 -n^2) + i(2 xy)$
    \begin{align*}
        \abs{z^2 - n^2} &= \sqrt{(x^2 - y^2 -n^2)^2 + 4 x^2y^2}\\ &\geq \sqrt{(x^2 - y^2 -n^2)^2}\\
        &= y^2 + n^2 - x^2\\
        \text{($x^2\leq \tfrac{1}{2}y^2$) } &\geq \frac{1}{2} y^2 + n^2\\
        &\geq \frac{1}{2} \br{y^2 + n^2}
    \end{align*}
    Then:
    \begin{align*}
        \abs{g(z)} = \abs{\lim_{N \to \infty } \sum_{\abs{n } \leq N} \frac{1}{z+n}} &= \abs{\frac{1}{z} + \sum_{n=1}^\infty \frac{2z}{z^2+n^2}}\\
        &\leq \abs{\frac{1}{z}} + \sum_{n=1}^\infty \abs{\frac{2z}{z^2+n^2}}\\
        &\leq 2 + 2 \sum_{n=1}^\infty \frac{\abs{z}}{\abs{z^2 + n^2}}\\
        &\leq 2 + 2 \sum_{n=1}^\infty \frac{\frac{3}{2}\abs{y}}{\frac{1}{2} \br{y^2 + n^2}}\\
        &= 2 + 6 \sum_{n=1}^\infty \frac{\abs{y}}{ \br{y^2 + n^2}}
    \end{align*}
Now consider the case that $y>1$. We then have that:
\begin{align*}
    \sum_{n=1}^\infty \frac{\abs{y}}{ \br{y^2 + n^2}} &= \sum_{n=1}^\infty \frac{y}{ \br{y^2 + n^2}}\\
    \text{(summand increasing in n) }&\leq \int_{0}^\infty \frac{y}{y^2+x^2} \dif x\\
    &= \arctan\br{\frac{x}{y}}\bigg|_{x=0}^{x= \infty}\\ &= \frac{\pi}{2}
\end{align*}
Thus:
\begin{align*}
    \abs{g(z)} \leq 2 + 6 \sum_{n=1}^\infty \frac{\abs{y}}{ \br{y^2 + n^2}} \leq 2 + 3 \pi
\end{align*}

Now consider the case that $y<-1$. We leave this as an exercise to the reader.
\end{enumerate}

Thus $f-g$ is bounded. By corollary (7.5) then $f-g$ is constant. Since $f-g$ is odd, it follows that $f-g \equiv 0$. Thus, $f=g$ and we are done.



\end{proof}


% HHHHHHHHHHHHHHHHHHHHHHHHHHHHHHHHHHHHHHHHHHHHHHHHHHHHHHHHHHHHHHHHHHHHHHHHHHH


% \begin{corollary}
% The only zeroes of $\zeta$ outside of the "critical strip" $\set{ 0 \leq \Re(s) \leq 1}$ are simple zeroes at $s \in \set{-2, -4, -6, \cdots}$.
% \end{corollary}

% \begin{proof}
% By corollary (23.3) we have that $\zeta(s) \neq 0 $ for $\Re(s) > 1$. We now  need to check only on the other side of this strip.\\

% Assume that $\Re(s) < 0$. Then note that the operation in theorem (24.1) exchanges the left and the right of this strip. That is to say that $\Re(1-s) > 0$.\\

% Note that by the duplication formula (applied to $2-s$) we have that:
% \begin{align*}
%     \Gamma\br{\frac{2-s}{2}} \cdot \Gamma\br{\frac{s+1}{2}} = \pi ^{\frac{1}{2}} \cdot 2^{1-s} \cdot \Gamma(s)
% \end{align*}

% We thus apply the functional equation to get:
% \begin{align*}
%     \zeta(s) &= 2^s \cdot \pi^{s-1} \cdot \sin\br{\frac{\pi s}{2}} \cdot \Gamma(1-s) \cdot \zeta(1-s)\\
%     \text{(prop (24.3)) }&=
% \end{align*}
% \end{proof}


% HHHHHHHHHHHHHHHHHHHHHHHHHHHHHHHHHHHHHHHHHHHHHHHHHHHHHHHHHHHHHHHHHHHHHHHHHHH


\begin{corollary}
$\zeta(0) = - \frac{1}{2}$
\end{corollary}

\begin{proof}
First we use the functional equation:
\begin{align*}
    \zeta(s) &= 2^s \cdot \pi^{s-1} \cdot \sin\br{\frac{\pi s}{2}} \cdot \Gamma(1-s) \cdot \zeta(1-s)\\
    &\Downarrow \text{ (mult by $(1-s)$)}\\
    (1-s)\cdot \zeta(s) &= 2^s \cdot \pi^{s-1} \cdot \sin\br{\frac{\pi s}{2}} \cdot \underbrace{(1-s)\cdot \Gamma(1-s)}_{\text{done already}} \cdot \zeta(1-s)\\
    \text{(proposition (21.1)) } &= 2^s \cdot \pi^{s-1} \cdot \sin\br{\frac{\pi s}{2}} \cdot \Gamma(2-s) \cdot \zeta(1-s)
\end{align*}
Recall that $\zeta$ has a simple pole at $s=1$ with residue $1$. Thus $\lim_{s\to 1} (s-1)\cdot \zeta(s) = 1$. Then:
\begin{align*}
    -1 = \lim_{s\to 1} (1-2)\cdot \zeta(s) &= \lim_{s\to 1} 2^s \cdot \pi^{s-1} \cdot \sin\br{\frac{\pi s}{2}} \cdot \Gamma(2-s) \cdot \zeta(1-s)\\
    & = 2^1 \cdot \pi^0 \cdot \sin\br{\frac{\pi}{2}} \cdot \Gamma(1) \cdot \zeta(0)\\
    &= 2 \cdot  \zeta(0)\\
    &\Downarrow\\
    \zeta(0) &= - \frac{1}{2}
\end{align*}
\end{proof}

\begin{corollary}
$\zeta(2n) =(-1)^{n+1} \frac{(2\pi)^{2n}}{2 \cdot (2n)!} B_{2n}, \; n \geq 1$, with $B_n$ the Bernoulli numbers.
\end{corollary}

\begin{note}
Finding $\zeta(2)$ was called the "Basel problem".
\end{note}

\begin{proof}
We shall use the complex cotangent, defined as:
\begin{align*}
    \cot(z) \defas \frac{\cos(s)}{ \sin(z)} \; \; z \in \C
\end{align*}
Both $\cos$ and $\sin$ are entire holomorphic, and $\sin$ has simple zeroes at $z \in \set{\pi k \mid k \in \Z}$. Thus $\cot (z)$ is a meromorphic function with simple poles at $z \in \set{\pi k \mid k \in \Z}$. Considering $\cot(\pi z)$ we have a meromorphic function with simple poles at $z \in \Z$.

Now we take $0<\abs{z} < 1$. Note that then $\big|\frac{z^2}{n^2} \big| < 1$. Then by proposition (24.4) we have:
\begin{align*}
    \pi \cot(\pi z) &= \frac{1}{z} + \sum_{n=1}^\infty \frac{2 z }{z^2-n^2}\\
    &= \frac{1}{z} - 2 \sum_{n=1}^\infty \frac{z }{n^2} \cdot \underbrace{\frac{1}{1-\frac{z^2}{n^2}}}_{\text{\stackanchor{sum of}{geo. series}}}\\
    &= \frac{1}{z} - 2 \sum_{n=1}^\infty \frac{z }{n^2}  \sum_{k=1}^\infty \br{\frac{z^2}{n^2}}^{k-1}\\
    &= \frac{1}{z} - 2 \sum_{n=1}^\infty \sum_{k=1}^\infty \frac{z^{2k-1}}{n^{2k}}\\
    \text{(abs. conv.) }&= \frac{1}{z} - 2 \sum_{k=1}^\infty \sum_{n=1}^\infty \frac{z^{2k-1}}{n^{2k}}\\
    &= \frac{1}{z} - 2 \sum_{k=1}^\infty z^{2k-1} \cdot \zeta(2k)
\end{align*}

Further note that for $\abs{z} < 1$ (this time $0$ is included) we have:
\begin{align*}
    \pi z \cot(\pi z) = 1 - 2 \sum_{k=1}^\infty \zeta(2k) \cdot z^{2k}
\end{align*}
The RHS is a power series in $z$ with the coefficients being the values that we want to find. We thus expand the LHS as a power series and equate the terms to find $\zeta(2k)$.
\begin{align*}
    \pi z \cot(\pi z) = \pi z \frac{\sin(z)}{\cos(z)} &= i \pi z \frac{e^{i \pi z} + e^{-i \pi z}}{e^{i \pi z} - e^{-i \pi z}}\\
    & = i \pi z \frac{e^{2 i \pi z} + 1}{e^{2 i \pi z} - 1}\\
    &= i \pi z \br{ 1 + \frac{2}{e^{2 i \pi z} - 1} }\\
    &= i \pi z + \frac{2i \pi z}{e^{2 i \pi z} - 1}
\end{align*}\\
We define the Bernoulli numbers $B_m \in \Q$ as the unique sequence satisfying (for $x \in \R$):
\begin{align*}
    \frac{x}{e^x - 1} = \sum_{m=0}^\infty \frac{B_m}{m!} x^m
\end{align*}
Calculating these terms is computationally tedious. It is is true however that $m>1 \text{ odd } \implies B_m = 0$. This follows from the fact that $\frac{x}{e^x - 1} + \frac{x}{2}$ is an even function. Indeed:
\begin{align*}\frac{x}{e^x - 1} + \frac{x}{2} &= \frac{x(1+e^x)}{2(e^x-1)} \\&= - \frac{x}{2} \cdot \frac{1+e^x}{1-e^x}\\
&= - \frac{x}{2} \cdot \frac{e^{-x} + 1}{e^{-x} - 1}\\
&= - \frac{x}{2} \cdot \frac{2+e^{-x} -1}{e^{-x} - 1}\\
&= - \frac{x}{e^{-x}-1} - \frac{x}{2}\\
&=  \frac{(-x)}{e^{(-x)}-1} + \frac{(-x)}{2}
\end{align*}
So this function is even.\\

Now we continue from
\begin{align*}
    1 - 2 \sum_{k=1}^\infty \zeta(2k) \cdot z^{2k} &= i \pi z + \frac{2i \pi z}{e^{2 i \pi z} - 1}\\
    \text{(Bernoulli numbers) }&= i \pi z + \sum_{m=0}^\infty \frac{B_m}{m!} (2 \pi i z)^m\\
    &= \cancel{i \pi z} + 1 - \frac{1}{2}\cancel{(2 \pi i z)} + \sum_{m=2}^\infty \frac{B_m}{m!} (2 \pi i z)^m\\
    \text{(odd $B_m$ vanish) } &= 1 + \sum_{m=1}^\infty \frac{B_{2m}}{(2m)!} (2 \pi i z)^{2m}\\
    &= 1 + (-2) \cdot \br{-\frac{1}{2}} \sum_{m=1}^\infty \frac{B_{2m}}{(2m)!} (2 \pi i z)^{2m}\\
    &= 1 - 2\sum_{m=1}^\infty \frac{B_{2m}}{2 (2m)!} (2 \pi)^{2m} \cdot (-1)^{m+1} z^{2m}
\end{align*}
These power series are equal, thus we can equate coefficients. Thus for all $m \geq 1$ we have that:
\begin{align*}
    \zeta(2m) = (-1)^{m+1} \frac{(2\pi)^{2m}}{2 \cdot (2m)!} B_{2m}
\end{align*}

\end{proof}

\begin{example} Noting that $B_2 = \frac{1}{6}$, we have that $\zeta(2) = \frac{(2\pi)^2}{2\cdot 2}\cdot B_2 = \pi ^ 2 \cdot \frac{1}{6} = \frac{\pi ^2 }{6}$.
\end{example}

\begin{corollary}
$\zeta(-(2n+1))= - \frac{B_{2n+2}}{2n+2}, \; n \geq 1$, with $B_n$ the Bernoulli numbers.
\end{corollary}

\begin{proof}
Note that if $n$ is a negative odd integer, then $1-n$ is a positive even integer. We use the functional equation. Let $m = 2n+1$. Then:
\begin{align*}
    \zeta(-m) &= 2^{-m} \cdot \pi^{-m-1} \cdot \sin\br{\frac{- m \pi}{2}} \cdot \Gamma(1+m) \cdot \zeta(1+m)\\
    &= 2^{-(2n+1)} \cdot \pi^{-2n-2} \cdot (-1)^{n+1} \cdot (2n+1)! \cdot \zeta(2n+2)\\
    \text{(corollary (24.6)) } &= 2^{-(2n+1)} \cdot \pi^{-2n-2} \cdot (-1)^{n+1} \cdot (2n+1)! \cdot \br{(-1)^n \cdot \frac{(2 \pi)^{2n+2}}{2 \cdot (2n+2)!} B_{2n+2} }\\
    \text{(simplify) }&= - \frac{B_{2n+2}}{2n+2}
\end{align*}
\end{proof}

\begin{example}
$\zeta(-1) = - \frac{B_{2}}{2} = - \frac{1}{12}$
\end{example}

What is missing now for $\zeta(\Z)$?

\begin{itemize}
    \item $\zeta(1) = "\infty"$
    \item $\zeta(-2n) = 0$ for $n \geq 1$ (this will be shown later)
    \item $\zeta(2n+1) = ???$ for $n \geq 1$ (this is still a mystery)
\end{itemize}